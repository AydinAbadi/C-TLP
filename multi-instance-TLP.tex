% !TEX root =main.tex




\section{Multi-instance  Time-lock Puzzle}




\subsection{Strawman Solution}\label{C-TLP-overview}



In the following, we elaborate on the  problems that would arise if an existing time-lock puzzle is used directly to handle  multiple puzzles at once.  Without loss of generality, to illustrate the problems, we use the well-known TLP scheme presented in Section \ref{Time-lock-Encryption}. 



Consider the case where a client wants a server to learn a vector of messages: $\vv{\bm{m}}=[m_{\scriptscriptstyle 1},...,m_{\scriptscriptstyle z}]$ at times  $[f_{\scriptscriptstyle 1},...,f_{\scriptscriptstyle z}]$ respectively, where the client is available and online only at an earlier time $f_{\scriptscriptstyle 0}< f_{\scriptscriptstyle 1}$.  For the sake of simplicity, let $\Delta=f_{\scriptscriptstyle 1}-f_{\scriptscriptstyle 0}$ and $\Delta=f_{\scriptscriptstyle j+1}-f_{\scriptscriptstyle j}$, where $1\leq j \leq z$. A naive way to address the problem is that the client uses the TLP  to encrypt each message $m_{\scriptscriptstyle j}$ separately, such that it can be decrypted at time $f_{\scriptscriptstyle j}$ if  all ciphertexts and public keys are passed on to the server at time $t_{\scriptscriptstyle 0}$.  For the server to decrypt the messages  on time, it needs to start decrypting \emph{all of them} as soon as the ciphertexts and public keys are given to it. 



\noindent\textit{\textbf{Parallel Composition Problem}}. The above naive approach yields two serious issues: (a) imposing a high computation cost, as  the server has to perform $S\Delta \sum\limits_{\scriptscriptstyle j=1}^{\scriptscriptstyle z}j$ squaring to decrypt all   messages, and (b) demanding a high level of parallelisation, as each puzzle has to be dealt with separately in parallel to the rest.  The  issues can be cast  as  ``\emph{parallel composition problem}'', where $z$ instances of a puzzle scheme are given at once to a server whose only option, to find solutions on time, is to solve them in parallel.\footnote{It should not be confused with the ``universally composable'' notion put forth in \cite{Canetti01}.} Also, for the client  to efficiently compute $a_{\scriptscriptstyle j}$  for each  message $m_{\scriptscriptstyle j}$,  where $j>1$, it has to perform at least one modular multiplication, i.e. $a_{\scriptscriptstyle j}=a_{\scriptscriptstyle 1} a_{\scriptscriptstyle j-1}=2^{\scriptscriptstyle j  T}$, where $a_{\scriptscriptstyle 1}=2^{\scriptscriptstyle T}$. In this step, in total $z-1$ modular multiplications are required  to compute all $a_{\scriptscriptstyle j}$ values, for $z$ messages (which is not optimal). Note, we do not see the above issues as  previous schemes' flaws, because they were not initially designed for the multi-puzzle setting.  
 
%\vspace{-3mm}



 \subsection{An Overview of our Solutions}\label{Overview-of-our-Solutions}
 
 %\vspace{-2mm}
 
 Our key observation is, in the naive approach, the process of decrypting  messages has many overlaps  leading to a high  computation cost. So,  by removing the overlaps, we can considerably lower the overall cost both in \emph{puzzle solving} and \emph{puzzle creating} phases.  One of our core ideas  is to chain the puzzles. While chaining different puzzles may seem a relatively obvious approach to tackle the  issues, designing a secure protocol that also can make black-box use of a standard time-lock puzzle scheme, supports public verifiability, and has low costs is challenging (we refer readers to Remark \ref{remark::trivial-chaining} for a detailed discussion). In our solution, a client  first encrypts the message  that is supposed to be decrypted after the rest and embeds the information needed for decrypting it into the ciphertext of the message that will be decrypted before that message. In other words, the client integrates the information (i.e. a part of public keys) needed to decrypt message $m_{\scriptscriptstyle j}$ into the ciphertext related to message $m_{\scriptscriptstyle j-1}$. In this case, the server after learning message $m_{\scriptscriptstyle j-1}$ at time $f_{\scriptscriptstyle j-1}$ learns the public key needed to perform the sequential squaring to decrypt the next message: $m_{\scriptscriptstyle j}$. This means after fully decrypting $m_{\scriptscriptstyle j-1}$, the server starts  squaring sequentially to decrypt $m_{\scriptscriptstyle j}$
  
  
  
  \noindent\textit{\textbf{Addressing Parallel Composition Problem}}.  The above approach solves the parallel composition problem for two main reasons.  First, the total  number of squaring required to decrypt all $z$ messages is now much lower, i.e. $S \Delta z$, and is equivalent to the number of squaring needed to solve only the last puzzle,  i.e. $z\text{-th}$ one. Second, it does not call for  high parallelisation. Because now the server does not need to deal with all of the puzzles in parallel; instead, it solves them sequentially one after another.  
  
  
  
  \noindent\textit{\textbf{Adding Efficient Publicly Verifiable Algorithm}}. To let the  scheme  support  efficient public verifiability, we use the following novel trick. The client uses a commitment scheme to commit to every message: $m_{\scriptscriptstyle i}$ and publishes the  commitment. Then, it uses the time-lock encryption to encrypt the commitment's opening, i.e. a combination of $m_{\scriptscriptstyle i}$ and a random value. But, unlike the traditional commitment, the client does not open the commitment itself. Instead, the server does that, after it discovers the puzzle's solution.  When it finds a solution, it decodes the solution to find the opening and sends it to the public who can check the solution correctness. So, to verify  a solution's correctness,   a verifier  only needs to run the commitment's verification algorithm that is: (a)  publicly verifiable, and (b)   efficient. It can be built in the random oracle  or  the standard model.
  
    The approach also allows  the client at the setup to compute only a single $a=2^{\scriptscriptstyle T}$  reusable for all $z$ puzzles, imposing only $O(1)$  cost. 

%\vspace{-5mm}

\subsection{Multi-instance   Time-lock Puzzle Definition}\label{Section::Multi-instance-Time-lock Puzzle-Definition}

%\vspace{-1mm}

In this section, we provide a formal definition of a multi-instance time-lock puzzle. Our starting point is the  time-lock puzzle definition, i.e. Definition \ref{Def::Time-lock-Puzzle}, but we extend it from several  perspectives, so it can: (a) handle multiple  solutions/messages in setup, (b)  produce multiple puzzles for the messages,   (c) solve the puzzles given the puzzles and public parameters, and (d) support public verifiability. In the following, we provide the formal definition of a multi-instance  time-lock puzzle.
\begin{definition}[Multi-instance Time-lock Puzzle] A multi-instance time-lock puzzle has the following  five algorithms and satisfies completeness and efficiency properties. 



\begin{itemize}[leftmargin=.43cm]
\item \textbf{Algorithms}:
\begin{itemize} 
\item[$\bullet$]$\mathtt{Setup}(1^{\scriptscriptstyle\lambda},\Delta,z)\rightarrow (pk,sk,\vv{\bm{d}})$:  a probabilistic algorithm that takes as input  security: $1^{\scriptscriptstyle\lambda}$ and time:  $\Delta$ parameters and the total number of solutions/puzzles: $z$. Let     $j \Delta$ be a time period after which $j\text{\small{-th}}$ solution is found.   It outputs public-private key pair: $(pk,sk)$ and a vector of fixed size  secret witnesses: $\vv{\bm{d}}$


%\vv{\bm{s}}
\item[$\bullet$]$\mathtt {GenPuz}(\vv{\bm{m}}, pk, sk,\vv{\bm{d}})\rightarrow \ddot{o}$:  a probabilistic algorithm that takes as  input  a  message vector: $\vv{\bm{m}}=[m_{\scriptscriptstyle 1},...,m_{\scriptscriptstyle z}]$,  the public-private key pair: $(pk,sk)$, and the witness vector: $\vv{\bm{d}}$. It  outputs $\ddot{o}:(\vv{\bm{o}},\vv{\bm{h}})$, where $\vv{\bm{o}}$ is a puzzle vector, and $\vv{\bm{h}}$ is a commitment vector. Each $j\text{\small{-th}}$ element in  vectors $\vv{\bm{o}}$ and $\vv{\bm{h}}$ corresponds to a solution $s_{\scriptscriptstyle j}$ of the form: $s_{\scriptscriptstyle j}=m_{\scriptscriptstyle j}||d_{\scriptscriptstyle j}$ %Given $s_{\scriptscriptstyle j}$ and $b$, a public decoding function, $\mathtt{Decode}()$ returns $m_{\scriptscriptstyle j}$, i.e. $\mathtt{Decode}(s_{\scriptscriptstyle j},b)\rightarrow m_{\scriptscriptstyle j}$. 
 
\item[$\bullet$]$\mathtt {SolvPuz}(pk,\vv{\bm{o}})\rightarrow \vv{\bm{s}}$:   a deterministic algorithm that takes as input  the public key: $pk$ and  puzzle vector: $\vv{\bm{o}}$. It outputs a solution vector: $\vv{\bm{s}}$

\item[$\bullet$]$\mathtt {Prove}(pk,s_{\scriptscriptstyle j})\rightarrow \ddot{p}_{\scriptscriptstyle j}$:  a deterministic algorithm that takes the public key: $pk$ and a solution: $s_{\scriptscriptstyle j}\in\vv{\bm{s}}$. It outputs a proof, $\ddot{p}_{\scriptscriptstyle j}:(m_{\scriptscriptstyle j},d_{\scriptscriptstyle j})$ given to the verifier.

\item[$\bullet$]$\mathtt {Verify}(pk,\ddot{p}_{\scriptscriptstyle j},h_{\scriptscriptstyle j})\rightarrow \{0,1\}$:  a deterministic algorithm that takes  public key: $pk$,  proof: $\ddot{p}_{\scriptscriptstyle j}$ and commitment: $h_{\scriptscriptstyle j}\in \vv{\bm{h}}$. It outputs  $0$ if it rejects, or $1$ if it accepts. 
\end{itemize}
\item \textbf{Completeness}: for any honest prover and verifier, it always holds that: 
\begin{itemize}
\item$\mathtt{SolvPuz}(pk,[o_{\scriptscriptstyle 1},...,o_{\scriptscriptstyle j}])=[s_{\scriptscriptstyle1},...,s_{\scriptscriptstyle j}]$, for every $j$, $1\leq j\leq z$

\item $\mathtt {Verify}(pk,\mathtt {Prove}(pk,s_{\scriptscriptstyle j}),h_{\scriptscriptstyle j})\rightarrow 1$
\end{itemize}
\item \textbf{Efficiency}: the run-time of algorithm $\mathtt {SolvPuz}(pk,[o_{\scriptscriptstyle 1},...,o_{\scriptscriptstyle j}])=[s_{\scriptscriptstyle1},...s_{\scriptscriptstyle j}]$ is bounded by:  $ poly(j\Delta,\lambda)$, where $poly(.)$ is a fixed polynomial and  $1\leq j\leq z$
\end{itemize}
\end{definition}
 
Informally, a multi-instance time-lock puzzle is secure if it satisfies two properties:  a solution's \emph{privacy} and  \emph{validity}. The former  requires  its $j\text{\small{-th}}$ solution   to remain hidden from all adversaries running in parallel within  time period: $j \Delta$, while the latter one requires that it is  infeasible for  a PPT adversary to come up with an invalid solution  and passes the verification. The two properties are formally defined in Definitions \ref{Def::Solution-Privacy} and \ref{Def::Solution-Validity}.
 

 
 
 
% \begin{definition}[Chained Time-lock Puzzle's Sequentiality] For functions  $\pi(t)$ and $\delta(t)$, a  chained time-lock puzzle is $(\pi,\delta)$-sequential if for any pair of randomised algorithm $\mathcal{A} : (\mathcal{A}_{\scriptscriptstyle 1},\mathcal{A}_{\scriptscriptstyle 2})$, where $\mathcal{A}_{\scriptscriptstyle 1}$ runs in total time $O(poly(t,\lambda))$ and $\mathcal{A}_{\scriptscriptstyle 2}$ runs in  time $\delta(t)$ using at most $\pi(t)$ parallel processors, there exists a negligible function $\mu(.)$ such that: 
% 
 
 
% $$ Pr\left[    \begin{array}{l}  \mathcal{A}_{\scriptscriptstyle 2}(pk, \ddot{o},state)\rightarrow s \\
% s.t.\\
% s=\mathtt {SolvPuz}(pk,\theta)
% 
%   \end{array}
%   \middle |
%    \begin{array}{l}
%\mathtt{Setup}(1^{\scriptscriptstyle\lambda},\Delta,1)\rightarrow (pk,sk,\vv{\bm{d}})\\
%\mathcal{A}_{\scriptscriptstyle 1}(1^{\scriptscriptstyle\lambda},pk, \Delta,1)\rightarrow state\\
%m\stackrel{\scriptscriptstyle\$}\leftarrow \mathcal{M}\\
%\mathtt {GenPuz}(m, pk, sk)\rightarrow \ddot{o}\\
%\end{array}    \right]\leq \mu(\lambda)$$
%  \end{definition}
%  where $\theta\in \ddot{o}$.
  
%   $$ Pr\left[    \begin{array}{l}  \mathcal{A}_{\scriptscriptstyle 2}(pk, \ddot{o},state)\rightarrow a \\
% s.t.\\
%m'=\mathtt{Decode}(\mathtt {SolvPuz}(pk,\theta),b)\\
% a=m'
% 
%   \end{array}
%   \middle |
%    \begin{array}{l}
%\mathtt{Setup}(1^{\scriptscriptstyle\lambda},\Delta,1)\rightarrow (pk,sk,\vv{\bm{d}})\\
%\mathcal{A}_{\scriptscriptstyle 1}(1^{\scriptscriptstyle\lambda},pk, \Delta,1)\rightarrow state\\
%m\stackrel{\scriptscriptstyle\$}\leftarrow \mathcal{M}\\
%\mathtt {GenPuz}(m, pk, sk)\rightarrow \ddot{o}\\
%\end{array}    \right]\leq \mu(\lambda)$$
%  \end{definition}
%  where $\theta\in \ddot{o}$.
%  
%  
%  
%  xxx The above definition also captures the sequentiality for a single solution as well that means the adversary cannot find a single solution significantly less than required steps. 
%  
 
   %\vspace{-1mm}
  
\begin{definition}[Multi-instance Time-lock Puzzle's Solution-Privacy]\label{Def::Solution-Privacy} A multi-instance time-lock puzzle  is privacy-preserving  if for all $\lambda$ and  $\Delta$,  any number of puzzle: $z\geq1$, any $j$ (where $1\leq j \leq z$), any pair of randomised algorithm $\mathcal{A} : (\mathcal{A}_{\scriptscriptstyle 1},\mathcal{A}_{\scriptscriptstyle 2})$, where $\mathcal{A}_{\scriptscriptstyle 1}$ runs in  time $O(poly(j\Delta,\lambda))$ and $\mathcal{A}_{\scriptscriptstyle 2}$ runs in  time $\delta(j\Delta)<j\Delta$ using at most $\pi(\Delta)$ parallel processors, there exists a negligible function $\mu(.)$, such that: 
\small{
$$ Pr\left[  \begin{array}{l} 
 \mathcal{A}_{\scriptscriptstyle 2}(pk,\ddot{o},\text{state})\rightarrow \ddot{a}\\
 \text{s.t.}\\
\ddot{a}:(b_{\scriptscriptstyle i},i)\\
  m_{\scriptscriptstyle b_{\scriptscriptstyle i},i}=m_{\scriptscriptstyle b_{\scriptscriptstyle j},j} 
  \end{array}
 \middle |
    \begin{array}{l}
\mathtt{Setup}(1^{\scriptscriptstyle\lambda},\Delta,z)\rightarrow (pk,sk,\vv{\bm{d}})\\
%\mathcal{A}_{\scriptscriptstyle 1}(1^{\scriptscriptstyle\lambda},pk,z)\rightarrow ([(m_{\scriptscriptstyle 0,1},m_{\scriptscriptstyle 1,1}),...,(m_{\scriptscriptstyle 0,z},m_{\scriptscriptstyle 1,z})],state)\\
\mathcal{A}_{\scriptscriptstyle 1}(1^{\scriptscriptstyle\lambda},pk,z)\rightarrow (\vv{\bm{m}},\text{state})\\

\forall j', 1\leq j' \leq z: b_{\scriptscriptstyle j'}\stackrel{\scriptscriptstyle\$}\leftarrow \{0,1\}\\
%\left[b_{\scriptscriptstyle 1},...,b_{\scriptscriptstyle z}\right], b_{\scriptscriptstyle j}\stackrel{\scriptscriptstyle\$}\leftarrow \{0,1\}\\
%\mathtt {GenPuz}((m_{\scriptscriptstyle b_{\scriptscriptstyle 1},\scriptscriptstyle 1},..., m_{\scriptscriptstyle b_{\scriptscriptstyle z},\scriptscriptstyle z}), pk, sk,\vv{\bm{d}})\rightarrow \ddot{o}\\
\mathtt {GenPuz}(\vv{\bm{m}}', pk, sk,\vv{\bm{d}})\rightarrow \ddot{o}\\
\end{array}    \right]\leq \frac{1}{2}+\mu(\lambda)$$
}
where  $\vv{\bm{m}}: [(m_{\scriptscriptstyle 0,1},m_{\scriptscriptstyle 1,1}),...,(m_{\scriptscriptstyle 0,z},m_{\scriptscriptstyle 1,z})]$, $\vv{\bm{m}}':(m_{\scriptscriptstyle b_{\scriptscriptstyle 1},\scriptscriptstyle 1},..., m_{\scriptscriptstyle b_{\scriptscriptstyle z},\scriptscriptstyle z})$,  and $1\leq i\leq z$
%$b_{\scriptscriptstyle j'}\in \left[b_{\scriptscriptstyle 1},...,b_{\scriptscriptstyle z}\right]$. 
\end{definition}

%all probabilistic polynomial time adversaries $\mathcal{A}=(\mathcal{A}_{\scriptscriptstyle 1},\mathcal{A}_{\scriptscriptstyle 2})$ whose run-time is  bounded by  $T_{\scriptscriptstyle j}=j\cdot poly(\lambda,\Delta)$, where  $j\in [ 1,z]$, 


The  definition above also ensures  the  solutions to appear after $j\text{\small{-th}}$ one,  remain hidden from the adversary with a high probability, as well. Similar to \cite{BonehBBF18,MalavoltaT19,garay2019}, it captures that even if     $\mathcal{A}_{\scriptscriptstyle 1}$ computes on the public parameters for a polynomial time,  $\mathcal{A}_{\scriptscriptstyle 2}$  cannot find $j\text{\small{-th}}$  solution in time $\delta(j\Delta)<j\Delta$ using $\pi(\Delta)$ parallel processors, with a probability significantly greater than $\frac{1}{2}$. As highlighted in  \cite{BonehBBF18}, we can set $\delta(\Delta)=(1-\epsilon)\Delta$ for a small  $\epsilon$, where $0<\epsilon<1$
\begin{definition}[Multi-instance Time-lock Puzzle's Solution-Validity]\label{Def::Solution-Validity}
A multi-instance time-lock puzzle preserves a   solution validity,   if  for all $\lambda$ and  $\Delta$,  any number of puzzles: $z\geq1$, all probabilistic polynomial-time adversaries $\mathcal{A}=(\mathcal{A}_{\scriptscriptstyle 1},\mathcal{A}_{\scriptscriptstyle 2})$ that run in  time $O(poly(\Delta,\lambda))$ there is  negligible function $\mu(.)$, such that: 
\small{
$$ Pr\left[
    \begin{array}{l}
 \mathcal{A}_{\scriptscriptstyle 2}(pk,\vv{\bm{s}}, \ddot{o},\text{state})\rightarrow a\\ 
 
 \text{s.t.}\\ 
a:(j,\ddot{p}_{\scriptscriptstyle j} ,\ddot{p}')\\
 \ddot{p}_{\scriptscriptstyle j}: (m_{\scriptscriptstyle j},d_{\scriptscriptstyle j}), 
\ddot{p}':(m',d') \\
 m_{\scriptscriptstyle j}\in \vv{{\bm{m}}}, d_{\scriptscriptstyle j}\in\vv{{\bm{d}}},
m\neq m'\\
\mathtt {Verify}(pk,\ddot{p},h_{\scriptscriptstyle j})= 1\\
\mathtt {Verify}(pk,\ddot{p}',h_{\scriptscriptstyle j})= 1\\
\end{array} 
\middle |
\begin{array}{l}

\mathtt{Setup}(1^{\scriptscriptstyle\lambda},\Delta,z)\rightarrow (pk,sk,\vv{\bm{d}})\\
\mathcal{A}_{\scriptscriptstyle 1}(1^{\scriptscriptstyle\lambda},pk, \Delta,z)\rightarrow (\vv{{\bm{m}}},\text{state})\\

\mathtt {GenPuz}(\vv{{\bm{m}}}, pk, sk,\vv{\bm{d}})\rightarrow \ddot{o} \\
\mathtt {SolvPuz}(pk,\vv{\bm{o}})\rightarrow \vv{\bm{s}}

\end{array} 
   \right]\leq  \mu(\lambda)$$
   }
where $\vv{{\bm{m}}}=[m_{\scriptscriptstyle 1},...,m_{\scriptscriptstyle z}]$, and $h_{\scriptscriptstyle j}\in \vv{\bm{h}}\in \ddot{o}$
\end{definition}


%In Definition \ref{Def::Solution-Validity}, we do not need to bound the adversaries' parallel computation power, as it does not need to solve any puzzles, in fact  puzzles' solutions are provided to them. Therefore, they can run in polynomial time $O(poly(\Delta,\lambda))$.
% 
\begin{definition}[Multi-instance Time-lock Puzzle Security]\label{def::C-TLP-security} A multi-instance time-lock puzzle scheme  is secure if it meets solution-privacy and solution-validity properties. 
\end{definition}

%\vspace{-6mm}

\subsection{Chained  Time-lock Puzzle (C-TLP) Protocol}\label{Section::C-TLP-protocol}

%\vspace{-1mm}

In this section, we present the chained  time-lock puzzle (C-TLP), an instantiation of the multi-instance time lock puzzle. Since we have already presented an outline of C-TLP (in Section \ref{Overview-of-our-Solutions}), in this section we present C-TLP protocol in detail.  Recall, a client wants a server to learn a vector of messages: $\vv{\bm{m}}=[m_{\scriptscriptstyle 1},...,m_{\scriptscriptstyle z}]$ at times  $[f_{\scriptscriptstyle 1},...,f_{\scriptscriptstyle z}]$ respectively, where the client is available and online only at an earlier time $f_{\scriptscriptstyle 0}< f_{\scriptscriptstyle 1}$.  Also, the client wants to ensure that anyone can validate a solution found by the  server, i.e. supports public verifiability. For the sake of simplicity, let $\Delta=f_{\scriptscriptstyle 1}-f_{\scriptscriptstyle 0}$ and $\Delta=f_{\scriptscriptstyle j+1}-f_{\scriptscriptstyle j}$, where $1\leq j \leq z$ and $T=S \Delta$. Below, we provide C-TLP protocol. We refer readers to Appendix \ref{discussion-C-TLP} for further remarks on the protocol. 

% !TEX root =main.tex



  %\begin{figure}

%\centering

%\small{

%\begin{boxedminipage}{\columnwidth}


\begin{enumerate}[leftmargin=.4cm]

\item\textbf{Setup}: $\mathtt{Setup}(1^{\scriptscriptstyle\lambda}, \Delta,z)$.  
\begin{enumerate}

\item\label{call-RTLP-Setup} Call:  $\mathtt{TLP.Setup}(1^{\scriptscriptstyle\lambda}, \Delta)\rightarrow (\hat{pk},\hat{sk})$, s.t.  $\hat{pk}=(N,T,r_{\scriptscriptstyle 1})$ and $\hat{sk}=(q_{\scriptscriptstyle 1},q_{\scriptscriptstyle 2},a,k_{\scriptscriptstyle 1})$

\item Pick  $z-1$ fixed size  random generators: $\vv{\bm{r}}=[r_{\scriptscriptstyle 2},...,r_{\scriptscriptstyle z}]$ from $\mathbb{Z}^{\scriptscriptstyle *}_{ \scriptscriptstyle N}$


\item Pick  $z-1$ random keys: $[k_{\scriptscriptstyle 2},...,k_{\scriptscriptstyle z}]$ for a symmetric key encryption. Let $\vv{\bm{k}}=[k_{\scriptscriptstyle 1},...,k_{\scriptscriptstyle z}]$, where $k_{\scriptscriptstyle 1}\in \hat{sk}$. Also, pick $z$ fixed size sufficiently large random values: $\vv{\bm{d}}=[d_{\scriptscriptstyle 1},...,d_{\scriptscriptstyle z}]$, e.g. $|d_{\scriptscriptstyle j}|=128$-bit or $1024$-bit depending on the choice of a commitment scheme.  

\item Set $pk=( \text{aux},N,T, r_{\scriptscriptstyle 1})$ as  public key. Set $sk=(q_{\scriptscriptstyle 1},q_{\scriptscriptstyle 2},a, \vv{\bm{k}},\vv{\bm{r}},\vv{\bm{d}})$ as secret key. Note, $\text{aux}$ contains a cryptographic hash function's description and the size of the random values. Also,  note that  all generators, except $r_{\scriptscriptstyle 1}$ are kept secret. Output $pk$ and $sk$

\end{enumerate}
\item\label{Generate-Puzzle}\textbf{Generate Puzzle}: $\mathtt{GenPuz}(\vv{\bm{m}}, pk,sk)$ 


Encrypt the  messages, starting with $j=z$, in descending order. $\forall j, z\geq j \geq 1:$
\begin{enumerate}
\item\label{set-pk-in-loop} Set $pk_{\scriptscriptstyle j}=(N,T,r_{\scriptscriptstyle j})$ and $sk_{\scriptscriptstyle j}=(q_{\scriptscriptstyle 1},q_{\scriptscriptstyle 2},a,k_{\scriptscriptstyle j})$. Note, if $j=1$ then $r_{\scriptscriptstyle j} \in pk$; otherwise (when $j>1$), $r_{\scriptscriptstyle j} \in \vv{\bm{r}}$


\item\label{call-RTLP-GenPuz} Generate a puzzle (or ciphertext pair): 
\begin{itemize}
\item[$\bullet$]  if $j=z$, then run: $\mathtt{TLP.GenPuz}(m_{\scriptscriptstyle j}||d_{\scriptscriptstyle j},pk_{\scriptscriptstyle j},sk_{\scriptscriptstyle j})\rightarrow \ddot{o}_{\scriptscriptstyle j}=(o_{\scriptscriptstyle j,1},o_{\scriptscriptstyle j,2})$
 
\item[$\bullet$]  otherwise, run: $\mathtt{TLP.GenPuz}(m_{\scriptscriptstyle j}||d_{\scriptscriptstyle j}||r_{\scriptscriptstyle j+1},pk_{\scriptscriptstyle j},sk_{\scriptscriptstyle j})\rightarrow \ddot{o}_{\scriptscriptstyle j}=(o_{\scriptscriptstyle j,1},o_{\scriptscriptstyle j,2})$
\end{itemize}
%Recall that, in TLP,  $\ddot{o}_{\scriptscriptstyle j}=(o_{\scriptscriptstyle j,1},o_{\scriptscriptstyle j,2})$


\item\label{commit-} Commit to each message, e.g. $\mathtt{H}(m_{\scriptscriptstyle j}||d_{\scriptscriptstyle j})=h_{\scriptscriptstyle j}$ and output:  $h_{\scriptscriptstyle j}$ 
 
\item Output: $\ddot{o}_{\scriptscriptstyle j}=(o_{\scriptscriptstyle j,1},o_{\scriptscriptstyle j,2})$ as puzzle (or ciphertext pair). 

\end{enumerate}
By the end of this phase,  vectors of puzzles: $\vv{\bm{o}}=[\ddot{o}_{\scriptscriptstyle 1},..., \ddot{o}_{\scriptscriptstyle z}]$ and commitments: $\vv{\bm{h}}=[h_{\scriptscriptstyle 1},...h_{\scriptscriptstyle z}]$ are generated. All public parameters and puzzles are given to a server at time $t_{\scriptscriptstyle 0}<t_{\scriptscriptstyle1}$, where  $\Delta=f_{\scriptscriptstyle 1}-f_{\scriptscriptstyle 0}$ %The public parameters and $\vv{\bm{h}}$ are given to public verifiers.




\item\textbf{Solve Puzzle}:  $\mathtt{SolvPuz}(pk, \vv{\bm{o}})$ 

 Decrypt the messages, starting with $j=1$, in ascending order.  $\forall j, 1\leq j\leq z:$
 
\begin{enumerate}
\item If $j=1$, then set $r_{\scriptscriptstyle j}=r_{\scriptscriptstyle 1}$, where $r_{\scriptscriptstyle 1}\in pk$; Otherwise, set $r_{\scriptscriptstyle j}=u$

\item Set  $pk_{\scriptscriptstyle j}=(N,T,r_{\scriptscriptstyle j})$

\item\label{call-RTLP-SolvPuz} Run: $\mathtt{TLP.SolvPuz}(pk_{\scriptscriptstyle j},\ddot{o}_{\scriptscriptstyle j})\rightarrow x_{\scriptscriptstyle j}$, where $\ddot{o}_{\scriptscriptstyle j}\in \vv{\bm{o}}$


\item\label{Dec-message} Parse $x_{\scriptscriptstyle j}$. Note that if $j<z$ then $x_{\scriptscriptstyle j}=m_{\scriptscriptstyle j}||d_{\scriptscriptstyle j}||r_{\scriptscriptstyle j+1}$; otherwise,  we have $x_{\scriptscriptstyle j}=m_{\scriptscriptstyle j}||d_{\scriptscriptstyle j}$. Therefore, $x_{\scriptscriptstyle j}$ is parsed as follows.




\begin{itemize}
\item[$\bullet$] if $j<z: $
\begin{enumerate}

\item Parse $m_{\scriptscriptstyle j}||d_{\scriptscriptstyle j}||r_{\scriptscriptstyle j+1}$ into  $m_{\scriptscriptstyle j}||d_{\scriptscriptstyle j}$ and $u=r_{\scriptscriptstyle j+1}$
\item Output $s_{\scriptscriptstyle j}=m_{\scriptscriptstyle j}||d_{\scriptscriptstyle j}$
\end{enumerate}
\item[$\bullet$] otherwise (when $j=z$),  output $s_{\scriptscriptstyle j}=x_{\scriptscriptstyle j}=m_{\scriptscriptstyle j}||d_{\scriptscriptstyle j}$


\end{itemize}

\end{enumerate}

\item\label{prove-} \textbf{Prove}:  $\mathtt{Prove}(pk, s_{\scriptscriptstyle j})$. Parse  $s_{\scriptscriptstyle j}$ into $\ddot{p}_{\scriptscriptstyle j}: (m_{\scriptscriptstyle j},d_{\scriptscriptstyle j})$, and send the pair to the verifier. 
\item\label{verify-} \textbf{Verify}: $\mathtt{Verify}(pk,\ddot{p}_{\scriptscriptstyle j}, h_{\scriptscriptstyle j})$. Verifies the commitment,  $\mathtt{H}(m_{\scriptscriptstyle j},d_{\scriptscriptstyle j})\stackrel{\scriptscriptstyle?}=h_{\scriptscriptstyle j}$. If passed, accept the solution and output $1$; otherwise,  reject it and output $0$



\end{enumerate}



%\end{boxedminipage}
%}

%\caption{Chained  Time-lock Puzzle (C-TLP) Scheme} 
%\label{fig:CTE}
%\end{figure}
\vspace{-3mm}


%\vspace{-3mm}

   \begin{theorem}[C-TLP Security]\label{C-TLP-Sec}  C-TLP  is a secure multi-instance time-lock puzzle. 
   \end{theorem}


\begin{proof}[Outline]
The proof of Theorem \ref{C-TLP-Sec} relies on the security of the TLP, symmetric key encryption, and commitment schemes. It is also based on the fact that the probability to  find a certain random generator is negligible. It shows both C-TLP's solution privacy (due to security of the above three schemes) and validity (due to the security of the commitment) are satisfied.  We refer readers to Appendix \ref{CR-TLP-Proof} for  detailed proof.  \hfill\(\Box\)
\end{proof}

\vspace{-2.5mm}
%
%\begin{remark} Recall, to make each  puzzle instance, a distinct random generator: $r_{\scriptscriptstyle j}$, is used. This is the reason, in  Fig. \ref{fig:CTE}, before a puzzle  is  generated   in step \ref{call-RTLP-GenPuz},  a new public key is set in step \ref{set-pk-in-loop}.  Also, at the beginning of the protocol only $r_{\scriptscriptstyle 1}$ is public and the rest of the generators are kept secret. They are found and used sequentially after their related puzzle is solved. 
%\end{remark}
%
%
%\begin{remark} The commitments opening, including the commitment random values, are not known to other verifiers (than the puzzle generator) at the beginning of the protocol. At this point,  only the committed values are public. Once a solver solves each puzzle,  it  extracts one of the commitments' opening, and sends it to a public verifier who can check if the opening matches the commitment.  
%\end{remark}
%
%\begin{remark}
% In Fig. \ref{fig:CTE}, we use the folklore hash-based commitment scheme, in the random oracle model, only to achieve more computation improvement than that can be achieved in the standard model. But C-TLP can utilise any efficient non-interactive commitment scheme in the \emph{standard model} as well, e.g. Pedersen Commitment.
%\end{remark}
%
%
%\begin{remark}
%The efficiency of  C-TLP scheme stems from three crucial factors: (a) removing computation overlaps when solving different puzzles: even though solving $j\text{\small{-th}}$ puzzle, where $j>1$, requires $jT$ squaring, $(j-1) T$ of the squaring is used to solve previous puzzles that leads to $\frac{z+1}{2}$ times computation cost reduction at the server-side,  (b)  supporting reusable single  public parameter: $a=2^{\scriptscriptstyle T}$, generated only once that costs $O(1)$, as opposed to the RSA TLP whose cost is linear: $O(z)$, and (c) supporting efficient verification: due to the way each message is encoded (i.e. embedding the opening in a solution). 
%\end{remark}
%
%\begin{remark}
%C-TLP also can efficiently  be used in a multi-server setting,  where there are $z$ servers: $\{S_{\scriptscriptstyle 1},...,S_{\scriptscriptstyle z}\}$,  each $S_{\scriptscriptstyle j}$ needs to solve puzzle $\ddot{o}_{\scriptscriptstyle j}$ at time $f_{\scriptscriptstyle j}$ and passes on the solution to the next server $S_{\scriptscriptstyle j+1}$ to solve the next puzzle by time $f_{\scriptscriptstyle j+1}>f_{\scriptscriptstyle j}$. In this setting,  due to the scalability property of C-TLP (and unlike using the existing time-lock puzzles naively), other servers do not need to start solving the puzzle   as soon as the client releases puzzles public parameters. Instead, they can wait until the previous solution is issued that saves them significant cost. Furthermore, a  server can first  verify the correctness  of the solution found by the previous server (due to the public verifiability of C-TLP),  if accepted then   it starts finding the next solution. 
%\end{remark}
%
%
%\begin{remark}
%In the following, we outline an  approach that looks an option to construct an efficient C-TLP; however, as we will show it would not be secure. In particular,  one uses the TLP to generate $z$ public and secret key pairs. Then, it uses the TLP to compute $z\text{\small{-th}}$ puzzle as $\mathtt{TLP.GenPuZ}(m_{\scriptscriptstyle z},pk_{\scriptscriptstyle z},sk_{\scriptscriptstyle z})\rightarrow \ddot{o}_{\scriptscriptstyle z}$.  Then, it embeds $\ddot{o}_{\scriptscriptstyle z}$ into $(z-1)\text{\small{-th}}$ one, i.e. $\mathtt{TLP.GenPuZ}(m_{\scriptscriptstyle z-1}||\ddot{o}_{\scriptscriptstyle z},pk_{\scriptscriptstyle z-1},sk_{\scriptscriptstyle z-1})\rightarrow \ddot{o}_{\scriptscriptstyle z-1}$. This process goes on until $\ddot{o}_{\scriptscriptstyle 1}$  is created. It sends the combined puzzles and public key (including all random generators) to the server; with the hope that puzzles can be solved sequentially and the time gap between finding two solutions will be $\Delta$.  This approach is not secure, because as soon as the server accesses $\ddot{o}_{\scriptscriptstyle 1}$ and public parameters, it can in parallel perform $T$ squaring on every generator, i.e. $r^{\scriptscriptstyle 2^{\scriptscriptstyle T}}_{\scriptscriptstyle i}$, for all $i, 1\leq i\leq z$. In this case, as soon as $\ddot{o}_{\scriptscriptstyle 1}$ is solved and  $\ddot{o}_{\scriptscriptstyle 2}$  is extracted, it has enough information to  immediately solve $\ddot{o}_{\scriptscriptstyle 2}$ and accordingly the rest of the puzzles without doing any further exponentiation. 
 %\end{remark}
 
 
 % !TEX root =main.tex


\vspace{-5mm}
 
 \subsection{Cost Analysis Table}\label{TLP-cost-compare}
 
 We summarize C-TLP's cost analysis in Table \ref{table::puzzle-com}. It  considers a generic setting where the protocol deals with $z$ puzzles. We refer readers to Appendix \ref{TLP-cost-compare} for detailed  analysis. 
 
 \vspace{-7mm}
 
 \begin{table*}[!htbp]
\begin{footnotesize}

\begin{center}
\caption{ \small C-TLP's Detailed Cost Breakdown}\label{table::puzzle-com} 

\renewcommand{\arraystretch}{.85}
\scalebox{0.94}{
\begin{subtable}{.64\linewidth}%xxxx
\begin{minipage}{.88\linewidth}
\caption{\small Computation Cost}
\begin{tabular}{|c|c|c|c|c|c|c|c|c|c|c|c|c|c|c|} 
   \hline
\cellcolor[gray]{0.9}&\cellcolor[gray]{0.9} &
 \multicolumn{3}{c|}{\cellcolor[gray]{0.9}\scriptsize \underline{ \ \ \ \  \ \ \ \ \ Protocol Function \ \ \ \ \ \ \ \ \ }}&\cellcolor[gray]{0.9}\\
 %\cline{3-5}
\cellcolor[gray]{0.9} \multirow{-2}{*}{\scriptsize Protocol}&\cellcolor[gray]{0.9} \multirow{-2}{*} {\scriptsize Operation}&\cellcolor[gray]{0.9}\scriptsize$\mathtt{GenPuz}$&\cellcolor[gray]{0.9}\scriptsize$\mathtt{SolvPuz}$&\cellcolor[gray]{0.9}\scriptsize$\mathtt{Verify}$&\multirow{-2}{*} {\cellcolor[gray]{0.9}\scriptsize   Complexity} \\
\hline
\cellcolor[gray]{0.9} &\multirow{3}{*}{\rotatebox[origin=c]{0}{\scriptsize }} \cellcolor[gray]{0.9}\scriptsize Exp.&\scriptsize$z+1$&\scriptsize$T z$ &$-$&\multirow{4}{*}{\rotatebox[origin=c]{0}{\scriptsize $O(T  z)$}}\\
     \cline{2-5}  
 \cellcolor[gray]{0.9}     &\cellcolor[gray]{0.9}\scriptsize Add. or Mul.&\scriptsize$z$ &\scriptsize$z$&$-$ & \\
     \cline{2-5} 
 \cellcolor[gray]{0.9}         &\cellcolor[gray]{0.9}\scriptsize Commitment&\scriptsize$z$&$-$ &\scriptsize$z$&\\
     \cline{2-5} 
\cellcolor[gray]{0.9}   \multirow{-4}{*}{\rotatebox[origin=c]{0}{\scriptsize  C-TLP }}     &\cellcolor[gray]{0.9}\scriptsize Sym. Enc&\scriptsize$z$&\scriptsize$z$ &$-$&\\

 \hline
\end{tabular}
\end{minipage}%******
\end{subtable}%xxxx

\begin{subtable}{.46\linewidth}%xxxx
\renewcommand{\arraystretch}{1.68}
\begin{minipage}{.9\linewidth}
\caption{\small Communication Cost (in bit)}
\begin{tabular}{|c|c|c|c|c|c|c|c|c|c|c|c|c|c|c|} 
   \hline
 {\cellcolor[gray]{0.9}\scriptsize Protocol}&{\cellcolor[gray]{0.9}\scriptsize Model}&
{\cellcolor[gray]{0.9}\scriptsize Client}&{\cellcolor[gray]{0.9}\scriptsize Server}&{\cellcolor[gray]{0.9}\scriptsize  Complexity}\\
 \cline{3-4}

\hline
 \cellcolor[gray]{0.9}  &\cellcolor[gray]{0.9} \multirow{2}{*}{\rotatebox[origin=c]{0}}\scriptsize Standard&\scriptsize$3200 z$&\scriptsize$1524 z$ &\multirow{2}{*}{\rotatebox[origin=c]{0}{\scriptsize $O(z)$ }}\\
     \cline{2-4}  
  \multirow{-2}{*}{\rotatebox[origin=c]{0}{\cellcolor[gray]{0.9} \scriptsize  C-TLP }}&\cellcolor[gray]{0.9}\scriptsize R.O.&\scriptsize$2432  z$ &\scriptsize$628  z$& \\
   
 \hline
\end{tabular}
\end{minipage}%******
\end{subtable}%xxxx
}
\end{center}
\end{footnotesize}
\end{table*}

 \vspace{-3mm}

% \noindent\textbf{\textit{Computation Complexity}}. For a client to generate $z$ puzzles, it performs $z$ symmetric key-based encryption,  $z$ modular exponentiations  and $z$ modular additions. Also, to commit to values, it invokes a commitment scheme $z$ times; if a hash-based commitment is used then it would involve $z$ invocations of a hash function, and if Pedersen commitment is used then it would involve $2 z$ exponentiations and $z$ multiplications. Thus, the overall computation complexity of the client   is $O(z)$. For the server to solve $z$ puzzles, it calls $\mathtt{TLP.SolvPuz}(.)$ $z$ times, which leads to $O(Tz)$ computation complexity. Also, the verification cost  only involves $z$ invocations of the commitment scheme; if the hash-based commitment is used, then it would involve $z$ invocations of a hash function,  if Pedersen commitment is utilised, then it would involve $2 z$ exponentiations and $z$ multiplications. Thus, the  verification's complexity is $O(z)$.
%
%
%
%
% 
% 
% \noindent\textbf{\textit{Communication Complexity}}.  In step \ref{Generate-Puzzle}, the client publishes two vectors: $\vv{\bm{o}}$ and $\vv{\bm{h}}$, with  $2 z$ and $z$ elements respectively.  Each element of $\vv{\bm{o}}$ is a pair $(o_{\scriptscriptstyle j,1},o_{\scriptscriptstyle j,2})$, where $o_{\scriptscriptstyle j,1}$ is an output of symmetric key encryption, e.g.   $|o_{\scriptscriptstyle j,1}|=128$-bit, and $o_{\scriptscriptstyle j,2}$ is an element of $\mathbb{Z}_{\scriptscriptstyle N}$, e.g.  $|o_{\scriptscriptstyle j,2}|=2048$-bit. Also, each element $h_{\scriptscriptstyle j}$ of $\vv{\bm{h}}$ is either an output of a hash function, when a hash-based commitment is used, e.g.  $|h_{\scriptscriptstyle j}|=256$-bit, or an element of $\mathbb{F}_{\scriptscriptstyle q}$ when Pedersen commitment is used, e.g.  $|h_{\scriptscriptstyle j}|=1024$-bit. Thus, its  bandwidth is about $2432 z$ bits when the former,  or $3200 z$ bits when the latter commitment scheme is utilised. Also, its   complexity is $O(z)$. For the server to prove, in step \ref{prove-},   it sends $z$ pairs $(m_{\scriptscriptstyle j},d_{\scriptscriptstyle j})$ to the verifier, where $m_{\scriptscriptstyle j}$  is an arbitrary message, e.g.  $|m_{\scriptscriptstyle j}|=500$-bit, and  $d_{\scriptscriptstyle j}$ is either a long enough random value, e.g. $|d_{\scriptscriptstyle j}|=128$-bit, when the hash-based commitment is used, or an element of $\mathbb{F}_{\scriptscriptstyle q}$ when Pedersen scheme is used, e.g. $|d_{\scriptscriptstyle j}|=1024$-bit. So, its bandwidth is about either $628 z$ or $1524 z$ bits when the former or latter commitment scheme is used respectively. The solver's  communication complexity is $O(z)$. 
 

 

 % % !TEX root =main.tex

\section{C-TLP Security Proof}\label{CR-TLP-Proof}
 
In this section, we present   the security proof of C-TLP scheme. We first prove that without solving $j\text{\small{-th}}$ puzzle, a solver cannot find the parameters  needed to solve the next puzzle, i.e. $(j+1)\text{\small{-th}}$   one. 

 
  \begin{lemma}[Next Group Generator Privacy]\label{lemma::Next-Generator-Privacy}  Let $k$ be a random key for a symmetric key encryption, and  $N$ be a  sufficiently large RSA modulus. Let  the security parameter be $\lambda=|N|=|k|$.  In C-TLP, given puzzle vector: $\vv{\bm{o}}$ and public key: $pk$, an adversary $\mathcal{A}=(\mathcal{A}_{\scriptscriptstyle 1},\mathcal{A}_{\scriptscriptstyle 2})$, defined in Section \ref{Section::Multi-instance-Time-lock Puzzle-Definition},  cannot find the next group generator: 
$r_{\scriptscriptstyle j+1}$, where $r_{\scriptscriptstyle j+1} \stackrel{\scriptscriptstyle\$}\leftarrow \mathbb{Z}^{\scriptscriptstyle *}_{\scriptscriptstyle N}$and $j\geq1$, significantly smaller than   $T_{\scriptscriptstyle j}=\delta(j\Delta)$, except with a negligible probability in the security parameter, $\mu(\lambda)$ 
  \end{lemma}
 \begin{proof}
Since the next generator: $r_{\scriptscriptstyle j+1}$, is: (a) encrypted along with the $j\text{\small{-th}}$ puzzle solution: $s_{\scriptscriptstyle j}$, and (b)  picked uniformly at random from $\mathbb{Z}^{\scriptscriptstyle *}_{\scriptscriptstyle N}$, for the adversary to find $r_{\scriptscriptstyle j+1}$ without performing enough squaring, i.e. $T_{\scriptscriptstyle j}$, it has to either (a) break the symmetric key scheme, decrypt the related ciphertext: $s_{\scriptscriptstyle i}$ and extract the random value from it, or (b) correctly guess $r_{\scriptscriptstyle j+1}$. In both cases, the probability of success is negligible in secure parameter $\mu(\lambda)$,  i.e. $2^{\scriptscriptstyle -|k|}$ in the former case and $2^{\scriptscriptstyle -|N|}$ in the latter one.  
 \hfill\(\Box\)
  \end{proof} 
%. But its  probability of success is $2^{\scriptscriptstyle -|k|}$ that is negligible in the security parameter,  or  guess $r_{\scriptscriptstyle j+1}$, that has the probability of success  at most $2^{\scriptscriptstyle -|N|}$ which is negligible in $\lambda$ as well.    %\hfill\(\Box\)
 %t \end{proof} 
  
 In the following, we prove that the privacy of a solution in C-TLP scheme is preserved according to Definition \ref{Def::Solution-Privacy}. 
 
 
 \begin{theorem} [C-TLP Solution Privacy]\label{Solution-Privacy} Let $N$ be a  strong RSA modulus and $\Delta$ be a time parameter. If the sequential squaring assumption holds,  factoring $N$ is a hard problem, $\mathtt{H}(.)$ is a random oracle and the symmetric key encryption is  semantically secure, then  C-TLP encoding $z$ solutions is a privacy-preserving multi-instance time-lock puzzle w.r.t. Definition \ref{Def::Solution-Privacy}.
 \end{theorem}
  \begin{proof} In the following, we argue  for an adversary $\mathcal{A}=(\mathcal{A}_{\scriptscriptstyle 1},\mathcal{A}_{\scriptscriptstyle 2})$, where $\mathcal{A}_{\scriptscriptstyle 1}$ runs in total time $O(poly(j\Delta,\lambda))$,  $\mathcal{A}_{\scriptscriptstyle 2}$ runs in  time $\delta(j\Delta)<j\Delta$ using at most $\pi(\Delta)$ parallel processors, and  $j\in [1,z]$,  (a) when $z=1$: to find $s_{\scriptscriptstyle 1}$ earlier than $\delta(\Delta)$,  it has to  break the TLP scheme, and (b) when $z>1$: to find $s_{\scriptscriptstyle j}$ earlier than $T_{\scriptscriptstyle j}=\delta(j\Delta)$, it has to either find   at least one of the previous solutions earlier than it is supposed to (that ultimately requires breaking TLP scheme again), or find $j\text{\small{-th}}$ generator: $r_{\scriptscriptstyle j}$, earlier. Also, we argue that the commitments: $h_{\scriptscriptstyle j}$, are computationally hiding.   Specifically, when $z=1$, the security of C-TLP is reduced to the security of  the TLP and the scheme is secure as long as TLP is, as the two schemes would be identical. On the other hand, when $z>1$, the adversary has to either find $s_{\scriptscriptstyle j}$ earlier than $T_{\scriptscriptstyle j}$ as soon as the previous solution: $s_{\scriptscriptstyle j-1}$ is found that requires either breaking the TLP scheme, or finding any generator $r_{\scriptscriptstyle j}$  before $s_{\scriptscriptstyle j-1}$ is extracted, when $j\in [2,z]$. Nevertheless, the TLP scheme is secure (under RSA,  sequential squaring, and security of symmetric key encryption assumptions) according to  Theorem \ref{theorem::R-LTP-Sec}, and also the probability of finding the next generator: $r_{\scriptscriptstyle j}$ earlier than $T_{\scriptscriptstyle j-1}$ is negligible, according to Lemma \ref{lemma::Next-Generator-Privacy}. Moreover, for an adversary to find a solution earlier, it may also try to find a (partial information of) pre-image of the commitment: $h_{\scriptscriptstyle j}$ before fully (or without) solving the puzzle. But, this is infeasible for a PPT adversary, given  output of a random oracle: $\mathtt{H}(.)$. Thus, C-TLP is a privacy-preserving multi-instance time-lock puzzle scheme.  \hfill\(\Box\)
  \end{proof}
 
Next, we prove that the validity of a solution in C-TLP scheme is preserved according to Definition \ref{Def::Solution-Validity}. 
  \begin{theorem} [C-TLP Solution Validity]\label{Solution-Validity} Let $\mathtt{H}(.)$ be a hash function modeled as a random oracle. Then, C-TLP preserves a solution validity w.r.t. Definition \ref{Def::Solution-Validity}.  
\end{theorem}
\begin{proof}
 The proof  boils down to  the security (i.e. binding property) of the traditional hash-based commitment scheme. In particular, given an  opening pair, $\ddot{p}:(m_{\scriptscriptstyle j},d_{\scriptscriptstyle j})$ and the commitment $h_{\scriptscriptstyle j}=\mathtt{H}(m_{\scriptscriptstyle j},d_{\scriptscriptstyle j})$, for an adversary to break the solution validity, it has to come up $(m'_{\scriptscriptstyle j},d'_{\scriptscriptstyle j})$, such that $\mathtt{H}(m'_{\scriptscriptstyle j},d'_{\scriptscriptstyle j})=h_{\scriptscriptstyle j}$, where $m_{\scriptscriptstyle j}\neq m'_{\scriptscriptstyle j}$, i.e. finds a collision of $\mathtt{H}(.)$. However, this is infeasible for a PPT adversary, as $\mathtt{H}(.)$ is collision resistance, in the random oracle model. 
 \hfill\(\Box\)
\end{proof}
 
 In the following, we restate the main theorem presented in Section \ref{Section::C-TLP-protocol} and then prove it.  

\

\noindent\textbf{Theorem \ref{C-TLP-Sec} (C-TLP Security).} \textit{C-TLP  is a secure multi-instance time-lock puzzle. }

   
 \begin{proof} According to Theorems \ref{Solution-Privacy} and \ref{Solution-Validity}, the privacy and validity of a solution in C-TLP are preserved, respectively.   So, w.r.t. Definition \ref{def::C-TLP-security}, C-TLP is a secure multi-instance  time-lock puzzle.
  \hfill\(\Box\)
\end{proof}

