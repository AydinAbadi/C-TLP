%%
%% This is file `sample-sigconf.tex',
%% generated with the docstrip utility.
%%
%% The original source files were:
%%
%% samples.dtx  (with options: `sigconf')
%% 
%% IMPORTANT NOTICE:
%% 
%% For the copyright see the source file.
%% 
%% Any modified versions of this file must be renamed
%% with new filenames distinct from sample-sigconf.tex.
%% 
%% For distribution of the original source see the terms
%% for copying and modification in the file samples.dtx.
%% 
%% This generated file may be distributed as long as the
%% original source files, as listed above, are part of the
%% same distribution. (The sources need not necessarily be
%% in the same archive or directory.)
%%
%% The first command in your LaTeX source must be the \documentclass command.
%\documentclass[sigconf]{acmart}

%%
%% \BibTeX command to typeset BibTeX logo in the docs
%\AtBeginDocument{%
%  \providecommand\BibTeX{{%
%    \normalfont B\kern-0.5em{\scshape i\kern-0.25em b}\kern-0.8em\TeX}}}

%% Rights management information.  This information is sent to you
%% when you complete the rights form.  These commands have SAMPLE
%% values in them; it is your responsibility as an author to replace
%% the commands and values with those provided to you when you
%% complete the rights form.
%\setcopyright{.}
%\copyrightyear{}
%\acmYear{}
%\acmDOI{}

%% These commands are for a PROCEEDINGS abstract or paper.
%\acmConference[]{}{}{}
%\acmBooktitle{}
%\acmPrice{}
%\acmISBN{}


%%
%% Submission ID.
%% Use this when submitting an article to a sponsored event. You'll
%% receive a unique submission ID from the organizers
%% of the event, and this ID should be used as the parameter to this command.
%%\acmSubmissionID{123-A56-BU3}

%%
%% The majority of ACM publications use numbered citations and
%% references.  The command \citestyle{authoryear} switches to the
%% "author year" style.
%%
%% If you are preparing content for an event
%% sponsored by ACM SIGGRAPH, you must use the "author year" style of
%% citations and references.
%% Uncommenting
%% the next command will enable that style.
%%\citestyle{acmauthoryear}



%-------------------------------


\documentclass[orivec]{llncs}
%
\usepackage{color, colortbl}
\usepackage[inline]{enumitem}
\usepackage[table]{xcolor}
\usepackage{enumitem}
\usepackage{makeidx}  % allows for indexgeneration
\usepackage{amsfonts,amsmath,amssymb,graphicx,setspace,times,tipx}
\usepackage[ruled,linesnumbered]{algorithm2e}
%\usepackage{subfig}
\usepackage{framed}
\usepackage{esvect}
\usepackage{tikz}
\usepackage{latexsym}
\usepackage{multirow}

\usepackage{array}
\usepackage{subcaption}
\captionsetup{compatibility=false}

\newcommand{\resizeT}{\scalebox{1}}
\newcommand{\resizeS}{\scalebox{.56}}
\newcommand{\resizeSS}{\scalebox{.38}}

\usepackage{longtable}

\usepackage{tikz}
\usepackage{pgfplots}
\usepackage{pgfplots, pgfplotstable}

%\usepackage{kbordermatrix}% http://www.hss.caltech.edu/~kcb/TeX/kbordermatrix.sty
\usepackage{blkarray}% http://ctan.org/pkg/blkarra
\usepackage{mathtools}
\usepackage{amsmath}
\usepackage{bm}
\usepackage{adjustbox}
\usepackage{blindtext}
\usepackage{multicol}

\definecolor{bubblegum}{rgb}{0.99, 0.76, 0.8}




\usepackage{makecell, cellspace, caption}



\definecolor{Gray}{gray}{0.9}
%\usepackage[left=3cm,right=3cm,top=3cm,bottom=3cm]{geometry}
%

\usepackage[first=0,last=9]{lcg}
\newcommand{\ra}{\rand0.\arabic{rand}}

\usepackage[title]{appendix}

\usepackage{boxedminipage}

\usepackage{multirow}


%\usepackage{makeidx}  % allows for indexgeneration

%\usepackage[ruled,linesnumbered]{algorithm2e}
%\usepackage{enumitem}
%\usepackage[empty]{fullpage}% http://ctan.org/pkg/fullpage
%\usepackage{amsfonts,amsmath,amssymb,graphicx,setspace,times,tipx}
%\usepackage{kbordermatrix}% http://www.hss.caltech.edu/~kcb/TeX/kbordermatrix.sty
%\usepackage{blkarray}% http://ctan.org/pkg/blkarra


\usepackage{adjustbox}
\usepackage{blindtext}
\usepackage{multicol}




\usepackage{color, colortbl}
\definecolor{Gray}{gray}{0.9}
\usepackage{multirow}
\usepackage{esvect}
\usepackage{bm}
\usepackage{boxedminipage}
%\newtheorem{remark}{Remark}
\usepackage{subcaption}
 \usepackage{enumitem}
 %\newtheorem{assumption}{Assumption}



%\usepackage{lipsum}
%\usepackage{amsthm}
%\newtheorem*{assumption*}{\assumptionnumber}
\usepackage{tablefootnote}

\newtheorem{assumption}{Assumption}


\newcommand\NameEntry[1]{%
  \multirow{3}*{%
    \begin{varwidth}{10em}% --- or minipage, if you prefer a fixed width
    \flushright #1%
    \end{varwidth}}}




\newcommand{\specialcell}[2][c]{%
  \begin{tabular}[#1]{@{}c@{}}#2\end{tabular}}


















\begin{document}
  \setlength\abovedisplayskip{0pt}
  \setlength\belowdisplayskip{0pt}


\newenvironment{packed_item}{
\begin{itemize}
	\setlength{\topsep}{0pt}
	\setlength{\partopsep}{0pt}
  \setlength{\itemsep}{0pt}
  \setlength{\parskip}{0pt}
  \setlength{\parsep}{0pt}
}{\end{itemize}}

\newenvironment{packed_enum}{
\begin{enumerate}
	\setlength{\topsep}{0pt}
	\setlength{\partopsep}{0pt}
  \setlength{\itemsep}{0pt}
  \setlength{\parskip}{0pt}
  \setlength{\parsep}{0pt}
}{\end{enumerate}}




%----------------------------


\author{}
\institute{}
\maketitle  



%---------------------------My packages---------------------------------------

%\usepackage{amssymb,amsmath,amsthm}
%\newtheorem{theorem}{Theorem}
%%%%%%%%%%%%%%%%%%%%%%%%%%%%%%%%%%%%%%%%%%%%




%%
%% end of the preamble, start of the body of the document source.
\begin{document}

%%
%% The "title" command has an optional parameter,
%% allowing the author to define a "short title" to be used in page headers.
\title{Multi-instance Publicly Verifiable Time-lock Puzzle and  its Applications}

%%
%% The "author" command and its associated commands are used to define
%% the authors and their affiliations.
%% Of note is the shared affiliation of the first two authors, and the
%% "authornote" and "authornotemark" commands
%% used to denote shared contribution to the research.


%%%%%%%%%%%%%%%%%.   Authors. %%%%%%%%%%%%%%%%%%%%%%%

%\author{Aydin Abadi}
%%\authornote{Both authors contributed equally to this research.}
%\email{aydin.abadi@ed.ac.uk}
%%\orcid{1234-5678-9012}
%
%\affiliation{%
%  \institution{University of Edinburgh}
%}
%
%\author{Aggelos Kiayias}
%%\authornotemark[1]
%\email{akiayias@inf.ed.ac.uk}
%\affiliation{%
%  \institution{University of Edinburgh and IOHK}
%}


%\author{Ben Trovato}
%\authornote{Both authors contributed equally to this research.}
%\email{trovato@corporation.com}
%\orcid{1234-5678-9012}
%\author{G.K.M. Tobin}
%\authornotemark[1]
%\email{webmaster@marysville-ohio.com}
%\affiliation{%
%  \institution{Institute for Clarity in Documentation}
%  \streetaddress{P.O. Box 1212}
%  \city{Dublin}
%  \state{Ohio}
%  \postcode{43017-6221}
%}

%\author{Lars Th{\o}rv{\"a}ld}
%\affiliation{%
%  \institution{The Th{\o}rv{\"a}ld Group}
%  \streetaddress{1 Th{\o}rv{\"a}ld Circle}
%  \city{Hekla}
%  \country{Iceland}}
%\email{larst@affiliation.org}
%
%\author{Valerie B\'eranger}
%\affiliation{%
%  \institution{Inria Paris-Rocquencourt}
%  \city{Rocquencourt}
%  \country{France}
%}
%
%%%%%%%%%%%%%%%%%%%%%%%%%%%%%%%%%%%%%%%%


%%
%% By default, the full list of authors will be used in the page
%% headers. Often, this list is too long, and will overlap
%% other information printed in the page headers. This command allows
%% the author to define a more concise list
%% of authors' names for this purpose.
%\renewcommand{\shortauthors}{Trovato and Tobin, et al.}

%%
%% The abstract is a short summary of the work to be presented in the
%% article.
\begin{abstract}
Time-lock puzzles are vital cryptographic protocols with numerous  practical  applications. They  enable a party  to lock a message such that  no one else  can unlock it until a certain time  elapses. Nevertheless, existing puzzle schemes are not suitable in the multi-instance setting, where  a server is given $z$ instances of a puzzle at once and  it  must unlock each of them at different points time. In  this circumstance, the server has to start solving all puzzles as soon as it receives them, that  ultimately yields significant computation overhead and demands a high level of parallelisation. In this work, we put forth the concept of composing  instances of a puzzle scheme, where given instances' composition  at once, a server can unlock one puzzle after another, on time without the need to deal with all of them simultaneously.  We propose a candidate construction: ``\emph{chained  time-lock puzzle}'' (C-TLP). It allows a client to create and send to a server $z$ puzzles at once, while the server can solve every puzzle sequentially, without having to run parallel computations on them.  C-TLP makes black-box use of a standard time-lock puzzle scheme, it is efficient and its overall computation complexity of solving $z$ puzzles is  equivalent to that of solving only the last puzzle. It is also accompanied by a lightweight verification algorithm, publicly verifiable. It is the first time-lock puzzle scheme that offers a combination of   the above features.  We utilise C-TLP, as a key-escrow, to build a ``\emph{smarter outsourced  proofs of retrievability}'' which is the first  outsourced PoR  that supports \emph{real-time detection} and \emph{fair payment}, while having lower overheads than the state of the art. As another application of C-TLP,  we show in certain settings, one can substitute a verifiable delay function (VDF)  with C-TLP, to gain better efficiency. 
\end{abstract}

%%
%% The code below is generated by the tool at http://dl.acm.org/ccs.cfm.
%% Please copy and paste the code instead of the example below.
%%
%\begin{CCSXML}
%<ccs2012>
% <concept>
%  <concept_id>10010520.10010553.10010562</concept_id>
%  <concept_desc>Computer systems organization~Embedded systems</concept_desc>
%  <concept_significance>500</concept_significance>
% </concept>
% <concept>
%  <concept_id>10010520.10010575.10010755</concept_id>
%  <concept_desc>Computer systems organization~Redundancy</concept_desc>
%  <concept_significance>300</concept_significance>
% </concept>
% <concept>
%  <concept_id>10010520.10010553.10010554</concept_id>
%  <concept_desc>Computer systems organization~Robotics</concept_desc>
%  <concept_significance>100</concept_significance>
% </concept>
% <concept>
%  <concept_id>10003033.10003083.10003095</concept_id>
%  <concept_desc>Networks~Network reliability</concept_desc>
%  <concept_significance>100</concept_significance>
% </concept>
%</ccs2012>
%\end{CCSXML}

%\ccsdesc[500]{Computer systems organization~Embedded systems}
%\ccsdesc[300]{Computer systems organization~Redundancy}
%\ccsdesc{Computer systems organization~Robotics}
%\ccsdesc[100]{Networks~Network reliability}

%%
%% Keywords. The author(s) should pick words that accurately describe
%% the work being presented. Separate the keywords with commas.
%\keywords{Time-lock puzzle, Proofs of Retrievability, Blockchain, Smart Contract, Cloud computing.}

%% A "teaser" image appears between the author and affiliation
%% information and the body of the document, and typically spans the
%% page.
%\begin{teaserfigure}
%  \includegraphics[width=\textwidth]{pic/sampleteaser}
%  \caption{Seattle Mariners at Spring Training, 2010.}
%  \Description{Enjoying the baseball game from the third-base
%  seats. Ichiro Suzuki preparing to bat.}
%  \label{fig:teaser}
%\end{teaserfigure}

%%
%% This command processes the author and affiliation and title
%% information and builds the first part of the formatted document.
%\maketitle


% !TEX root =main.tex

\section{Introduction}


Time-lock puzzles are  interesting cryptographic primitives  that allow sending  information to the future. They enable  a party to lock a message such that,  no one else can unlock it until  a certain time has passed\footnote{ There exist protocols that use an assistance of a third party to support time-release of a secret.  This protocols' category is not our focus in this paper.}. They have a wide range of   applications, such as  e-voting \cite{ChenD12}, fair contract signing \cite{BonehN00},    and sealed-bid auctions \cite{Rivest:1996:TPT:888615}. Over the last two decades, a variety of time-lock puzzles have been proposed.  Nevertheless,  existing puzzle schemes  do not offer  any efficient remedy  for the multi-instance setting, where  a server is given multiple instances of a puzzle at once and it  should find one puzzle's solution after another.  It is a natural generalisation of the single puzzle setting. Application areas include, but not limited to:  (a) mass release of confidential documents  over time, (b) gradually revealing multiple secret keys, (c)    verifying continuous availability of cloud's services, e.g. data storage or secure hardware, or (d)  scheduled private payments, where  not only is every payment  made  after a certain  period, but also the payment details   remain confidential during the period. If  existing puzzle schemes are used directly in  the  multi-instance setting, then the server has to  deal with all puzzle instances, right after it receives them. This  results in a significant computation overhead and  requires a high level of parallelisation. %To date, there is no solution that mitigates  the aforementioned problem.
%It is a natural generalisation of the single puzzle setting.

In this paper, we propose  ``\emph{multi-instance time-luck puzzle}'', a primitive that allows composing a puzzle's instances, where given the composition, a server can deal with each instance sequentially. We formally define the primitive and  present an instantiation of it,  ``\emph{chained  time-lock puzzle}'' (C-TLP).  It makes black-box use of a standard time-lock puzzle scheme and is equipped  with a  \emph{lightweight} verification algorithm that allows anyone to check the correctness of a solution found by the server.    Its overall computation complexity of solving $z$ puzzles is  equivalent to that of solving only the last puzzle. The same procedure also imposes a  communication overhead linear with $z$, i.e. $O(z)$. C-TLP is the first time-lock puzzle scheme that offers all  the above features. Furthermore, we present  concrete applications of the primitive and demonstrate its use case in a blockchain-based solution. Specifically, we combine  the primitive's instantiation with a smart contract and apply the combination  to  ``outsourced proofs of  retrievability'' research line, and   propose ``\emph{smarter  outsourced proofs of retrievability}'' (SO-PoR) scheme which offers  a  combination of real-time detection (i.e. a data owner client is notified in almost real-time when a PoR proof is rejected) and fair payment (i.e. in every verification, the storage server is paid only if a PoR proof  is accepted) while  imposing very low overhead, that makes it particularly suitable for mission-critical data. SO-PoR verification and store phases impose $\frac{1}{4.5}$  and $\frac{1}{46\times 10^{\scriptscriptstyle 5}}$ of computation costs  imposed by the same phases in the fastest outsourced PoR. A server-side bandwidth of SO-PoR is much lower too;  for instance, for a $1$-GB file and $100$ verifications, a server in SO-PoR requires $9\times 10^{\scriptscriptstyle4}$ times fewer bits  than those required in the state of the art protocol.  Also, we show under certain circumstances  C-TLP can play the role of a ``verifiable delay function'' (VDF) but with much lower overhead, i.e. a prover's computation and communication costs will be reduced by factors of  $3$ and $6.5$ respectively.

% \emph{real-time detection}: the client is notified in almost real-time when a  proof is rejected, and (c) \emph{fair payment}: in every verification, the server is paid only if a  proof  is accepted

%The key observation that led us to the design of C-TLP is that if existing time-lock puzzles are utilised naively in the multi-puzzle setting,  the process of solving  puzzles (in parallel) has many overlaps that yield a high computation overhead. By eliminating the overlaps, we can considerably lower the overall cost. To attain that goal, in C-TLP, a client creates puzzles \emph{outside-in} while  they can be solved only \emph{inside-out}. In the sense that the client iteratively creates a puzzle for the solution  revealed after the rest, and integrates the information needed for solving  it into the puzzle for the solution revealed earlier. For the server to find the puzzles' solution, it  begins to find a solution revealed earlier than the rest. The found solution gives it enough information to find the next puzzle's solution,  after a certain time. This process proceeds until the last solution is discovered.  Also, we use the following novel idea to let C-TLP attain public verifiability. The client, when creating a puzzle  for a solution, commits to the solution. It combines the solution with the commitment opening and creates a puzzle on the combination.  Then, it attaches the commitment to the puzzle. Later on, when the server solves the puzzle (unlike  traditional commitment schemes in which the committer is  the prover), it plays the role of prover and opens the related commitment to the public who  can efficiently verify the solution's correctness. While chaining different puzzles seems a relatively obvious approach to tackle the aforementioned issues, design a secure protocol that also can make black-box use of a standard time-lock puzzle, supports public verifiability, and has low costs is not trivial. 


%
%\noindent\textbf{C-TLP and Outsourced Proofs of Retrievability.} Proofs of retrievability (PoR) schemes provide a strong guarantee to a client  that its data stored on a (potentially malicious) cloud  server  can be fully accessed  when required. Recently, researchers  developed \emph{outsourced} PoR schemes that  enable clients to  delegate the verification phase to a potentially malicious third-party auditor. Nevertheless, the existing outsourced PoRs  have a set of shortcomings, e.g. lack of real-time detection, lack of fair payment mechanism, or imposing  high costs.   In this paper, we present  \emph{smarter  outsourced proofs of retrievability} (SO-PoR) scheme,  the first efficient outsourced   PoR  that addresses the above issues. It allows a client to delegated PoR verification, meanwhile efficiently detecting the  server misbehavior in (almost) \emph{real-time} without the need to re-execute the verification itself. Supporting real-time detection makes SO-PoR  particularly suitable for mission-critical data.  It supports a \emph{fair payment}, meaning that the server gets paid only if it provides accepting proofs. SO-PoR outperforms the state of the art.  In particular, SO-PoR verification and store phases impose $\frac{1}{4.5}$  and $\frac{1}{46\times 10^{\scriptscriptstyle 5}}$ of computation costs  imposed by the same phases in the fastest outsourced PoR  (when the number of delegated verification is 100 and file size is $1$-GB). Also, SO-PoR has significantly lower communication cost than the fastest outsourced PoR.  The computation cost of verification and prove   in SO-PoR is identical to the costs in  the well-known \emph{privately} verifiable PoR scheme, while  both  have $O(1)$ proof size complexity.  In this scheme, C-TLP plays a focal role, and makes it  feasible  for SO-PoR to support efficient outsourced PoR verification and offer   the aforementioned features.  SO-PoR  leverages C-TLP, as an efficient key-escrow, to periodically disseminate PoR verification keys to a smart contract. The contract  performs the verification, on the client's behalf,  using lightweight message authenticated code (MAC). State it differently, in SO-PoR a smart contract performs PoR verifications using low-cost MAC; since MACs are only privately verifiable and smart contracts do not maintain a private state, SO-PoR also leverages C-TLP   to efficiently compensate for the smart contract's lack of privacy.
%
%
%
%\noindent\textbf{C-TLP and Verifiable Delay Function.} A verifiable delay function (VDF) allows a prover to provide publicly verifiable proofs proving it performed certain numbers of sequential
%computation. Recently, VDF's have drawn a lot of attention due to a wide range of applications they offer. A newly explored application of VDF's is in the settings where \emph{continuous availability of cloud's services} (e.g. data storage \cite{Storage-Time}) is desired. We show in such cases, VDF's can be replaced with C-TLP to gain better efficiency. As a result, the cloud's computation and communication overheads will be reduced by factors of  $3$ and $6.5$ respectively. 
%



\noindent\textbf{Summary of Our Contributions.} We (a) put forth the notion of multi-instance  time-lock puzzle,  formally define it, and identify its concrete applications,
(b) present a candidate construction, C-TLP,  the first multi-instance time-lock puzzle that is  built on a standard time-lock puzzle,  supports public verifiability, and  has low costs,   (c)  propose   the first  outsourced PoR that can offer  real-time detection and  fair payment while maintaining low costs, and (d) show in certain cases, a VDF can be replaced with C-TLP to gain better efficiency.



\vspace{-3mm}



% !TEX root =main.tex

\section{Related Work}\label{Related-Work}


In this section, we provide a summary of related work. For a comprehensive survey, we
refer readers to Appendix \ref{Survey-of-Related-Work}.
% !TEX root =main.tex

%\subsection{Time-lock Puzzles}
\noindent\textbf{Time-lock Puzzles}
The idea to send information into the \emph{future}, i.e.
time-lock puzzle/encryption, was first put forth by Timothy C. May. A time-lock puzzle allows a party to encrypt a message such that it cannot be decrypted  until a certain  time has passed. In general,  a  time-lock scheme should allow   generating (and verifying) a puzzle to take less time than solving it. The  scheme that May proposed relies on a trusted agent. Later, Rivest \textit{et al.} \cite{Rivest:1996:TPT:888615} propose an RSA-based puzzle scheme that does not require a trusted agent, and is secure against a receiver
who may have access to many  computation resources that run in parallel. The latter protocol has been the core of (almost) all later time-lock puzzle schemes that supports the encapsulation of an arbitrary message. Later, \cite{BonehN00,DBLP:conf/fc/GarayJ02} proposed a scheme which also let a puzzle generator  prove (in Zero-knowledge)  to a puzzle solver that the correct solution  will be recovered after a certain time.    Recently, \cite{MalavoltaT19,BrakerskiDGM19}  propose  homomorphic time-lock puzzles, where an arbitrary function can be run over puzzles before they are solved. In the protocols, all puzzles have an identical time parameter, and their solutions are supposed to be discovered at the same time. They  are based on the RSA-based puzzle  and   fully homomorphic encryption,  computationally expensive. Very recently, Chvojka \textit{et al.} in \cite{ChvojkaJSS20} propose  incremental time-release encryption which lets a server, given a set of encrypted messages, discover messages sequentially over time. It is the closest work to ours. Nevertheless, the scheme uses the RSA time-lock puzzle \cite{Rivest:1996:TPT:888615} in a \emph{non-black-box manner}, offers no (public) verification,  is based on  asymmetric key encryption instead of symmetric key encryption used in the majority of time lock-puzzle schemes, and uses a non-standard (asymmetric) encryption scheme. 




%%Two related but different notions are pricing puzzles and verifiable delay functions. 
%%
%%\noindent\textbf{\text{Pricing Puzzles.}} Also known as \emph{client puzzles}. It was first put forth by Dwork \textit{et al.} \cite{DworkN92} who defined it as a function that requires a certain amount of computation resources to solve a puzzle.  In general,  pricing puzzles are based on either hash inversion problems publicly verifiable, e.g.  \cite{DworkN92,groza2006chained}, or number-theoretic privately verifiable, e.g.  \cite{KuppusamyRSBN12,KarameC10}. The application area of such puzzles includes  defending against denial-of-service (DoS)  attacks and reaching a consensus in cryptocurrencies. However, unlike time-lock puzzles, pricing puzzles are not designed to encapsulate an arbitrary message. 
%%
%%
%%
%%
%%
%%\noindent\textbf{\text{Verifiable Delay Function (VDF).}} Allows a prover to provide a publicly verifiable proof stating  it has performed  a pre-determined number of sequential computations. VDF was first formalised by Boneh \textit{et al} in \cite{BonehBBF18} that proposed several VDF constructions. Later on,  \cite{Wesolowski19} improved the previous VDF's  from different perspectives and proposed a scheme  based on RSA time-lock encryption, in the random oracle model. To date, this protocol is the most efficient VDF.  Most of VDF schemes are built  upon time-lock puzzles, however the converse is not necessarily the case, as VDF's are not designed to conceal an  arbitrary private message, and they take a public message as input while time-lock puzzles are designed to conceal a private input message. 


%BonehBBF18,Wesolowski19



%\vspace{-2mm}









% !TEX root =main.tex

%\subsection{Proofs of Retrievability}\label{related-work-PoR-short}
%\vspace{-1mm}

\noindent\textbf{Outsourced Proofs of Retrievability}
Proofs of retrievability (PoR) schemes, introduced in \cite{DBLP:conf/ccs/JuelsK07},  ensures  a client's data on a cloud server is fully accessible. Ever since  a variety of PoR's has been proposed.  Recently,  \cite{armknecht2014outsourced,xu2016lightweight} present  \emph{outsourced} PoR protocols that let  clients outsource the   verification to a potentially malicious third-party auditor. The scheme in \cite{armknecht2014outsourced}   has the fastest prove and verification algorithms. It uses message authentication code (MAC) based tags, zero-knowledge proofs and error-correcting codes. But, it  has  several shortcomings, i.e. it offers no real-time detection,  provides no efficient way for fair payments and  has high costs of setup and auditor onboarding.   Xu \textit{et al.} in \cite{xu2016lightweight} propose a publicly verifiable outsourced PoR to improve the previous scheme's setup cost. It uses   BLS signatures-like tags, polynomial arithmetic and error-correcting codes.  It assumes an auditor is fully trusted during each verification whose overhead is higher  than \cite{armknecht2014outsourced}. Recently in \cite{Storage-Time} two protocols are proposed, ``basic PoSt'' and ``compact PoSt''. They ensure that a client's data remains available on a   server for a   period, without the client's involvement  in that period.  The basic PoSt  uses a Merkle tree-based PoR and VDF. It has a high communication cost. Since it  requires a verifier to validate VDF's outputs, it imposes a significant computation cost, in practice. The compact PoSt has a lower communication cost than the basic one, as it let the server combine PoR proofs. It is mainly based on a trapdoor delay function (TDF). 

%Note,  all  outsourced PoR schemes \cite{armknecht2014outsourced,xu2016lightweight,Storage-Time}  assume the client behaves honestly towards the server. Otherwise, a malicious client can generate the tags in a way that  makes an honest server generate invalid proofs. 

%Although in \cite{armknecht2014outsourced} a client is considered to be malicious from an auditor point of view,  the protocol assumes the client behaves honestly towards the server. %Similarly, in this paper, we assume the client is honest. 

%In these protocols,  a client uploads its data  to a server once, then  the server generates proofs of storage (e.g. PoR) periodically, collects them and sends the collection after  time $T$ to a verifier  who can check the proofs. 

%\vspace{-1mm}

\noindent\textbf{Blockchain-based PoR.} There exist distributed PoR schemes that let a client  distribute its file among different (tailored) blockchain nodes,  e.g. Permacoin \cite{MillerPermacoin}, Filecoin  \cite{Filecoin}, and KopperCoin \cite{KoppBK16}. But,  they have either a large proof size (e.g. in \cite{MillerPermacoin,Filecoin})  logarithmic with the  file size when a  Merkle tree is used, or high concrete verification overhead (e.g. in \cite{KoppBK16}) due to the use of BLS signatures. There are protocols that use blockchain to verify the retrievability of off-chain data \cite{RennerMK18,HaoXWJW18,ZhangDLZ18,Audita18,blockchain-data-audit-18,sia14}. Nevertheless, they either impose a high communication/computation cost \cite{RennerMK18,HaoXWJW18,Audita18,blockchain-data-audit-18,sia14}, or  clients have to be online for each verification  \cite{ZhangDLZ18}.  Campanelli \textit{et al.}  \cite{CampanelliGGN17}  present a fair exchange mechanism over a blockchain that ensures the server gets paid if it provides an accepting PoR proof. But, this scheme assumes either the client can   perform the verification itself or a third-party, acting on the client's behalf,  carries out the verification honestly. 


%BonehBBF18,Wesolowski19




%\vspace{-2mm}






 
% !TEX root =main.tex



\section{Preliminaries}

\vspace{-2mm}

 In this section we provide the main primitives used in this work. We provide a notation table  in Appendix \ref{sec:notation-table}.

\vspace{-4mm}

\subsection{Smart Contract} 

\vspace{-2mm}

Cryptocurrencies, such as Bitcoin and Ethereum, in addition to offering a decentralised currency,  support  computations on  transactions. In this setting, often a certain computation logic is encoded in a computer program, called \emph{``smart contract''}. To date, Ethereum is the most predominant cryptocurrency framework that enables users to define arbitrary smart contracts. In this framework,  contract code is stored on the blockchain and  executed by all parties (i.e. miners) maintaining the cryptocurrency,  when the program inputs are provided by transactions. The program execution's  correctness  is  guaranteed by the security of the underlying blockchain components. To prevent  a denial of service attack, the framework requires a transaction creator to pay a  fee, called \emph{``gas''}, depending on the complexity of the contract running on  it.  






%Nonetheless,  Ethereum smart contracts suffer from an important   issue; namely, the \emph{lack of privacy}, as it requires  every contract's data to be public, which is a major impediment  to  the broad adoption of  smart contracts when a certain level of privacy is desired. To address the issue, researchers/users may either (a)  utilise existing decentralised frameworks  which support privacy-preserving smart contracts, e.g. \cite{KosbaMSWP16}. But, due to the use of generic and computationally expensive cryptographic tools,  they impose a significant cost to their users. Or (b)  design  efficient tailored cryptographic protocols  that preserve (contracts) data privacy, even though non-private smart contracts are used. We take the latter approach in this work. 

\vspace{-4mm}

\subsection{Commitment Scheme} 
\vspace{-1.4mm}

A commitment scheme involves two parties:  \emph{sender} and  \emph{receiver}, and includes  two phases: \emph{commit} and  \emph{open}. In the commit phase, the sender  commits to a message: $m$ as $\mathtt{Com}(m,d)=h$, that involves a secret value: $d$. At the end of the commit phase,  the commitment: $h$ is sent to the receiver. In the open phase, the sender sends the opening: $\ddot{p}=(m,d)$ to the receiver who verifies its correctness: $\mathtt{Ver}(h,\ddot{p})\stackrel{\scriptscriptstyle ?}=1$ and accepts if the output is $1$.  A commitment scheme must satisfy two properties: (a) \textit{hiding}: infeasible for an adversary  to learn any information about the committed  value: $m$, until the commitment: $h$ is opened, and (b) \textit{binding}:   infeasible for an adversary (i.e. the sender) to open a commitment: $h$ to different values: $\ddot{p}'=(m',d')$ than that used in the commit phase, i.e. infeasible to find  $\ddot{p}'$, \textit{s.t.} $\mathtt{Ver}(h,\ddot{p})=\mathtt{Ver}(h,\ddot{p}')=1$, where $\ddot{p}\neq \ddot{p}'$.  There exist efficient non-interactive  commitment schemes both in (a) the random oracle model using the well-known hash-based scheme  that $\mathtt{Com}(m,d)$ involves computing: $\mathtt{H}(m||d)=h$ and $\mathtt{Ver}(h,\ddot{p})$ requires checking: $\mathtt{H}(m||d)\stackrel{\scriptscriptstyle ?}=h$, where $\mathtt{H}(.)$ is a collision resistance hash function, and (b)  the standard model, e.g. Pedersen scheme \cite{Pedersen91}. 


\vspace{-5mm}


\subsection{Pseudorandom Function}

\vspace{-2mm}

Informally, a pseudorandom function ($\mathtt{PRF}$) is a deterministic function that takes a key and an input; and outputs a value  indistinguishable from that of  a truly random function with the same input.   A $\mathtt{PRF}$ is formally defined as follows \cite{DBLP:books/crc/KatzLindell2007}. 
\begin{definition} Let $W:\{0,1\}^{\scriptscriptstyle\psi}\times \{0,1\}^{\scriptscriptstyle \eta}\rightarrow \{0,1\}^{\scriptscriptstyle  \iota}$ be an efficient  keyed function. It is said $W$ is a pseudorandom function if for all probabilistic polynomial-time distinguishers $B$, there is a negligible function, $\mu(.)$, such that:
\begin{equation*}
\small{
\bigg | Pr[B^{\scriptscriptstyle W_{k}(.)}(1^{\scriptscriptstyle \psi})=1]- Pr[B^{\scriptscriptstyle \omega(.)}(1^{\scriptscriptstyle \psi})=1] \bigg |\leq \mu(\psi)
}
\end{equation*}
where  the key, $k\stackrel{\scriptscriptstyle\$}\leftarrow\{0,1\}^{\scriptscriptstyle\psi}$, is chosen uniformly at random and $\omega$ is chosen uniformly at random from the set of functions mapping $\eta$-bit strings to $\iota$-bit strings. 
\end{definition}


%In practice, we are interested in  pseudorandom functions that can be efficiently built on smart contracts given the tools  that a smart contract framework (e.g. Ethereum) offers. HMAC \cite{DBLP:conf/crypto/BellareCK96} satisfies the requirements above.

\vspace{-3mm}
 
\subsection{ Time-lock  Puzzle}\label{Time-lock-Encryption} 

\vspace{-2mm}

In this section, we restate the formal definition of a time-lock puzzle as well as RSA-based  time-lock puzzle protocol \cite{Rivest:1996:TPT:888615}. We consider the RSA-based puzzle because of its simplicity and  being the core of (almost) all later time-lock puzzle schemes.

\vspace{-1mm}

\begin{definition}[Time-lock Puzzle]\label{Def::Time-lock-Puzzle} A time-lock puzzle comprises the following efficient three algorithms, such that the puzzle satisfies completeness and efficiency properties. 
\begin{itemize}[leftmargin=.37cm]
\item \textbf{Algorithms}:
\begin{itemize}
\item[$\bullet$]$\mathtt{Setup}(1^{\scriptscriptstyle\lambda},\Delta)\rightarrow (pk,sk)$: a probabilistic algorithm that takes as input  security: $1^{\scriptscriptstyle\lambda}$ and time:  $\Delta$ parameters. It outputs a public-private key pairs: $(pk,sk)$

\item[$\bullet$]$\mathtt{GenPuz}(s, pk, sk)\rightarrow \ddot{o}$: a probabilistic algorithm that takes as input a solution: $s$ and the public-private key pairs: $(pk,sk)$. It  outputs a puzzle: $\ddot{o}$

\item[$\bullet$]$\mathtt{SolvPuz}(pk,\ddot{o})\rightarrow s$:  a deterministic algorithm that takes as input  the public key: $pk$ and  puzzle: $\ddot{o}$. It outputs a solution: $s$
\end{itemize}
\item \textbf{Completeness}: always  $\mathtt{SolvPuz}(pk,\mathtt{GenPuz}(s,pk,sk))=s$


\item \textbf{Efficiency}: the run-time of algorithm $\mathtt{SolvPuz}(pk,\ddot{o})$ is bounded by  $poly(\Delta,\lambda)$, where $poly(.)$ is a  polynomial.
\end{itemize}
\end{definition}


Informally, a time-lock puzzle's security requires that the puzzle solution  remain hidden from all adversaries running in parallel within the time period, $\Delta$.   It is essential that no adversary can find a solution   in  time $\delta(\Delta)<\Delta$, using  $\pi(\Delta)$  processors running in parallel and after a potentially large amount of pre-computation. So, such factors are explicitly incorporated  into the puzzle's definitions \cite{BonehBBF18,MalavoltaT19,garay2019}. 
\begin{definition}[Time-lock Puzzle Security] A time-lock puzzle is secure if for all $\lambda$ and $\Delta$, all probabilistic polynomial time adversaries $\mathcal{A}=(\mathcal{A}_{\scriptscriptstyle 1},\mathcal{A}_{\scriptscriptstyle 2})$ where $\mathcal{A}_{\scriptscriptstyle 1}$ runs in total time $O(poly(\Delta,\lambda))$ and $\mathcal{A}_{\scriptscriptstyle 2}$ runs in  time $\delta(\Delta)<\Delta$ using at most $\pi(\Delta)$ parallel processors, there exists a negligible function $\mu(.)$, such that: 
%\footnotesize{
\small{
$$ Pr\left[
  \begin{array}{l}
\mathcal{A}_{\scriptscriptstyle 2}(pk, \ddot{o},\text{state})  \rightarrow b
\end{array} \middle |
    \begin{array}{l}
\mathtt{Setup}(1^{\scriptscriptstyle\lambda},\Delta)\rightarrow (pk,sk)\\
\mathcal{A}_{\scriptscriptstyle 1}(1^{\scriptscriptstyle\lambda},pk, \Delta)\rightarrow (s_{\scriptscriptstyle 0},s_{\scriptscriptstyle 1},\text{state})\\
b\stackrel{\scriptscriptstyle\$}\leftarrow \{0,1\}\\
\mathtt {GenPuz}(s_{\scriptscriptstyle b}, pk, sk)\rightarrow \ddot{o}\\
\end{array}    \right]\leq \frac{1}{2}+\mu(\lambda)$$
}
\end{definition}






An  RSA-based time-lock puzzle construction that realises the above definitions was proposed in  \cite{Rivest:1996:TPT:888615}. The construction is as follows. 




\begin{enumerate}[leftmargin=.43cm]
\item \textbf{Setup}: $\mathtt{TLP.Setup}(1^{\scriptscriptstyle\lambda}, \Delta)$
\begin{enumerate}


\item Compute $N=q_{\scriptscriptstyle 1}q_{\scriptscriptstyle 2}$, where $q_{\scriptscriptstyle i}$  is a large randomly chosen prime number. Then   compute Euler's totient function of $N$, as: $\phi(N)=(q_{\scriptscriptstyle 1}-1)(q_{\scriptscriptstyle 2}-1)$ 
\item  Set  $T=S\Delta$ as the total number of squaring needed to decrypt an encrypted message $m$, where $\Delta$ is the period (in seconds) within  which the message should remain private and $S$ is the maximum  number of squaring modulo $N$ per second that can be performed by a solver. 
\item\label{TLP::pick-k}  Choose a random key: $k$ for a semantically secure symmetric key encryption   that has  three algorithms: $(\mathtt{GenKey}, \mathtt{Enc},\mathtt{Dec}$) 
\item Pick a uniformly random value $r$ from  $\mathbb{Z}^{\scriptscriptstyle *}_{\scriptscriptstyle N} $
\item Compute $a=2^{\scriptscriptstyle T}\bmod \phi(N)$
\item Set $pk=(N,T,r)$ as  public key and set $sk=(q_{\scriptscriptstyle 1},q_{\scriptscriptstyle 2},a,k)$ as secret key.
\end{enumerate}

\item\label{Generate-Puzzle-} \textbf{Generate Puzzle}: $\mathtt{TLP.GenPuz}(m,pk,sk)$ %$\mathcal{ENC}^{\scriptscriptstyle pk}_{\scriptscriptstyle sk,T}()$ 

\begin{enumerate}
\item\label{R-TLP::enc-message}  Encrypt the message using the symmetric key encryption: $o_{\scriptscriptstyle 1}= \mathtt{Enc}(k,m)$
\item\label{TLP::mask-k} Encrypt the key: $k$, as: $o_{\scriptscriptstyle 2}= k+r^{\scriptscriptstyle a}\bmod N$
%\end{enumerate*}
\item Sets: $\ddot{o}=(o_{\scriptscriptstyle 1},o_{\scriptscriptstyle 2})$ as ciphertext or puzzle. Next, output $\ddot{o}$
\end{enumerate}




\item\textbf{Solve Puzzle}: $\mathtt{TLP.SolvPuz}(pk,\ddot{o})$ 

\begin{enumerate}
\item\label{R-TLP::find-b} Find $b$, where $b=r^{\scriptscriptstyle 2^{\scriptscriptstyle T}}\bmod N$, by using $T$ number of squaring $r$ modulo $N$
\item\label{R-TLP::dec-key} Decrypt the key's ciphertext: $k=o_{\scriptscriptstyle 2}-b\bmod N$
\item\label{R-TLP::dec-message} Decrypt the message's ciphertext: $m=\mathtt{Dec}(k,o_{\scriptscriptstyle 1})$.  Output $m$
\end{enumerate}
\end{enumerate}

 \vspace{-1mm}
 
Informally, the  time-lock puzzle's security relies on the hardness of factoring problem,     the security of the symmetric key encryption, and sequential squaring assumption. We refer readers to Appendix \ref{R-TLP-proof} for more discussion on the construction and its security.

\vspace{-3mm}

%% !TEX root =main.tex



%\section{Notations}\label{sec:notation-table}
%
%We summarise our notation in Table \ref{table:notation-table}.
%
%\vspace{-4mm}
%\begin{table}[!htbp]
%%\begin{footnotesize}
%\small{
%\begin{center}
%%\footnotesize{
%
%%\scalebox{.98}{
%%\begin{minipage}{.9\linewidth}
%\caption{ \small Notation Table.}\label{table:notation-table} 
%\renewcommand{\arraystretch}{.7}
%%\resizebox{\columnwidth}{!}{
%
%% 1st table
%\begin{tabular}{|c|c|c|c|c|c|c|c|c|c|c|c|c|c|} 
%
%\hline 
%
%\cellcolor{gray!30}\scriptsize \textbf{Setting} &\cellcolor{gray!30} \scriptsize \textbf{Symbol}&\cellcolor{gray!30} \scriptsize \textbf{Description}  \\
%    \hline
%    
%     \hline
%
%%Generic
%\multirow{9}{*}{\rotatebox[origin=c]{90}{\scriptsize \textbf{Generic}}}
%
% &\cellcolor{white!20}\scriptsize$z$&\cellcolor{white!20}\scriptsize \text{Number of puzzles or delegated verifications}\\   
% 
%&\cellcolor{gray!20}\scriptsize$h$, $h_{\scriptscriptstyle j}$ &\cellcolor{gray!20}\scriptsize Hash values  \\  
%
%&\cellcolor{white!20}\scriptsize$d$, $d_{\scriptscriptstyle j}$ &\cellcolor{white!20}\scriptsize Randomness  of commitment\\ 
%
%&\cellcolor{gray!20}\scriptsize$n$ &\cellcolor{gray!20}\scriptsize Number of file blocks\\ 
%&\cellcolor{white!20}\scriptsize$m$, $m_{\scriptscriptstyle j}$ &\cellcolor{white!20}\scriptsize Plaintext messages\\   
%  &\cellcolor{gray!20}\scriptsize$\ddot{o}$ &\cellcolor{gray!20}\scriptsize Pair representation\\
%&\cellcolor{white!20}\scriptsize$\vv{\bm{o}}$ &\cellcolor{white!20}\scriptsize Vector representation\\                  
%&\cellcolor{gray!20}\scriptsize$\ddot{p}:(m_{\scriptscriptstyle j},d_{\scriptscriptstyle j})$ &\cellcolor{gray!20}\scriptsize Commitment opening\\ 
%                      
%&\cellcolor{white!20}\scriptsize$\mathtt{H}$ &\cellcolor{white!20}\scriptsize Hash function\\
%
%                      
% \hline
% 
%  \hline
%  
%  
%    %CR-TLP
%\multirow{11}{*}{\rotatebox[origin=c]{90}{\scriptsize  \textbf{C-TLP}}}
%
%&\cellcolor{gray!20}\scriptsize$\lambda$&\cellcolor{gray!20}\cellcolor{gray!20}\scriptsize TLP security parameter\\
%
%&\cellcolor{white!20}\scriptsize$\Delta$&\cellcolor{white!20}\scriptsize Time win. message remains hidden\\ 
%       
%&\cellcolor{gray!20}\scriptsize $S$&\cellcolor{gray!20}\scriptsize Max. squaring  done per sec.    \\ 
%    
%&\cellcolor{white!20}\scriptsize $T$&\cellcolor{white!20}\scriptsize $T=S\Delta$    \\
%
%&\cellcolor{gray!20}\scriptsize $s_{\scriptscriptstyle j}$&\cellcolor{gray!20}\scriptsize $j\text{-th}$ solution\\
%
%&\cellcolor{white!20}\scriptsize $\ddot{o}_{\scriptscriptstyle j}$&\cellcolor{white!20}\scriptsize $j\text{-th}$ puzzle, $\ddot{o}_{\scriptscriptstyle j}:(o_{\scriptscriptstyle j,1},o_{\scriptscriptstyle j,2})$\\
%
%&\cellcolor{gray!20}\scriptsize $f_{\scriptscriptstyle j}$&\cellcolor{gray!20}\scriptsize Time  when  $j{\text{-th}}$ solution is found\\
%
%&\cellcolor{white!20}\scriptsize $k, k_{\scriptscriptstyle j}$&\cellcolor{white!20}\scriptsize Sym. key encryption keys\\
%
%&\cellcolor{gray!20}\scriptsize $pk, sk$&\cellcolor{gray!20}\scriptsize Public and secret keys\\
%
%&\cellcolor{white!20}\scriptsize $q_{\scriptscriptstyle 1}, q_{\scriptscriptstyle 2}$&\cellcolor{white!20}\scriptsize Large prime numbers\\
%
%&\cellcolor{gray!20}\scriptsize $N$&\cellcolor{gray!20}\scriptsize RSA modulus, $N=q_{\scriptscriptstyle 1} q_{\scriptscriptstyle 2}$\\
%
%\hline 
%
%
%  \hline
%  
%  
%%SO-PoR 
%  \multirow{20}{*}{\rotatebox[origin=c]{90}{\scriptsize \textbf{SO-PoR}}}
%&\cellcolor{gray!20}\scriptsize$\mathtt{PRF}$&\cellcolor{gray!20}\scriptsize Pseudorandom function\\  
%                   
%    &\cellcolor{white!20}\scriptsize$\hat{k},v_{\scriptscriptstyle j},l_{\scriptscriptstyle j}$&\cellcolor{white!20}\scriptsize $\mathtt{PRF}$'s keys\\ 
%&\cellcolor{gray!20}\scriptsize$\iota$&\cellcolor{gray!20}\scriptsize Security parameter, $\iota=128$-bit\\ 
%&\cellcolor{white!20}\scriptsize$p$&\cellcolor{white!20}\scriptsize Large prime number, $|p|=\iota$\\ 
%
%&\cellcolor{gray!20}\scriptsize$w$&\cellcolor{gray!20}\scriptsize  Blockchain block index\\ 
%                    
%&\cellcolor{white!20}\scriptsize$g$&\cellcolor{white!20}\scriptsize Blockchain security parameter: chain quality    \\    
%      
%  &\cellcolor{gray!20}\scriptsize$\lambda'$&\cellcolor{gray!20}\scriptsize Blockchain generic security parameter\\  
%&\cellcolor{white!20}\scriptsize${\bm{F}}$&\cellcolor{white!20}\scriptsize Outsourced encoded file\\ 
%&\cellcolor{gray!20}\scriptsize$F_{\scriptscriptstyle j}$&\cellcolor{gray!20}\scriptsize A file block\\ 
%&\cellcolor{white!20}\scriptsize$|{\bm{F}}|$&\cellcolor{white!20}\scriptsize Number of file blocks, $|{\bm{F}}|=n$\\ 
% &\cellcolor{gray!20}\scriptsize$||{\bm{F}}||$&\cellcolor{gray!20}\scriptsize File bit-size\\     
%% \multirow{8}{*}{\rotatebox[origin=c]{90}{\scriptsize \textbf{SO-PoR}}}
%
%&\cellcolor{white!20}\scriptsize$\sigma_{\scriptscriptstyle i}$&\cellcolor{white!20}\scriptsize Permanent tag   \\  
% &\cellcolor{gray!20}\scriptsize$\sigma_{\scriptscriptstyle b,j}$&\cellcolor{gray!20}\scriptsize Disposable tag   \\    
% 
%    &\cellcolor{white!20}\scriptsize$\alpha, \alpha_{\scriptscriptstyle j}, r_{\scriptscriptstyle i},r_{\scriptscriptstyle b,j}$&\cellcolor{white!20}\scriptsize Pseudorandom values  \\ 
%                
%&\cellcolor{gray!20}\scriptsize$c$&\cellcolor{gray!20}\scriptsize Number of challenges \\ 
%        
%&\cellcolor{white!20}\scriptsize$\mathcal {B}_{\scriptscriptstyle j}$&\cellcolor{white!20}\scriptsize Blockchain's $j{\text{-th}}$ block\\ 
%
%&\cellcolor{gray!20}\scriptsize$(\mu_{\scriptscriptstyle j},\xi_{\scriptscriptstyle j})$&\cellcolor{gray!20}\scriptsize $j{\text{-th}}$ PoR proof\\ 
%&\cellcolor{white!20}\scriptsize$\Delta_{\scriptscriptstyle 1}$&\cellcolor{white!20}\scriptsize Time taken to generate  a PoR\\ 
%
%&\cellcolor{gray!20}\scriptsize$\Delta_{\scriptscriptstyle 2}$&\cellcolor{gray!20}\cellcolor{gray!20}\scriptsize Time taken a contract gets a message\\ 
%
%&\cellcolor{white!20}\scriptsize$e$&\cellcolor{white!20}\scriptsize Coins paid for an accepting PoR\\ 
%
%\hline
%
%
%        
%
%
%\end{tabular}
%
%\end{center}
%}
%%\end{footnotesize}
%\end{table}
%

%%%%%%%%%%%%%%%%%%%%%%%%%%%%%%%%%%%%%%%%%%


\section{Notations}\label{sec:notation-table}

We summarise our notations in Table \ref{table:notation-table}.
%\vspace{-5mm}

\begin{table*}[!htbp]
\begin{scriptsize}
\begin{center}
\footnotesize{

\caption{ \small Notation Table.}\label{commu-breakdown-party} 
\renewcommand{\arraystretch}{.84}
\scalebox{0.86}{
% 1st table
\begin{tabular}{|c|c|c|c|c|c|c|c|c|c|c|c|c|c|} 

\hline 

\cellcolor{gray!15}\scriptsize \textbf{Setting} &\cellcolor{gray!15} \scriptsize \textbf{Symbol}&\cellcolor{gray!15} \scriptsize \textbf{Description}  \\
    \hline
    
     \hline

%Generic
\multirow{9}{*}{\rotatebox[origin=c]{90}{\scriptsize \textbf{Generic}}}

 &\cellcolor{white!20}\scriptsize$z$&\cellcolor{white!20}\scriptsize \text{Number of puzzles or delegated verifications}\\   
 
&\cellcolor{gray!20}\scriptsize$h$, $h_{\scriptscriptstyle j}$ &\cellcolor{gray!20}\scriptsize Hash values  \\  

&\cellcolor{white!20}\scriptsize$d$, $d_{\scriptscriptstyle j}$ &\cellcolor{white!20}\scriptsize Randomness  of commitment\\ 

&\cellcolor{gray!20}\scriptsize$n$ &\cellcolor{gray!20}\scriptsize Number of file blocks\\ 
&\cellcolor{white!20}\scriptsize$m$, $m_{\scriptscriptstyle j}$ &\cellcolor{white!20}\scriptsize Plaintext messages\\   
  &\cellcolor{gray!20}\scriptsize$\ddot{o}$ &\cellcolor{gray!20}\scriptsize Pair representation\\
&\cellcolor{white!20}\scriptsize$\vv{\bm{o}}$ &\cellcolor{white!20}\scriptsize Vector representation\\                  
&\cellcolor{gray!20}\scriptsize$\ddot{p}:(m_{\scriptscriptstyle j},d_{\scriptscriptstyle j})$ &\cellcolor{gray!20}\scriptsize Commitment opening\\ 
                      
&\cellcolor{white!20}\scriptsize$\mathtt{H}$ &\cellcolor{white!20}\scriptsize Hash function\\

                      
 \hline
 
  \hline
  
  %CR-TLP
  %CR-TLP
\multirow{11}{*}{\rotatebox[origin=c]{90}{\scriptsize  \textbf{C-TLP}}}

&\cellcolor{gray!20}\scriptsize$\lambda$&\cellcolor{gray!20}\cellcolor{gray!20}\scriptsize TLP security parameter\\

&\cellcolor{white!20}\scriptsize$\Delta$&\cellcolor{white!20}\scriptsize Time win. message remains hidden\\ 
       
&\cellcolor{gray!20}\scriptsize $S$&\cellcolor{gray!20}\scriptsize Max. squaring  done per sec.    \\ 
    
&\cellcolor{white!20}\scriptsize $T$&\cellcolor{white!20}\scriptsize $T=S\Delta$    \\

&\cellcolor{gray!20}\scriptsize $s_{\scriptscriptstyle j}$&\cellcolor{gray!20}\scriptsize $j\text{-th}$ solution\\

&\cellcolor{white!20}\scriptsize $\ddot{o}_{\scriptscriptstyle j}$&\cellcolor{white!20}\scriptsize $j\text{-th}$ puzzle, $\ddot{o}_{\scriptscriptstyle j}:(o_{\scriptscriptstyle j,1},o_{\scriptscriptstyle j,2})$\\

&\cellcolor{gray!20}\scriptsize $f_{\scriptscriptstyle j}$&\cellcolor{gray!20}\scriptsize Time  when  $j{\text{-th}}$ solution is found\\

&\cellcolor{white!20}\scriptsize $k, k_{\scriptscriptstyle j}$&\cellcolor{white!20}\scriptsize Sym. key encryption keys\\

&\cellcolor{gray!20}\scriptsize $pk, sk$&\cellcolor{gray!20}\scriptsize Public and secret keys\\

&\cellcolor{white!20}\scriptsize $q_{\scriptscriptstyle 1}, q_{\scriptscriptstyle 2}$&\cellcolor{white!20}\scriptsize Large prime numbers\\

&\cellcolor{gray!20}\scriptsize $N$&\cellcolor{gray!20}\scriptsize RSA modulus, $N=q_{\scriptscriptstyle 1} q_{\scriptscriptstyle 2}$\\

\hline 

   

\end{tabular}

% 2nd table
\begin{tabular}{|c|c|c|c|c|c|c|c|c|c|c|c|c|c|} 
    \hline
\cellcolor{gray!15}\scriptsize \textbf{Setting} &\cellcolor{gray!15} \scriptsize \textbf{Symbol}&\cellcolor{gray!15} \scriptsize \textbf{Description}  \\
    \hline
    
\hline



\hline


%SO-PoR right
 \multirow{20}{*}{\rotatebox[origin=c]{90}{\scriptsize \textbf{SO-PoR}}}
&\cellcolor{gray!20}\scriptsize$\mathtt{PRF}$&\cellcolor{gray!20}\scriptsize Pseudorandom function\\  
                   
    &\cellcolor{white!20}\scriptsize$\hat{k},v_{\scriptscriptstyle j},l_{\scriptscriptstyle j}$&\cellcolor{white!20}\scriptsize $\mathtt{PRF}$'s keys\\ 
&\cellcolor{gray!20}\scriptsize$\iota$&\cellcolor{gray!20}\scriptsize Security parameter, $\iota=128$-bit\\ 
&\cellcolor{white!20}\scriptsize$p$&\cellcolor{white!20}\scriptsize Large prime number, $|p|=\iota$\\ 

&\cellcolor{gray!20}\scriptsize$w$&\cellcolor{gray!20}\scriptsize  Blockchain block index\\ 
                    
&\cellcolor{white!20}\scriptsize$g$&\cellcolor{white!20}\scriptsize Blockchain security parameter: chain quality    \\    
      
  &\cellcolor{gray!20}\scriptsize$\lambda'$&\cellcolor{gray!20}\scriptsize Blockchain generic security parameter\\  
&\cellcolor{white!20}\scriptsize${\bm{F}}$&\cellcolor{white!20}\scriptsize Outsourced encoded file\\ 
&\cellcolor{gray!20}\scriptsize$F_{\scriptscriptstyle j}$&\cellcolor{gray!20}\scriptsize A file block\\ 
&\cellcolor{white!20}\scriptsize$|{\bm{F}}|$&\cellcolor{white!20}\scriptsize Number of file blocks, $|{\bm{F}}|=n$\\ 
 &\cellcolor{gray!20}\scriptsize$||{\bm{F}}||$&\cellcolor{gray!20}\scriptsize File bit-size\\     
% \multirow{8}{*}{\rotatebox[origin=c]{90}{\scriptsize \textbf{SO-PoR}}}

&\cellcolor{white!20}\scriptsize$\sigma_{\scriptscriptstyle i}$&\cellcolor{white!20}\scriptsize Permanent tag   \\  
 &\cellcolor{gray!20}\scriptsize$\sigma_{\scriptscriptstyle b,j}$&\cellcolor{gray!20}\scriptsize Disposable tag   \\    
 
    &\cellcolor{white!20}\scriptsize$\alpha, \alpha_{\scriptscriptstyle j}, r_{\scriptscriptstyle i},r_{\scriptscriptstyle b,j}$&\cellcolor{white!20}\scriptsize Pseudorandom values  \\ 
                
&\cellcolor{gray!20}\scriptsize$c$&\cellcolor{gray!20}\scriptsize Number of challenges \\ 
        
&\cellcolor{white!20}\scriptsize$\mathcal {B}_{\scriptscriptstyle j}$&\cellcolor{white!20}\scriptsize Blockchain's $j{\text{-th}}$ block\\ 

&\cellcolor{gray!20}\scriptsize$(\mu_{\scriptscriptstyle j},\xi_{\scriptscriptstyle j})$&\cellcolor{gray!20}\scriptsize $j{\text{-th}}$ PoR proof\\ 
&\cellcolor{white!20}\scriptsize$\Delta_{\scriptscriptstyle 1}$&\cellcolor{white!20}\scriptsize Time taken to generate  a PoR\\ 

&\cellcolor{gray!20}\scriptsize$\Delta_{\scriptscriptstyle 2}$&\cellcolor{gray!20}\cellcolor{gray!20}\scriptsize Time taken a contract gets a message\\ 

&\cellcolor{white!20}\scriptsize$e$&\cellcolor{white!20}\scriptsize Coins paid for an accepting PoR\\ 

\hline
   

           
           
           
\end{tabular}\label{table:notation-table}

}}
\end{center}
\end{scriptsize}
\end{table*}








%%%%%%%%%%%%%%%%%%%%%%%%%%%%%%%%%%%%%%%%%%%
























% !TEX root =main.tex



\vspace{-3mm}
\section{Multi-instance  Time-lock Puzzle}

\vspace{-4mm}


\subsection{Strawman Solution}\label{C-TLP-overview}

\vspace{-3mm}

In the following, we elaborate on the  problems that would arise if an existing time-lock puzzle is used directly to handle  multiple puzzles at once.  Without loss of generality, to illustrate the problems, we use the well-known TLP scheme presented in Section \ref{Time-lock-Encryption}. 



Consider the case where a client wants a server to learn a vector of messages: $\vv{\bm{m}}=[m_{\scriptscriptstyle 1},...,m_{\scriptscriptstyle z}]$ at times  $[f_{\scriptscriptstyle 1},...,f_{\scriptscriptstyle z}]$ respectively, where the client is available and online only at an earlier time $f_{\scriptscriptstyle 0}< f_{\scriptscriptstyle 1}$.  For the sake of simplicity, let $\Delta=f_{\scriptscriptstyle 1}-f_{\scriptscriptstyle 0}$ and $\Delta=f_{\scriptscriptstyle j+1}-f_{\scriptscriptstyle j}$, where $1\leq j \leq z$. A naive way to address the problem is that the client uses the TLP  to encrypt each message $m_{\scriptscriptstyle j}$ separately, such that it can be decrypted at time $f_{\scriptscriptstyle j}$ if  all ciphertexts and public keys are passed on to the server at time $t_{\scriptscriptstyle 0}$.  For the server to decrypt the messages  on time, it needs to start decrypting \emph{all of them} as soon as the ciphertexts and public keys are given to it. 



\noindent\textit{\textbf{Parallel Composition Problem}}. The above naive approach yields two serious issues: (a) imposing a high computation cost, as  the server has to perform $S\Delta \sum\limits_{\scriptscriptstyle j=1}^{\scriptscriptstyle z}j$ squaring to decrypt all   messages, and (b) demanding a high level of parallelisation, as each puzzle has to be dealt with separately in parallel to the rest.  The  issues can be cast  as  ``\emph{parallel composition problem}'', where $z$ instances of a puzzle scheme are given at once to a server whose only option, to find solutions on time, is to solve them in parallel.\footnote{It should not be confused with the ``universally composable'' notion put forth in \cite{Canetti01}.} Also, for the client  to efficiently compute $a_{\scriptscriptstyle j}$  for each  message $m_{\scriptscriptstyle j}$,  where $j>1$, it has to perform at least one modular multiplication, i.e. $a_{\scriptscriptstyle j}=a_{\scriptscriptstyle 1} a_{\scriptscriptstyle j-1}=2^{\scriptscriptstyle j  T}$, where $a_{\scriptscriptstyle 1}=2^{\scriptscriptstyle T}$. In this step, in total $z-1$ modular multiplications are required  to compute all $a_{\scriptscriptstyle j}$ values, for $z$ messages (which is not optimal). Note, we do not see the above issues as  previous schemes' flaws, because they were not initially designed for the multi-puzzle setting.  
 
\vspace{-3mm}



 \subsection{An Overview of our Solutions}\label{Overview-of-our-Solutions}
 
 \vspace{-2mm}
 
 Our key observation is, in the naive approach, the process of decrypting  messages has many overlaps  leading to a high  computation cost. So,  by removing the overlaps, we can considerably lower the overall cost both in \emph{puzzle solving} and \emph{puzzle creating} phases.  One of our core ideas  is to chain the puzzles. While chaining different puzzles may seem a relatively obvious approach to tackle the  issues, designing a secure protocol that also can make black-box use of a standard time-lock puzzle scheme, supports public verifiability, and has low costs is challenging (we refer readers to Remark \ref{remark::trivial-chaining} for a detailed discussion). In our solution, a client  first encrypts the message  that is supposed to be decrypted after the rest and embeds the information needed for decrypting it into the ciphertext of the message that will be decrypted before that message. In other words, the client integrates the information (i.e. a part of public keys) needed to decrypt message $m_{\scriptscriptstyle j}$ into the ciphertext related to message $m_{\scriptscriptstyle j-1}$. In this case, the server after learning message $m_{\scriptscriptstyle j-1}$ at time $f_{\scriptscriptstyle j-1}$ learns the public key needed to perform the sequential squaring to decrypt the next message: $m_{\scriptscriptstyle j}$. This means after fully decrypting $m_{\scriptscriptstyle j-1}$, the server starts  squaring sequentially to decrypt $m_{\scriptscriptstyle j}$
  
  
  
  \noindent\textit{\textbf{Addressing Parallel Composition Problem}}.  The above approach solves the parallel composition problem for two main reasons.  First, the total  number of squaring required to decrypt all $z$ messages is now much lower, i.e. $S \Delta z$, and is equivalent to the number of squaring needed to solve only the last puzzle,  i.e. $z\text{-th}$ one. Second, it does not call for  high parallelisation. Because now the server does not need to deal with all of the puzzles in parallel; instead, it solves them sequentially one after another.  
  
  
  
  \noindent\textit{\textbf{Adding Efficient Publicly Verifiable Algorithm}}. To let the  scheme  support  efficient public verifiability, we use the following novel trick. The client uses a commitment scheme to commit to every message: $m_{\scriptscriptstyle i}$ and publishes the  commitment. Then, it uses the time-lock encryption to encrypt the commitment's opening, i.e. a combination of $m_{\scriptscriptstyle i}$ and a random value. But, unlike the traditional commitment, the client does not open the commitment itself. Instead, the server does that, after it discovers the puzzle's solution.  When it finds a solution, it decodes the solution to find the opening and sends it to the public who can check the solution correctness. So, to verify  a solution's correctness,   a verifier  only needs to run the commitment's verification algorithm that is: (a)  publicly verifiable, and (b)   efficient. It can be built in the random oracle  or  the standard model.
  
    The approach also allows  the client at the setup to compute only a single $a=2^{\scriptscriptstyle T}$  reusable for all $z$ puzzles, imposing only $O(1)$  cost. 

\vspace{-5mm}

\subsection{Multi-instance   Time-lock Puzzle Definition}\label{Section::Multi-instance-Time-lock Puzzle-Definition}

\vspace{-1mm}

In this section, we provide a formal definition of a multi-instance time-lock puzzle. Our starting point is the  time-lock puzzle definition, i.e. Definition \ref{Def::Time-lock-Puzzle}, but we extend it from several  perspectives, so it can: (a) handle multiple  solutions/messages in setup, (b)  produce multiple puzzles for the messages,   (c) solve the puzzles given the puzzles and public parameters, and (d) support public verifiability. In the following, we provide the formal definition of a multi-instance  time-lock puzzle.
\begin{definition}[Multi-instance Time-lock Puzzle] A multi-instance time-lock puzzle has the following  five algorithms and satisfies completeness and efficiency properties. 



\begin{itemize}[leftmargin=.43cm]
\item \textbf{Algorithms}:
\begin{itemize} 
\item[$\bullet$]$\mathtt{Setup}(1^{\scriptscriptstyle\lambda},\Delta,z)\rightarrow (pk,sk,\vv{\bm{d}})$:  a probabilistic algorithm that takes as input  security: $1^{\scriptscriptstyle\lambda}$ and time:  $\Delta$ parameters and the total number of solutions/puzzles: $z$. Let     $j \Delta$ be a time period after which $j\text{\small{-th}}$ solution is found.   It outputs public-private key pair: $(pk,sk)$ and a vector of fixed size  secret witnesses: $\vv{\bm{d}}$


%\vv{\bm{s}}
\item[$\bullet$]$\mathtt {GenPuz}(\vv{\bm{m}}, pk, sk,\vv{\bm{d}})\rightarrow \ddot{o}$:  a probabilistic algorithm that takes as  input  a  message vector: $\vv{\bm{m}}=[m_{\scriptscriptstyle 1},...,m_{\scriptscriptstyle z}]$,  the public-private key pair: $(pk,sk)$, and the witness vector: $\vv{\bm{d}}$. It  outputs $\ddot{o}:(\vv{\bm{o}},\vv{\bm{h}})$, where $\vv{\bm{o}}$ is a puzzle vector, and $\vv{\bm{h}}$ is a commitment vector. Each $j\text{\small{-th}}$ element in  vectors $\vv{\bm{o}}$ and $\vv{\bm{h}}$ corresponds to a solution $s_{\scriptscriptstyle j}$ of the form: $s_{\scriptscriptstyle j}=m_{\scriptscriptstyle j}||d_{\scriptscriptstyle j}$ %Given $s_{\scriptscriptstyle j}$ and $b$, a public decoding function, $\mathtt{Decode}()$ returns $m_{\scriptscriptstyle j}$, i.e. $\mathtt{Decode}(s_{\scriptscriptstyle j},b)\rightarrow m_{\scriptscriptstyle j}$. 
 
\item[$\bullet$]$\mathtt {SolvPuz}(pk,\vv{\bm{o}})\rightarrow \vv{\bm{s}}$:   a deterministic algorithm that takes as input  the public key: $pk$ and  puzzle vector: $\vv{\bm{o}}$. It outputs a solution vector: $\vv{\bm{s}}$

\item[$\bullet$]$\mathtt {Prove}(pk,s_{\scriptscriptstyle j})\rightarrow \ddot{p}_{\scriptscriptstyle j}$:  a deterministic algorithm that takes the public key: $pk$ and a solution: $s_{\scriptscriptstyle j}\in\vv{\bm{s}}$. It outputs a proof, $\ddot{p}_{\scriptscriptstyle j}:(m_{\scriptscriptstyle j},d_{\scriptscriptstyle j})$ given to the verifier.

\item[$\bullet$]$\mathtt {Verify}(pk,\ddot{p}_{\scriptscriptstyle j},h_{\scriptscriptstyle j})\rightarrow \{0,1\}$:  a deterministic algorithm that takes  public key: $pk$,  proof: $\ddot{p}_{\scriptscriptstyle j}$ and commitment: $h_{\scriptscriptstyle j}\in \vv{\bm{h}}$. It outputs  $0$ if it rejects, or $1$ if it accepts. 
\end{itemize}
\item \textbf{Completeness}: for any honest prover and verifier, it always holds that: 
\begin{itemize}
\item$\mathtt{SolvPuz}(pk,[o_{\scriptscriptstyle 1},...,o_{\scriptscriptstyle j}])=[s_{\scriptscriptstyle1},...,s_{\scriptscriptstyle j}]$, for every $j$, $1\leq j\leq z$

\item $\mathtt {Verify}(pk,\mathtt {Prove}(pk,s_{\scriptscriptstyle j}),h_{\scriptscriptstyle j})\rightarrow 1$
\end{itemize}
\item \textbf{Efficiency}: the run-time of algorithm $\mathtt {SolvPuz}(pk,[o_{\scriptscriptstyle 1},...,o_{\scriptscriptstyle j}])=[s_{\scriptscriptstyle1},...s_{\scriptscriptstyle j}]$ is bounded by:  $ poly(j\Delta,\lambda)$, where $poly(.)$ is a fixed polynomial and  $1\leq j\leq z$
\end{itemize}
\end{definition}
 
Informally, a multi-instance time-lock puzzle is secure if it satisfies two properties:  a solution's \emph{privacy} and  \emph{validity}. The former  requires  its $j\text{\small{-th}}$ solution   to remain hidden from all adversaries running in parallel within  time period: $j \Delta$, while the latter one requires that it is  infeasible for  a PPT adversary to come up with an invalid solution  and passes the verification. The two properties are formally defined in Definitions \ref{Def::Solution-Privacy} and \ref{Def::Solution-Validity}.
 

 
 
 
% \begin{definition}[Chained Time-lock Puzzle's Sequentiality] For functions  $\pi(t)$ and $\delta(t)$, a  chained time-lock puzzle is $(\pi,\delta)$-sequential if for any pair of randomised algorithm $\mathcal{A} : (\mathcal{A}_{\scriptscriptstyle 1},\mathcal{A}_{\scriptscriptstyle 2})$, where $\mathcal{A}_{\scriptscriptstyle 1}$ runs in total time $O(poly(t,\lambda))$ and $\mathcal{A}_{\scriptscriptstyle 2}$ runs in  time $\delta(t)$ using at most $\pi(t)$ parallel processors, there exists a negligible function $\mu(.)$ such that: 
% 
 
 
% $$ Pr\left[    \begin{array}{l}  \mathcal{A}_{\scriptscriptstyle 2}(pk, \ddot{o},state)\rightarrow s \\
% s.t.\\
% s=\mathtt {SolvPuz}(pk,\theta)
% 
%   \end{array}
%   \middle |
%    \begin{array}{l}
%\mathtt{Setup}(1^{\scriptscriptstyle\lambda},\Delta,1)\rightarrow (pk,sk,\vv{\bm{d}})\\
%\mathcal{A}_{\scriptscriptstyle 1}(1^{\scriptscriptstyle\lambda},pk, \Delta,1)\rightarrow state\\
%m\stackrel{\scriptscriptstyle\$}\leftarrow \mathcal{M}\\
%\mathtt {GenPuz}(m, pk, sk)\rightarrow \ddot{o}\\
%\end{array}    \right]\leq \mu(\lambda)$$
%  \end{definition}
%  where $\theta\in \ddot{o}$.
  
%   $$ Pr\left[    \begin{array}{l}  \mathcal{A}_{\scriptscriptstyle 2}(pk, \ddot{o},state)\rightarrow a \\
% s.t.\\
%m'=\mathtt{Decode}(\mathtt {SolvPuz}(pk,\theta),b)\\
% a=m'
% 
%   \end{array}
%   \middle |
%    \begin{array}{l}
%\mathtt{Setup}(1^{\scriptscriptstyle\lambda},\Delta,1)\rightarrow (pk,sk,\vv{\bm{d}})\\
%\mathcal{A}_{\scriptscriptstyle 1}(1^{\scriptscriptstyle\lambda},pk, \Delta,1)\rightarrow state\\
%m\stackrel{\scriptscriptstyle\$}\leftarrow \mathcal{M}\\
%\mathtt {GenPuz}(m, pk, sk)\rightarrow \ddot{o}\\
%\end{array}    \right]\leq \mu(\lambda)$$
%  \end{definition}
%  where $\theta\in \ddot{o}$.
%  
%  
%  
%  xxx The above definition also captures the sequentiality for a single solution as well that means the adversary cannot find a single solution significantly less than required steps. 
%  
 
   \vspace{-1mm}
  
\begin{definition}[Multi-instance Time-lock Puzzle's Solution-Privacy]\label{Def::Solution-Privacy} A multi-instance time-lock puzzle  is privacy-preserving  if for all $\lambda$ and  $\Delta$,  any number of puzzle: $z\geq1$, any $j$ (where $1\leq j \leq z$), any pair of randomised algorithm $\mathcal{A} : (\mathcal{A}_{\scriptscriptstyle 1},\mathcal{A}_{\scriptscriptstyle 2})$, where $\mathcal{A}_{\scriptscriptstyle 1}$ runs in  time $O(poly(j\Delta,\lambda))$ and $\mathcal{A}_{\scriptscriptstyle 2}$ runs in  time $\delta(j\Delta)<j\Delta$ using at most $\pi(\Delta)$ parallel processors, there exists a negligible function $\mu(.)$, such that: 
\small{
$$ Pr\left[  \begin{array}{l} 
 \mathcal{A}_{\scriptscriptstyle 2}(pk,\ddot{o},\text{state})\rightarrow \ddot{a}\\
 \text{s.t.}\\
\ddot{a}:(b_{\scriptscriptstyle i},i)\\
  m_{\scriptscriptstyle b_{\scriptscriptstyle i},i}=m_{\scriptscriptstyle b_{\scriptscriptstyle j},j} 
  \end{array}
 \middle |
    \begin{array}{l}
\mathtt{Setup}(1^{\scriptscriptstyle\lambda},\Delta,z)\rightarrow (pk,sk,\vv{\bm{d}})\\
%\mathcal{A}_{\scriptscriptstyle 1}(1^{\scriptscriptstyle\lambda},pk,z)\rightarrow ([(m_{\scriptscriptstyle 0,1},m_{\scriptscriptstyle 1,1}),...,(m_{\scriptscriptstyle 0,z},m_{\scriptscriptstyle 1,z})],state)\\
\mathcal{A}_{\scriptscriptstyle 1}(1^{\scriptscriptstyle\lambda},pk,z)\rightarrow (\vv{\bm{m}},\text{state})\\

\forall j', 1\leq j' \leq z: b_{\scriptscriptstyle j'}\stackrel{\scriptscriptstyle\$}\leftarrow \{0,1\}\\
%\left[b_{\scriptscriptstyle 1},...,b_{\scriptscriptstyle z}\right], b_{\scriptscriptstyle j}\stackrel{\scriptscriptstyle\$}\leftarrow \{0,1\}\\
%\mathtt {GenPuz}((m_{\scriptscriptstyle b_{\scriptscriptstyle 1},\scriptscriptstyle 1},..., m_{\scriptscriptstyle b_{\scriptscriptstyle z},\scriptscriptstyle z}), pk, sk,\vv{\bm{d}})\rightarrow \ddot{o}\\
\mathtt {GenPuz}(\vv{\bm{m}}', pk, sk,\vv{\bm{d}})\rightarrow \ddot{o}\\
\end{array}    \right]\leq \frac{1}{2}+\mu(\lambda)$$
}
where  $\vv{\bm{m}}: [(m_{\scriptscriptstyle 0,1},m_{\scriptscriptstyle 1,1}),...,(m_{\scriptscriptstyle 0,z},m_{\scriptscriptstyle 1,z})]$, $\vv{\bm{m}}':(m_{\scriptscriptstyle b_{\scriptscriptstyle 1},\scriptscriptstyle 1},..., m_{\scriptscriptstyle b_{\scriptscriptstyle z},\scriptscriptstyle z})$,  and $1\leq i\leq z$
%$b_{\scriptscriptstyle j'}\in \left[b_{\scriptscriptstyle 1},...,b_{\scriptscriptstyle z}\right]$. 
\end{definition}

%all probabilistic polynomial time adversaries $\mathcal{A}=(\mathcal{A}_{\scriptscriptstyle 1},\mathcal{A}_{\scriptscriptstyle 2})$ whose run-time is  bounded by  $T_{\scriptscriptstyle j}=j\cdot poly(\lambda,\Delta)$, where  $j\in [ 1,z]$, 


The  definition above also ensures  the  solutions to appear after $j\text{\small{-th}}$ one,  remain hidden from the adversary with a high probability, as well. Similar to \cite{BonehBBF18,MalavoltaT19,garay2019}, it captures that even if     $\mathcal{A}_{\scriptscriptstyle 1}$ computes on the public parameters for a polynomial time,  $\mathcal{A}_{\scriptscriptstyle 2}$  cannot find $j\text{\small{-th}}$  solution in time $\delta(j\Delta)<j\Delta$ using $\pi(\Delta)$ parallel processors, with a probability significantly greater than $\frac{1}{2}$. As highlighted in  \cite{BonehBBF18}, we can set $\delta(\Delta)=(1-\epsilon)\Delta$ for a small  $\epsilon$, where $0<\epsilon<1$
\begin{definition}[Multi-instance Time-lock Puzzle's Solution-Validity]\label{Def::Solution-Validity}
A multi-instance time-lock puzzle preserves a   solution validity,   if  for all $\lambda$ and  $\Delta$,  any number of puzzles: $z\geq1$, all probabilistic polynomial-time adversaries $\mathcal{A}=(\mathcal{A}_{\scriptscriptstyle 1},\mathcal{A}_{\scriptscriptstyle 2})$ that run in  time $O(poly(\Delta,\lambda))$ there is  negligible function $\mu(.)$, such that: 
\small{
$$ Pr\left[
    \begin{array}{l}
 \mathcal{A}_{\scriptscriptstyle 2}(pk,\vv{\bm{s}}, \ddot{o},\text{state})\rightarrow a\\ 
 
 \text{s.t.}\\ 
a:(j,\ddot{p}_{\scriptscriptstyle j} ,\ddot{p}')\\
 \ddot{p}_{\scriptscriptstyle j}: (m_{\scriptscriptstyle j},d_{\scriptscriptstyle j}), 
\ddot{p}':(m',d') \\
 m_{\scriptscriptstyle j}\in \vv{{\bm{m}}}, d_{\scriptscriptstyle j}\in\vv{{\bm{d}}},
m\neq m'\\
\mathtt {Verify}(pk,\ddot{p},h_{\scriptscriptstyle j})= 1\\
\mathtt {Verify}(pk,\ddot{p}',h_{\scriptscriptstyle j})= 1\\
\end{array} 
\middle |
\begin{array}{l}

\mathtt{Setup}(1^{\scriptscriptstyle\lambda},\Delta,z)\rightarrow (pk,sk,\vv{\bm{d}})\\
\mathcal{A}_{\scriptscriptstyle 1}(1^{\scriptscriptstyle\lambda},pk, \Delta,z)\rightarrow (\vv{{\bm{m}}},\text{state})\\

\mathtt {GenPuz}(\vv{{\bm{m}}}, pk, sk,\vv{\bm{d}})\rightarrow \ddot{o} \\
\mathtt {SolvPuz}(pk,\vv{\bm{o}})\rightarrow \vv{\bm{s}}

\end{array} 
   \right]\leq  \mu(\lambda)$$
   }
where $\vv{{\bm{m}}}=[m_{\scriptscriptstyle 1},...,m_{\scriptscriptstyle z}]$, and $h_{\scriptscriptstyle j}\in \vv{\bm{h}}\in \ddot{o}$
\end{definition}


%In Definition \ref{Def::Solution-Validity}, we do not need to bound the adversaries' parallel computation power, as it does not need to solve any puzzles, in fact  puzzles' solutions are provided to them. Therefore, they can run in polynomial time $O(poly(\Delta,\lambda))$.
% 
\begin{definition}[Multi-instance Time-lock Puzzle Security]\label{def::C-TLP-security} A multi-instance time-lock puzzle scheme  is secure if it meets solution-privacy and solution-validity properties. 
\end{definition}

\vspace{-6mm}

\subsection{Chained  Time-lock Puzzle (C-TLP) Protocol}\label{Section::C-TLP-protocol}

\vspace{-1mm}

In this section, we present the chained  time-lock puzzle (C-TLP), an instantiation of the multi-instance time lock puzzle. Since we have already presented an outline of C-TLP (in Section \ref{Overview-of-our-Solutions}), in this section we present C-TLP protocol in detail.  Recall, a client wants a server to learn a vector of messages: $\vv{\bm{m}}=[m_{\scriptscriptstyle 1},...,m_{\scriptscriptstyle z}]$ at times  $[f_{\scriptscriptstyle 1},...,f_{\scriptscriptstyle z}]$ respectively, where the client is available and online only at an earlier time $f_{\scriptscriptstyle 0}< f_{\scriptscriptstyle 1}$.  Also, the client wants to ensure that anyone can validate a solution found by the  server, i.e. supports public verifiability. For the sake of simplicity, let $\Delta=f_{\scriptscriptstyle 1}-f_{\scriptscriptstyle 0}$ and $\Delta=f_{\scriptscriptstyle j+1}-f_{\scriptscriptstyle j}$, where $1\leq j \leq z$ and $T=S \Delta$. Below, we provide C-TLP protocol. We refer readers to Appendix \ref{discussion-C-TLP} for further remarks on the protocol. 

% !TEX root =main.tex



  %\begin{figure}

%\centering

%\small{

%\begin{boxedminipage}{\columnwidth}


\begin{enumerate}[leftmargin=.4cm]

\item\textbf{Setup}: $\mathtt{Setup}(1^{\scriptscriptstyle\lambda}, \Delta,z)$.  
\begin{enumerate}

\item\label{call-RTLP-Setup} Call:  $\mathtt{TLP.Setup}(1^{\scriptscriptstyle\lambda}, \Delta)\rightarrow (\hat{pk},\hat{sk})$, s.t.  $\hat{pk}=(N,T,r_{\scriptscriptstyle 1})$ and $\hat{sk}=(q_{\scriptscriptstyle 1},q_{\scriptscriptstyle 2},a,k_{\scriptscriptstyle 1})$

\item Pick  $z-1$ fixed size  random generators: $\vv{\bm{r}}=[r_{\scriptscriptstyle 2},...,r_{\scriptscriptstyle z}]$ from $\mathbb{Z}^{\scriptscriptstyle *}_{ \scriptscriptstyle N}$


\item Pick  $z-1$ random keys: $[k_{\scriptscriptstyle 2},...,k_{\scriptscriptstyle z}]$ for a symmetric key encryption. Let $\vv{\bm{k}}=[k_{\scriptscriptstyle 1},...,k_{\scriptscriptstyle z}]$, where $k_{\scriptscriptstyle 1}\in \hat{sk}$. Also, pick $z$ fixed size sufficiently large random values: $\vv{\bm{d}}=[d_{\scriptscriptstyle 1},...,d_{\scriptscriptstyle z}]$, e.g. $|d_{\scriptscriptstyle j}|=128$-bit or $1024$-bit depending on the choice of a commitment scheme.  

\item Set $pk=( \text{aux},N,T, r_{\scriptscriptstyle 1})$ as  public key. Set $sk=(q_{\scriptscriptstyle 1},q_{\scriptscriptstyle 2},a, \vv{\bm{k}},\vv{\bm{r}},\vv{\bm{d}})$ as secret key. Note, $\text{aux}$ contains a cryptographic hash function's description and the size of the random values. Also,  note that  all generators, except $r_{\scriptscriptstyle 1}$ are kept secret. Output $pk$ and $sk$

\end{enumerate}
\item\label{Generate-Puzzle}\textbf{Generate Puzzle}: $\mathtt{GenPuz}(\vv{\bm{m}}, pk,sk)$ 


Encrypt the  messages, starting with $j=z$, in descending order. $\forall j, z\geq j \geq 1:$
\begin{enumerate}
\item\label{set-pk-in-loop} Set $pk_{\scriptscriptstyle j}=(N,T,r_{\scriptscriptstyle j})$ and $sk_{\scriptscriptstyle j}=(q_{\scriptscriptstyle 1},q_{\scriptscriptstyle 2},a,k_{\scriptscriptstyle j})$. Note, if $j=1$ then $r_{\scriptscriptstyle j} \in pk$; otherwise (when $j>1$), $r_{\scriptscriptstyle j} \in \vv{\bm{r}}$


\item\label{call-RTLP-GenPuz} Generate a puzzle (or ciphertext pair): 
\begin{itemize}
\item[$\bullet$]  if $j=z$, then run: $\mathtt{TLP.GenPuz}(m_{\scriptscriptstyle j}||d_{\scriptscriptstyle j},pk_{\scriptscriptstyle j},sk_{\scriptscriptstyle j})\rightarrow \ddot{o}_{\scriptscriptstyle j}=(o_{\scriptscriptstyle j,1},o_{\scriptscriptstyle j,2})$
 
\item[$\bullet$]  otherwise, run: $\mathtt{TLP.GenPuz}(m_{\scriptscriptstyle j}||d_{\scriptscriptstyle j}||r_{\scriptscriptstyle j+1},pk_{\scriptscriptstyle j},sk_{\scriptscriptstyle j})\rightarrow \ddot{o}_{\scriptscriptstyle j}=(o_{\scriptscriptstyle j,1},o_{\scriptscriptstyle j,2})$
\end{itemize}
%Recall that, in TLP,  $\ddot{o}_{\scriptscriptstyle j}=(o_{\scriptscriptstyle j,1},o_{\scriptscriptstyle j,2})$


\item\label{commit-} Commit to each message, e.g. $\mathtt{H}(m_{\scriptscriptstyle j}||d_{\scriptscriptstyle j})=h_{\scriptscriptstyle j}$ and output:  $h_{\scriptscriptstyle j}$ 
 
\item Output: $\ddot{o}_{\scriptscriptstyle j}=(o_{\scriptscriptstyle j,1},o_{\scriptscriptstyle j,2})$ as puzzle (or ciphertext pair). 

\end{enumerate}
By the end of this phase,  vectors of puzzles: $\vv{\bm{o}}=[\ddot{o}_{\scriptscriptstyle 1},..., \ddot{o}_{\scriptscriptstyle z}]$ and commitments: $\vv{\bm{h}}=[h_{\scriptscriptstyle 1},...h_{\scriptscriptstyle z}]$ are generated. All public parameters and puzzles are given to a server at time $t_{\scriptscriptstyle 0}<t_{\scriptscriptstyle1}$, where  $\Delta=f_{\scriptscriptstyle 1}-f_{\scriptscriptstyle 0}$ %The public parameters and $\vv{\bm{h}}$ are given to public verifiers.




\item\textbf{Solve Puzzle}:  $\mathtt{SolvPuz}(pk, \vv{\bm{o}})$ 

 Decrypt the messages, starting with $j=1$, in ascending order.  $\forall j, 1\leq j\leq z:$
 
\begin{enumerate}
\item If $j=1$, then set $r_{\scriptscriptstyle j}=r_{\scriptscriptstyle 1}$, where $r_{\scriptscriptstyle 1}\in pk$; Otherwise, set $r_{\scriptscriptstyle j}=u$

\item Set  $pk_{\scriptscriptstyle j}=(N,T,r_{\scriptscriptstyle j})$

\item\label{call-RTLP-SolvPuz} Run: $\mathtt{TLP.SolvPuz}(pk_{\scriptscriptstyle j},\ddot{o}_{\scriptscriptstyle j})\rightarrow x_{\scriptscriptstyle j}$, where $\ddot{o}_{\scriptscriptstyle j}\in \vv{\bm{o}}$


\item\label{Dec-message} Parse $x_{\scriptscriptstyle j}$. Note that if $j<z$ then $x_{\scriptscriptstyle j}=m_{\scriptscriptstyle j}||d_{\scriptscriptstyle j}||r_{\scriptscriptstyle j+1}$; otherwise,  we have $x_{\scriptscriptstyle j}=m_{\scriptscriptstyle j}||d_{\scriptscriptstyle j}$. Therefore, $x_{\scriptscriptstyle j}$ is parsed as follows.




\begin{itemize}
\item[$\bullet$] if $j<z: $
\begin{enumerate}

\item Parse $m_{\scriptscriptstyle j}||d_{\scriptscriptstyle j}||r_{\scriptscriptstyle j+1}$ into  $m_{\scriptscriptstyle j}||d_{\scriptscriptstyle j}$ and $u=r_{\scriptscriptstyle j+1}$
\item Output $s_{\scriptscriptstyle j}=m_{\scriptscriptstyle j}||d_{\scriptscriptstyle j}$
\end{enumerate}
\item[$\bullet$] otherwise (when $j=z$),  output $s_{\scriptscriptstyle j}=x_{\scriptscriptstyle j}=m_{\scriptscriptstyle j}||d_{\scriptscriptstyle j}$


\end{itemize}

\end{enumerate}

\item\label{prove-} \textbf{Prove}:  $\mathtt{Prove}(pk, s_{\scriptscriptstyle j})$. Parse  $s_{\scriptscriptstyle j}$ into $\ddot{p}_{\scriptscriptstyle j}: (m_{\scriptscriptstyle j},d_{\scriptscriptstyle j})$, and send the pair to the verifier. 
\item\label{verify-} \textbf{Verify}: $\mathtt{Verify}(pk,\ddot{p}_{\scriptscriptstyle j}, h_{\scriptscriptstyle j})$. Verifies the commitment,  $\mathtt{H}(m_{\scriptscriptstyle j},d_{\scriptscriptstyle j})\stackrel{\scriptscriptstyle?}=h_{\scriptscriptstyle j}$. If passed, accept the solution and output $1$; otherwise,  reject it and output $0$



\end{enumerate}



%\end{boxedminipage}
%}

%\caption{Chained  Time-lock Puzzle (C-TLP) Scheme} 
%\label{fig:CTE}
%\end{figure}
\vspace{-3mm}


\vspace{-3mm}

   \begin{theorem}[C-TLP Security]\label{C-TLP-Sec}  C-TLP  is a secure multi-instance time-lock puzzle. 
   \end{theorem}


\begin{proof}[Outline]
The proof of Theorem \ref{C-TLP-Sec} relies on the security of the TLP, symmetric key encryption, and commitment schemes. It is also based on the fact that the probability to  find a certain random generator is negligible. It shows both C-TLP's solution privacy (due to security of the above three schemes) and validity (due to the security of the commitment) are satisfied.  We refer readers to Appendix \ref{CR-TLP-Proof} for  detailed proof.  \hfill\(\Box\)
\end{proof}

\vspace{-2.5mm}
%
%\begin{remark} Recall, to make each  puzzle instance, a distinct random generator: $r_{\scriptscriptstyle j}$, is used. This is the reason, in  Fig. \ref{fig:CTE}, before a puzzle  is  generated   in step \ref{call-RTLP-GenPuz},  a new public key is set in step \ref{set-pk-in-loop}.  Also, at the beginning of the protocol only $r_{\scriptscriptstyle 1}$ is public and the rest of the generators are kept secret. They are found and used sequentially after their related puzzle is solved. 
%\end{remark}
%
%
%\begin{remark} The commitments opening, including the commitment random values, are not known to other verifiers (than the puzzle generator) at the beginning of the protocol. At this point,  only the committed values are public. Once a solver solves each puzzle,  it  extracts one of the commitments' opening, and sends it to a public verifier who can check if the opening matches the commitment.  
%\end{remark}
%
%\begin{remark}
% In Fig. \ref{fig:CTE}, we use the folklore hash-based commitment scheme, in the random oracle model, only to achieve more computation improvement than that can be achieved in the standard model. But C-TLP can utilise any efficient non-interactive commitment scheme in the \emph{standard model} as well, e.g. Pedersen Commitment.
%\end{remark}
%
%
%\begin{remark}
%The efficiency of  C-TLP scheme stems from three crucial factors: (a) removing computation overlaps when solving different puzzles: even though solving $j\text{\small{-th}}$ puzzle, where $j>1$, requires $jT$ squaring, $(j-1) T$ of the squaring is used to solve previous puzzles that leads to $\frac{z+1}{2}$ times computation cost reduction at the server-side,  (b)  supporting reusable single  public parameter: $a=2^{\scriptscriptstyle T}$, generated only once that costs $O(1)$, as opposed to the RSA TLP whose cost is linear: $O(z)$, and (c) supporting efficient verification: due to the way each message is encoded (i.e. embedding the opening in a solution). 
%\end{remark}
%
%\begin{remark}
%C-TLP also can efficiently  be used in a multi-server setting,  where there are $z$ servers: $\{S_{\scriptscriptstyle 1},...,S_{\scriptscriptstyle z}\}$,  each $S_{\scriptscriptstyle j}$ needs to solve puzzle $\ddot{o}_{\scriptscriptstyle j}$ at time $f_{\scriptscriptstyle j}$ and passes on the solution to the next server $S_{\scriptscriptstyle j+1}$ to solve the next puzzle by time $f_{\scriptscriptstyle j+1}>f_{\scriptscriptstyle j}$. In this setting,  due to the scalability property of C-TLP (and unlike using the existing time-lock puzzles naively), other servers do not need to start solving the puzzle   as soon as the client releases puzzles public parameters. Instead, they can wait until the previous solution is issued that saves them significant cost. Furthermore, a  server can first  verify the correctness  of the solution found by the previous server (due to the public verifiability of C-TLP),  if accepted then   it starts finding the next solution. 
%\end{remark}
%
%
%\begin{remark}
%In the following, we outline an  approach that looks an option to construct an efficient C-TLP; however, as we will show it would not be secure. In particular,  one uses the TLP to generate $z$ public and secret key pairs. Then, it uses the TLP to compute $z\text{\small{-th}}$ puzzle as $\mathtt{TLP.GenPuZ}(m_{\scriptscriptstyle z},pk_{\scriptscriptstyle z},sk_{\scriptscriptstyle z})\rightarrow \ddot{o}_{\scriptscriptstyle z}$.  Then, it embeds $\ddot{o}_{\scriptscriptstyle z}$ into $(z-1)\text{\small{-th}}$ one, i.e. $\mathtt{TLP.GenPuZ}(m_{\scriptscriptstyle z-1}||\ddot{o}_{\scriptscriptstyle z},pk_{\scriptscriptstyle z-1},sk_{\scriptscriptstyle z-1})\rightarrow \ddot{o}_{\scriptscriptstyle z-1}$. This process goes on until $\ddot{o}_{\scriptscriptstyle 1}$  is created. It sends the combined puzzles and public key (including all random generators) to the server; with the hope that puzzles can be solved sequentially and the time gap between finding two solutions will be $\Delta$.  This approach is not secure, because as soon as the server accesses $\ddot{o}_{\scriptscriptstyle 1}$ and public parameters, it can in parallel perform $T$ squaring on every generator, i.e. $r^{\scriptscriptstyle 2^{\scriptscriptstyle T}}_{\scriptscriptstyle i}$, for all $i, 1\leq i\leq z$. In this case, as soon as $\ddot{o}_{\scriptscriptstyle 1}$ is solved and  $\ddot{o}_{\scriptscriptstyle 2}$  is extracted, it has enough information to  immediately solve $\ddot{o}_{\scriptscriptstyle 2}$ and accordingly the rest of the puzzles without doing any further exponentiation. 
 %\end{remark}
 
 
 % !TEX root =main.tex


\vspace{-5mm}
 
 \subsection{Cost Analysis Table}\label{TLP-cost-compare}
 
 We summarize C-TLP's cost analysis in Table \ref{table::puzzle-com}. It  considers a generic setting where the protocol deals with $z$ puzzles. We refer readers to Appendix \ref{TLP-cost-compare} for detailed  analysis. 
 
 \vspace{-7mm}
 
 \begin{table*}[!htbp]
\begin{footnotesize}

\begin{center}
\caption{ \small C-TLP's Detailed Cost Breakdown}\label{table::puzzle-com} 

\renewcommand{\arraystretch}{.85}
\scalebox{0.94}{
\begin{subtable}{.64\linewidth}%xxxx
\begin{minipage}{.88\linewidth}
\caption{\small Computation Cost}
\begin{tabular}{|c|c|c|c|c|c|c|c|c|c|c|c|c|c|c|} 
   \hline
\cellcolor[gray]{0.9}&\cellcolor[gray]{0.9} &
 \multicolumn{3}{c|}{\cellcolor[gray]{0.9}\scriptsize \underline{ \ \ \ \  \ \ \ \ \ Protocol Function \ \ \ \ \ \ \ \ \ }}&\cellcolor[gray]{0.9}\\
 %\cline{3-5}
\cellcolor[gray]{0.9} \multirow{-2}{*}{\scriptsize Protocol}&\cellcolor[gray]{0.9} \multirow{-2}{*} {\scriptsize Operation}&\cellcolor[gray]{0.9}\scriptsize$\mathtt{GenPuz}$&\cellcolor[gray]{0.9}\scriptsize$\mathtt{SolvPuz}$&\cellcolor[gray]{0.9}\scriptsize$\mathtt{Verify}$&\multirow{-2}{*} {\cellcolor[gray]{0.9}\scriptsize   Complexity} \\
\hline
\cellcolor[gray]{0.9} &\multirow{3}{*}{\rotatebox[origin=c]{0}{\scriptsize }} \cellcolor[gray]{0.9}\scriptsize Exp.&\scriptsize$z+1$&\scriptsize$T z$ &$-$&\multirow{4}{*}{\rotatebox[origin=c]{0}{\scriptsize $O(T  z)$}}\\
     \cline{2-5}  
 \cellcolor[gray]{0.9}     &\cellcolor[gray]{0.9}\scriptsize Add. or Mul.&\scriptsize$z$ &\scriptsize$z$&$-$ & \\
     \cline{2-5} 
 \cellcolor[gray]{0.9}         &\cellcolor[gray]{0.9}\scriptsize Commitment&\scriptsize$z$&$-$ &\scriptsize$z$&\\
     \cline{2-5} 
\cellcolor[gray]{0.9}   \multirow{-4}{*}{\rotatebox[origin=c]{0}{\scriptsize  C-TLP }}     &\cellcolor[gray]{0.9}\scriptsize Sym. Enc&\scriptsize$z$&\scriptsize$z$ &$-$&\\

 \hline
\end{tabular}
\end{minipage}%******
\end{subtable}%xxxx

\begin{subtable}{.46\linewidth}%xxxx
\renewcommand{\arraystretch}{1.68}
\begin{minipage}{.9\linewidth}
\caption{\small Communication Cost (in bit)}
\begin{tabular}{|c|c|c|c|c|c|c|c|c|c|c|c|c|c|c|} 
   \hline
 {\cellcolor[gray]{0.9}\scriptsize Protocol}&{\cellcolor[gray]{0.9}\scriptsize Model}&
{\cellcolor[gray]{0.9}\scriptsize Client}&{\cellcolor[gray]{0.9}\scriptsize Server}&{\cellcolor[gray]{0.9}\scriptsize  Complexity}\\
 \cline{3-4}

\hline
 \cellcolor[gray]{0.9}  &\cellcolor[gray]{0.9} \multirow{2}{*}{\rotatebox[origin=c]{0}}\scriptsize Standard&\scriptsize$3200 z$&\scriptsize$1524 z$ &\multirow{2}{*}{\rotatebox[origin=c]{0}{\scriptsize $O(z)$ }}\\
     \cline{2-4}  
  \multirow{-2}{*}{\rotatebox[origin=c]{0}{\cellcolor[gray]{0.9} \scriptsize  C-TLP }}&\cellcolor[gray]{0.9}\scriptsize R.O.&\scriptsize$2432  z$ &\scriptsize$628  z$& \\
   
 \hline
\end{tabular}
\end{minipage}%******
\end{subtable}%xxxx
}
\end{center}
\end{footnotesize}
\end{table*}

 \vspace{-3mm}

% \noindent\textbf{\textit{Computation Complexity}}. For a client to generate $z$ puzzles, it performs $z$ symmetric key-based encryption,  $z$ modular exponentiations  and $z$ modular additions. Also, to commit to values, it invokes a commitment scheme $z$ times; if a hash-based commitment is used then it would involve $z$ invocations of a hash function, and if Pedersen commitment is used then it would involve $2 z$ exponentiations and $z$ multiplications. Thus, the overall computation complexity of the client   is $O(z)$. For the server to solve $z$ puzzles, it calls $\mathtt{TLP.SolvPuz}(.)$ $z$ times, which leads to $O(Tz)$ computation complexity. Also, the verification cost  only involves $z$ invocations of the commitment scheme; if the hash-based commitment is used, then it would involve $z$ invocations of a hash function,  if Pedersen commitment is utilised, then it would involve $2 z$ exponentiations and $z$ multiplications. Thus, the  verification's complexity is $O(z)$.
%
%
%
%
% 
% 
% \noindent\textbf{\textit{Communication Complexity}}.  In step \ref{Generate-Puzzle}, the client publishes two vectors: $\vv{\bm{o}}$ and $\vv{\bm{h}}$, with  $2 z$ and $z$ elements respectively.  Each element of $\vv{\bm{o}}$ is a pair $(o_{\scriptscriptstyle j,1},o_{\scriptscriptstyle j,2})$, where $o_{\scriptscriptstyle j,1}$ is an output of symmetric key encryption, e.g.   $|o_{\scriptscriptstyle j,1}|=128$-bit, and $o_{\scriptscriptstyle j,2}$ is an element of $\mathbb{Z}_{\scriptscriptstyle N}$, e.g.  $|o_{\scriptscriptstyle j,2}|=2048$-bit. Also, each element $h_{\scriptscriptstyle j}$ of $\vv{\bm{h}}$ is either an output of a hash function, when a hash-based commitment is used, e.g.  $|h_{\scriptscriptstyle j}|=256$-bit, or an element of $\mathbb{F}_{\scriptscriptstyle q}$ when Pedersen commitment is used, e.g.  $|h_{\scriptscriptstyle j}|=1024$-bit. Thus, its  bandwidth is about $2432 z$ bits when the former,  or $3200 z$ bits when the latter commitment scheme is utilised. Also, its   complexity is $O(z)$. For the server to prove, in step \ref{prove-},   it sends $z$ pairs $(m_{\scriptscriptstyle j},d_{\scriptscriptstyle j})$ to the verifier, where $m_{\scriptscriptstyle j}$  is an arbitrary message, e.g.  $|m_{\scriptscriptstyle j}|=500$-bit, and  $d_{\scriptscriptstyle j}$ is either a long enough random value, e.g. $|d_{\scriptscriptstyle j}|=128$-bit, when the hash-based commitment is used, or an element of $\mathbb{F}_{\scriptscriptstyle q}$ when Pedersen scheme is used, e.g. $|d_{\scriptscriptstyle j}|=1024$-bit. So, its bandwidth is about either $628 z$ or $1524 z$ bits when the former or latter commitment scheme is used respectively. The solver's  communication complexity is $O(z)$. 
 

 

 % % !TEX root =main.tex

\section{C-TLP Security Proof}\label{CR-TLP-Proof}
 
In this section, we present   the security proof of C-TLP scheme. We first prove that without solving $j\text{\small{-th}}$ puzzle, a solver cannot find the parameters  needed to solve the next puzzle, i.e. $(j+1)\text{\small{-th}}$   one. 

 
  \begin{lemma}[Next Group Generator Privacy]\label{lemma::Next-Generator-Privacy}  Let $k$ be a random key for a symmetric key encryption, and  $N$ be a  sufficiently large RSA modulus. Let  the security parameter be $\lambda=|N|=|k|$.  In C-TLP, given puzzle vector: $\vv{\bm{o}}$ and public key: $pk$, an adversary $\mathcal{A}=(\mathcal{A}_{\scriptscriptstyle 1},\mathcal{A}_{\scriptscriptstyle 2})$, defined in Section \ref{Section::Multi-instance-Time-lock Puzzle-Definition},  cannot find the next group generator: 
$r_{\scriptscriptstyle j+1}$, where $r_{\scriptscriptstyle j+1} \stackrel{\scriptscriptstyle\$}\leftarrow \mathbb{Z}^{\scriptscriptstyle *}_{\scriptscriptstyle N}$and $j\geq1$, significantly smaller than   $T_{\scriptscriptstyle j}=\delta(j\Delta)$, except with a negligible probability in the security parameter, $\mu(\lambda)$ 
  \end{lemma}
 \begin{proof}
Since the next generator: $r_{\scriptscriptstyle j+1}$, is: (a) encrypted along with the $j\text{\small{-th}}$ puzzle solution: $s_{\scriptscriptstyle j}$, and (b)  picked uniformly at random from $\mathbb{Z}^{\scriptscriptstyle *}_{\scriptscriptstyle N}$, for the adversary to find $r_{\scriptscriptstyle j+1}$ without performing enough squaring, i.e. $T_{\scriptscriptstyle j}$, it has to either (a) break the symmetric key scheme, decrypt the related ciphertext: $s_{\scriptscriptstyle i}$ and extract the random value from it, or (b) correctly guess $r_{\scriptscriptstyle j+1}$. In both cases, the probability of success is negligible in secure parameter $\mu(\lambda)$,  i.e. $2^{\scriptscriptstyle -|k|}$ in the former case and $2^{\scriptscriptstyle -|N|}$ in the latter one.  
 \hfill\(\Box\)
  \end{proof} 
%. But its  probability of success is $2^{\scriptscriptstyle -|k|}$ that is negligible in the security parameter,  or  guess $r_{\scriptscriptstyle j+1}$, that has the probability of success  at most $2^{\scriptscriptstyle -|N|}$ which is negligible in $\lambda$ as well.    %\hfill\(\Box\)
 %t \end{proof} 
  
 In the following, we prove that the privacy of a solution in C-TLP scheme is preserved according to Definition \ref{Def::Solution-Privacy}. 
 
 
 \begin{theorem} [C-TLP Solution Privacy]\label{Solution-Privacy} Let $N$ be a  strong RSA modulus and $\Delta$ be a time parameter. If the sequential squaring assumption holds,  factoring $N$ is a hard problem, $\mathtt{H}(.)$ is a random oracle and the symmetric key encryption is  semantically secure, then  C-TLP encoding $z$ solutions is a privacy-preserving multi-instance time-lock puzzle w.r.t. Definition \ref{Def::Solution-Privacy}.
 \end{theorem}
  \begin{proof} In the following, we argue  for an adversary $\mathcal{A}=(\mathcal{A}_{\scriptscriptstyle 1},\mathcal{A}_{\scriptscriptstyle 2})$, where $\mathcal{A}_{\scriptscriptstyle 1}$ runs in total time $O(poly(j\Delta,\lambda))$,  $\mathcal{A}_{\scriptscriptstyle 2}$ runs in  time $\delta(j\Delta)<j\Delta$ using at most $\pi(\Delta)$ parallel processors, and  $j\in [1,z]$,  (a) when $z=1$: to find $s_{\scriptscriptstyle 1}$ earlier than $\delta(\Delta)$,  it has to  break the TLP scheme, and (b) when $z>1$: to find $s_{\scriptscriptstyle j}$ earlier than $T_{\scriptscriptstyle j}=\delta(j\Delta)$, it has to either find   at least one of the previous solutions earlier than it is supposed to (that ultimately requires breaking TLP scheme again), or find $j\text{\small{-th}}$ generator: $r_{\scriptscriptstyle j}$, earlier. Also, we argue that the commitments: $h_{\scriptscriptstyle j}$, are computationally hiding.   Specifically, when $z=1$, the security of C-TLP is reduced to the security of  the TLP and the scheme is secure as long as TLP is, as the two schemes would be identical. On the other hand, when $z>1$, the adversary has to either find $s_{\scriptscriptstyle j}$ earlier than $T_{\scriptscriptstyle j}$ as soon as the previous solution: $s_{\scriptscriptstyle j-1}$ is found that requires either breaking the TLP scheme, or finding any generator $r_{\scriptscriptstyle j}$  before $s_{\scriptscriptstyle j-1}$ is extracted, when $j\in [2,z]$. Nevertheless, the TLP scheme is secure (under RSA,  sequential squaring, and security of symmetric key encryption assumptions) according to  Theorem \ref{theorem::R-LTP-Sec}, and also the probability of finding the next generator: $r_{\scriptscriptstyle j}$ earlier than $T_{\scriptscriptstyle j-1}$ is negligible, according to Lemma \ref{lemma::Next-Generator-Privacy}. Moreover, for an adversary to find a solution earlier, it may also try to find a (partial information of) pre-image of the commitment: $h_{\scriptscriptstyle j}$ before fully (or without) solving the puzzle. But, this is infeasible for a PPT adversary, given  output of a random oracle: $\mathtt{H}(.)$. Thus, C-TLP is a privacy-preserving multi-instance time-lock puzzle scheme.  \hfill\(\Box\)
  \end{proof}
 
Next, we prove that the validity of a solution in C-TLP scheme is preserved according to Definition \ref{Def::Solution-Validity}. 
  \begin{theorem} [C-TLP Solution Validity]\label{Solution-Validity} Let $\mathtt{H}(.)$ be a hash function modeled as a random oracle. Then, C-TLP preserves a solution validity w.r.t. Definition \ref{Def::Solution-Validity}.  
\end{theorem}
\begin{proof}
 The proof  boils down to  the security (i.e. binding property) of the traditional hash-based commitment scheme. In particular, given an  opening pair, $\ddot{p}:(m_{\scriptscriptstyle j},d_{\scriptscriptstyle j})$ and the commitment $h_{\scriptscriptstyle j}=\mathtt{H}(m_{\scriptscriptstyle j},d_{\scriptscriptstyle j})$, for an adversary to break the solution validity, it has to come up $(m'_{\scriptscriptstyle j},d'_{\scriptscriptstyle j})$, such that $\mathtt{H}(m'_{\scriptscriptstyle j},d'_{\scriptscriptstyle j})=h_{\scriptscriptstyle j}$, where $m_{\scriptscriptstyle j}\neq m'_{\scriptscriptstyle j}$, i.e. finds a collision of $\mathtt{H}(.)$. However, this is infeasible for a PPT adversary, as $\mathtt{H}(.)$ is collision resistance, in the random oracle model. 
 \hfill\(\Box\)
\end{proof}
 
 In the following, we restate the main theorem presented in Section \ref{Section::C-TLP-protocol} and then prove it.  

\

\noindent\textbf{Theorem \ref{C-TLP-Sec} (C-TLP Security).} \textit{C-TLP  is a secure multi-instance time-lock puzzle. }

   
 \begin{proof} According to Theorems \ref{Solution-Privacy} and \ref{Solution-Validity}, the privacy and validity of a solution in C-TLP are preserved, respectively.   So, w.r.t. Definition \ref{def::C-TLP-security}, C-TLP is a secure multi-instance  time-lock puzzle.
  \hfill\(\Box\)
\end{proof}


% !TEX root =main.tex


\vspace{-8mm}

\section{Smarter Outsourced PoR (SO-PoR) Using C-TLP}
   %\vspace{-5mm}
 As discussed in Section \ref{Related-Work}, the existing outsourced PoR's  have serious shortcomings, e.g. having high costs, not supporting real-time detection, or suffering from the lack of a fair payment mechanism. In this section, we present SO-PoR to addresses them. 
 
 

 \vspace{-3mm}
 
\subsection{SO-PoR Overview} 

 \vspace{-2.2mm}
 
%SO-PoR uses a unique combination of (a) homomorphic MAC-based PoR \cite{DBLP:conf/asiacrypt/ShachamW08}, (b) C-TLP, and (c) a smart contract. It uses the MAC-based PoR, due to its high efficiency. Since the MAC's are privately verifiable and secret verification keys are needed to check PoR proofs, it also uses C-TLP to efficiently make them publicly verifiable. In this case, C-TLP encapsulates the verification keys and reveals each of them to verifiers only after a certain time. This combination allows the protocol to \emph{take advantage of MAC's efficiency in the setting where public verifiability is needed}. The combination of the two primitives has applications beyond PoR. SO-PoR also utilises a smart contract who acts as a public verifier on the client's behalf to verify proofs and pay an honest server. The MAC's and CTL combination also makes it possible to use a smart contract (which does not inherently support a private state) while keeping costs very low. 

SO-PoR uses a unique combination of (a) homomorphic MAC-based PoR \cite{DBLP:conf/asiacrypt/ShachamW08}, (b) C-TLP,  (c) a smart contract,  (d) a pre-computation technique, and (e) blockchain-based random extraction beacon \cite{DBLP:journals/iacr/AbadiCKZ19,armknecht2014outsourced}. It uses the MAC-based PoR, due to its high efficiency. Since the MAC's are privately verifiable and secret verification keys are needed to check PoR proofs, SO-PoR also uses C-TLP to efficiently make them publicly verifiable. In this case, C-TLP encapsulates the verification keys and reveals each of them to verifiers only after a certain time. SO-PoR also uses a smart contract which acts as a public verifier on the client's behalf to verify proofs and pay an honest server. The pre-computation technique allows the client at setup to generate a constant number of \emph{disposable} homomorphic MAC's for each verification.  The combination of disposable homomorphic MAC's and C-TLP  makes it possible to (a) use a smart contract  and (b) take advantage of MAC's efficiency in the setting where public verifiability is needed. This combination has applications beyond PoR.  A blockchain-based random extraction beacon allows the server to independently  derive a set of unpredictable random values from the blockchain such that the values' correctness is publicly verifiable. 



At a high-level  SO-PoR works as follows. The client encodes its file using an error-correcting code and  for each $j\text{\small{-th}}$ verification it does the following. It picks two   random keys: $(v_{\scriptscriptstyle j},l_{\scriptscriptstyle j})$ of a $\mathtt{PRF}$. It uses $v_{\scriptscriptstyle j}$ to generate $c$ random blocks' indices, i.e. challenged blocks. It uses  $l_{\scriptscriptstyle j}$ to generate a disposable MAC on each challenged block. It also uses C-TLP to make two puzzles, one that encapsulates $v_{\scriptscriptstyle j}$, and  another that encapsulates $l_{\scriptscriptstyle j}$. It deposits enough coins to cover $z$ successful PoR verifications in a smart contract. The client sends the encoded file, tags and the puzzles to the server. When $j\text{\small{-th}}$ PoR proof is needed, the server manages to discover key $v_{\scriptscriptstyle j}$ that lets it determine which file blocks are challenges. The server also uses the beacon to extract a set of random values from the blockchain. Using the MAC's,  challenged blocks, and  beacon's outputs, the server generates a compact PoR proof. The server sends the proof to the contract. After that, it can delete the related disposable MAC's.  For the same verification, after a fixed time, it manages to find the related MAC's verification key: $l_{\scriptscriptstyle j}$. It sends the key to the server who checks the correctness of $l_{\scriptscriptstyle j}$ and  PoR proof. If the contract accepts all proofs, then it pays the server  for $j\text{\small{-th}}$ verification; otherwise, it notifies the client.  




\vspace{-7mm}

\subsection {SO-PoR Model Overview}\label{SO-PoR-Model}

\vspace{-2mm}
SO-PoR model is built upon the traditional PoR paradigm \cite{DBLP:conf/asiacrypt/ShachamW08} which   is a challenge-response  protocol where a server proves to  an honest client that its file is retrievable (see  Appendix \ref{PoR-Model} for a formal definition of the PoR). In  SO-PoR, however,  a client may not be available for   verification. So, it wants to  delegate  a set of verifications that it cannot carry out. Informally, in this setting, it (in addition to file retrievability)  must have three guarantees: (a) \emph{verification correctness}: every verification is performed honestly, so  the client can trust the verification's result  without redoing it, (b) \emph{real-time detection}: the client is notified in almost real-time when a  proof is rejected, and (c) \emph{fair payment}: in every verification, the server is paid only if a  proof  is accepted. In SO-PoR, three parties are involved: an honest client, potentially malicious server  and a standard smart contract. SO-PoR also, analogous to  \cite{DBLP:conf/asiacrypt/ShachamW08},  allows a client to perform the verification itself,  when it is available. We present our formal definition of SO-PoR in  Appendix \ref{SO-PoR-Model}.  

\vspace{-3mm}
%SO-PoR uses a novel combination of disposable tags, pre-computation technique,  C-TLP scheme, smart contract,  pseudorandom functions and commitment scheme.  At a high-level, SO-PoR works as follows. The client generates a set of authenticator tags on the file. These tags will allow the client to verify its data availability  when it is online. Also, for every verification that the client cannot be online, it precomputes a (small) set of \emph{disposable} tags related to the file's blocks that will be challenged for that verification. However, unlike standard PoR schemes in which   a subset of file blocks are challenged by picking their indices  randomly on the fly just before the verification, in SO-PoR, the challenged blocks' indices are picked \emph{pseudorandomly} by the client in the setup phase. Then, the client for, each verification, encodes the secret key that allows regeneration of the pseudorandom indices and a secret key used for a PoR verification into two puzzles. The client stores the file, tags, and puzzles on the server. It stores  commitments of the secret values that will be used for PoR verification  in a smart contract. Also, it deposits enough coins in the smart contract, to pay the server if each proof (given by the server) is accepted. At each verification time, the server first solves a puzzle and fully recovers the key for random indices. Using the key,  corresponding tags and the file, it generates a PoR and sends it to the contract. After a certain  period, for the same verification, it manages to fully find another puzzle solution  which is the verification key. It sends the key to the contract  who first checks the correctness of the key and then verifies the PoR. If accepted, the contract  for that verification pays the cloud server  who can now delete all metadata (e.g. tags, encrypted, and decrypted values) for that verification.  



%First, client breaks up its file into blocks and apply an error-correcting code on every blocks. Then, it generates a set of MAC-based tags on every  blocks. These tags will allow the client to verify its data availability when it is online. For the sake of simplicity, let us assume the client does not want to perform  $z$ consecutive verifications. For  every $j^{\scriptscriptstyle th}$ verification  ($1\leq j\leq z$)  the client determines the random indices of the blocks that will be challenged for this verification and also precomputes (small) set of MAC-based tags for those blocks. It uses time-lock encryption scheme to encrypts the random indices and secret verification values (for the tags) and stores the encrypted values on the server. It also  stores the hash of random indices and  secret verification values in a smart contract. The client also stores the indices of the blocks that will appear in the blockchain from which a set of random value will be extracted. 

%% !TEX root =main.tex




\section {SO-PoR Model}\label{SO-PoR-Model}
In this section, we provide a formal definition  of SO-PoR. As previously stated, it builds upon the traditional PoR model \cite{DBLP:conf/asiacrypt/ShachamW08}, presented in Appendix \ref{PoR-Model}.  In  SO-PoR, unlike the traditional PoR, a client may not be available every time  verification is needed. Therefore, it wants to  delegate  a set of verifications that it cannot carry out itself. In this setting, it (in addition to file retrievability)  must have three guarantees: (a) \emph{verification correctness}: every verification is performed honestly, so  the client can rely on the verification result  without the need to re-do it, (b) \emph{real-time detection}: the client is notified in almost real-time when server's  proof is rejected, and (c) \emph{fair payment}: in every verification, the server is paid only if the server's  proof  is accepted. In SO-PoR, three parties are involved: an honest client, potentially malicious server  and a standard smart contract. SO-PoR also allows a client to perform the verification itself, analogous to the traditional  PoR, when it is available. 

%To satisfy the aforementioned requirements, and keep verifications' cost low, SO-PoR mainly utilises a smart contract (for verification and payment) and the chained time-lock puzzle to eventually release secret values used to: (a) generate challenges and (b) verify proofs. Therefore, i


\begin{definition}
A Smart Outsourced PoR (SO-PoR) scheme consists of seven algorithms ($\mathtt{Setup}, \mathtt{Store},$ $ \mathtt {SolvPuz}, $ $ \mathtt{GenChall}, \mathtt{Prove},$ $ \mathtt{Verify},  \mathtt{Pay}$) defined below: 


\
\begin{itemize}
\item[$\bullet$] $\mathtt{Setup}(1^{\scriptscriptstyle\lambda},\Delta, z)\rightarrow (\hat{sk},\hat{pk})$:  a probabilistic algorithm, run by a client.  It  takes as input a security: $1^{\scriptscriptstyle\lambda}$, time parameter: $\Delta$, and the number of verification delegated: $z$. It  outputs a set of  secret and public keys.

\

\item[$\bullet$] $\mathtt{Store}(\hat{sk},\hat{pk}, F,z)\rightarrow ({\bm{F}}, \sigma, \vv{\bm{o}},aux)$: a probabilistic algorithm, run only once by a client. It  takes as  input the secret key: $\hat{sk}$, public key: $\hat{pk}$, a file: $F$, and the number of verifications: $z$ that the client wants to delegate. It outputs an encoded file: ${\bm{F}}$,  a set of tags: $\sigma$, a set of $z$ puzzles: $\vv{\bm{o}}$, and public auxiliary data: $aux$. First three outputs are stored on the server and last output: $aux$, is   stored on a smart contract. 

\

\item[$\bullet$] $\mathtt {SolvPuz}(\hat{pk},\vv{\bm{o}})\rightarrow \vv{\bm{s}}$:  a deterministic algorithm that takes as input the public key: $\hat{pk}$ and puzzle vector: $\vv{\bm{o}}$.  It for each  $j\text{\small{-th}}$ verification outputs a  pair: $\ddot{s}_{\scriptscriptstyle j}:(v_{\scriptscriptstyle j},l_{\scriptscriptstyle j})$ of solutions, where $v_{\scriptscriptstyle j}$ and $l_{\scriptscriptstyle j}$ are outputted at time $t_{\scriptscriptstyle j}$ and $t'_{\scriptscriptstyle j}$ respectively and $t'_{\scriptscriptstyle j}> t_{\scriptscriptstyle j}$. Therefore, the algorithm in total outputs $z$ pairs. Value $l_{\scriptscriptstyle j}$ is sent  to the smart contract right after it is discovered. This algorithm is run  by the server.

\


\item[$\bullet$] $\mathtt{GenChall}(j,|{\bm{F}}|, 1^{\scriptscriptstyle\lambda},\ddot{s}_{\scriptscriptstyle j},aux)\rightarrow \vv{\bm{c}}$: a probabilistic algorithm that takes as input a verification index: $j$, the encoded file size: $|{\bm{F}}|$, security parameter: $1^{\scriptscriptstyle\lambda}$, first component of the related solution pair, $v_{\scriptscriptstyle j}\in \ddot{s}_{\scriptscriptstyle j}$, and public parameters: $pp\in aux$ containing  a blockchain and its parameters. It outputs pairs $\ddot{c}_{\scriptscriptstyle j} : (x_{\scriptscriptstyle j} , y_{\scriptscriptstyle j} )$, where each pair includes a pseudorandom  block's index:  $x_{\scriptscriptstyle j}$ and random coefficient: $y_{\scriptscriptstyle j}$. Also, values $x_{\scriptscriptstyle j}$ are derived from $v_{\scriptscriptstyle j}$ while $y_{\scriptscriptstyle j}$ are derived from $pp$. This algorithm is run by the server for each verification. 


%$pp$ is a public parameters for the beacon and it includes, blockchain, chain quality, and index. $\mathtt{GenCoeffs}()$ is called here

\

\item[$\bullet$] $\mathtt{Prove}(j,{\bm{F}}, \sigma,  \vv{\bm{c}})\rightarrow \pi$: a probabilistic algorithm that takes the verification index $j$, encoded file: ${\bm{F}}$ , (a subset of) tags: $\sigma$, and a vector of unpredictable challenges: $\vv{\bm{c}}$, as inputs and outputs a proof of  file retrievability. It is run by the server for each verification.

\

\item[$\bullet$] $\mathtt{Verify}(j,\pi,\ddot{s}_{\scriptscriptstyle j},aux)\rightarrow d:\{0,1\}$: a deterministic algorithm that takes the verification index $j$, proof: $\pi$,  second component of the related solution pair: $l_{\scriptscriptstyle j}\in \ddot{s}_{\scriptscriptstyle j}$, and public auxiliary data: $aux$.  If the proof is accepted, it outputs $d=1$; otherwise, outputs $d=0$. The default value of $d$ is $0$. This algorithm is run by the smart contract for each verification and invoked only once for each verification by only the server. 

\

\item[$\bullet$] $\mathtt{Pay}(j,d)\rightarrow d'=\{0,1\}$: a deterministic algorithm that takes the verification index $j$, the verification output: $d$. If $d=1$, it transfers $e$ amounts to the server and outputs $1$. Otherwise, it does not transfer anything, and outputs $0$. The default value of $d'$ is $0$. The algorithm is run by the  contract, and  invoked only by $\mathtt{Verify}(.)$. 
\end{itemize}
\end{definition}





%
%note that in the above, $\mathtt{Store}$ is a wrapper function that calls $\mathtt{GenPuz}(\vv{\bm{m}},\hat{sk},\hat{pk})$ and $\mathtt{Store}(\hat{sk},F)$ as subroutine,{\color{blue}xx explain what $\vv{\bm{m}}$ is for}
%




  
A SO-PoR scheme must satisfy two main properties: \emph{correctness} and \emph{soundness}. The correctness requires, for any: file, public-private key pairs, and puzzle solutions, both the verification  and pay algorithms, i.e. $\mathtt{Verify}(.)$ and $\mathtt{Pay}(.)$, output $1$ when interacting with  the  prover, verifier, and client  all of which are honest.  The soundness however is split into four properties: extractability, verification correctness, real-time detection, and fair payment, formally defined below.  Before we define the first property,  extractability, we provide the following  experiment between an environment: $\mathcal{E}$ and  adversary: $\mathcal{A}$ who corrupts $C\subsetneq\{\mathcal{S},\mathcal{M}_{\scriptscriptstyle 1},...,\mathcal{M}_{\scriptscriptstyle\beta}\}$, where $\beta$ is the maximum number of miners which can be corrupted in a secure blockchain. In this game, $\mathcal{A}$ plays the role of corrupt parties and $\mathcal{E}$ simulating an honest party's role. 


\begin{enumerate}
\item $\mathcal{E}$ executes $\mathtt{Setup}(.)$ algorithm and provides public key: $\hat{pk}$, to $\mathcal{A}$.   
\item $\mathcal{A}$ can pick  arbitrary file $F'$, and  uses it to make queries to  $\mathcal{E}$ to run:  $\mathtt{Store}(\hat{sk},\hat{pk},$ $ F',z)$ $\rightarrow (F'^{\scriptscriptstyle *}, \sigma, \vv{\bm{o}},aux)$  and return the output to $\mathcal{A}$. Also, upon receiving the output of $\mathtt{Store}()$, $\mathcal{A}$ can locally run  algorithms: $\mathtt {SolvPuz}(\hat{pk},\vv{\bm{o}})$ and   $\mathtt{GenChall}(j,$ $|F'^{\scriptscriptstyle *}|, $ $ 1^{\scriptscriptstyle\lambda},\ddot{s}_{\scriptscriptstyle j},aux)\rightarrow \vv{\bm{c}}$ as well as  $\mathtt{Prove}(j,F^{\scriptscriptstyle *}, \sigma, $ $ \vv{\bm{c}})\rightarrow \pi$,  to get their outputs as well. 
\item $\mathcal{A}$ can request $\mathcal{E}$ the execution of $\mathtt{Verify}(j,\pi,\ddot{s}_{\scriptscriptstyle j},aux)$ for any $F'$ used to query $\mathtt{Store}()$. Accordingly, $\mathcal{E}$ informs  $\mathcal{A}$ about the verification output. The adversary can send a polynomial number of queries to $\mathcal{E}$. Finally, $\mathcal{A}$ outputs the description of a prover: $\mathcal{A}'$ for any file it has already chosen above. 
\end{enumerate}

It is said a cheating prover: $\mathcal{A}'$ is $\epsilon$-admissible if it convincingly answers $\epsilon$ fraction of verification challenges \cite{DBLP:conf/asiacrypt/ShachamW08}. Informally, a SO-PoR scheme supports extractability, if there is an extractor algorithm: $\mathtt{Ext}(\hat{sk},\hat{pk},\mathtt{P}')$, that takes the secret-public keys and the description of the  machine implementing the prover's role: $\mathcal{A}'$ and outputs the file: $F'$. The extractor can reset the adversary to the beginning of the challenge phase and repeat this step polynomially many times for  of extraction, i.e. the extractor can rewind it.

\begin{definition}[$\epsilon$-extractable]\label{extractable} A SO-PoR scheme is $\epsilon$-extractable if  for every adversary: $\mathcal{A}$ who corrupts $C\subsetneq\{\mathcal{S},\mathcal{M}_{\scriptscriptstyle 1} $ $,..., \mathcal{M}_{\scriptscriptstyle\beta}\}$, plays the experiment above, and outputs an $\epsilon$-admissible cheating prover: $\mathcal{A}'$ for a file $F'$,  there exists an extraction algorithm that recovers $F'$ from $\mathcal{A}'$, given honest parties public-private keys and $\mathcal{A}'$,  i.e. $\mathtt{Ext}(\hat{sk},\hat{pk},\mathcal{A}')\rightarrow F'$, except with a negligible probability. 
\end{definition}

% . The extractor has the ability to reset the adversary to the beginning of the challenge phase and repeat this step polynomially many times for the purpose of extraction

In the above game, the environment, acting on honest parties' behalf, performs the verification correctly; which is not always the case in SO-PoR. As the verification can be run by miners a subset of which are potentially corrupted. Even in this case, the verification correctness must hold, e.g.  if a corrupt server sends an  invalid proof then even if $\beta-1$ miners are corrupt (and colluding with it) the verification function will not output $1$ and if the server is honest and submits a valid proof then the verification function does not output $0$ even if $\beta$ miners are corrupt, except with a negligible probability. This is formalised below. 


\begin{definition}[Verification Correctness]\label{Verification-Correctness} Let $\beta$ be the maximum number of miners that can be corrupted in a secure blockchain network and $\lambda'$ be the blockchain security parameter. Also, let $\mathcal{A}$ be the adversary who (plays the above game and) corrupts parties in either $C\subseteq\{\mathcal{S},\mathcal{M}_{\scriptscriptstyle 1},...,\mathcal{M}_{\scriptscriptstyle\beta-1}\}$ or $C'\subseteq\{\mathcal{M}_{\scriptscriptstyle 1},...,\mathcal{M}_{\scriptscriptstyle\beta}\}$.  In SO-PoR, we say the correctness of $j\text{\small-th}$ verification  is guaranteed if: 
 
$$\begin{array}{l}
\text{in the former case}: Pr[\mathtt{Verify}_{\scriptscriptstyle C}(j,\pi,\ddot{s}_{\scriptscriptstyle j},aux)=1]\leq \mu(\lambda')\\
\text{in the latter case}: Pr[\mathtt{Verify}_{\scriptscriptstyle C'}(j,\pi,\ddot{s}_{\scriptscriptstyle j},aux)=0]\leq \mu(\lambda')
\end{array}$$
where $\mu(.)$ is a negligible function. 
\end{definition}

Also, a client needs to have a guarantee that for each verification it can get a correct result within a (fixed) time period. 

\begin{definition}[$\Upsilon$-real-time Detection]\label{real-time Detection} Let $\mathcal{A}$, as defined above, be the adversary who corrupts either $C$ or $C'$.
A client, for each $j\text{\small{-th}}$ delegated verification, will get a correct output of  $\mathtt{Verify(.)}$, by  means of reading a blockchain, within time window $\Upsilon$, after the time when the server is supposed to send  its proof  to the blockchain network. Formally,

$$\mathtt{Read}(\Upsilon,\mathtt{Verify}_{\scriptscriptstyle D}(j,\pi,\ddot{s}_{\scriptscriptstyle j},aux))\rightarrow \{0,1\}$$
where $D\subsetneq\{C,C'\}$, except with a negligible probability. 
\end{definition}





\begin{definition}[Fair Payment]\label{Fair-Payment}  SO-PoR supports a fair payment if the client and server fairness are satisfied: 

\begin{itemize}
\item[$\bullet$] \textit{\textbf{Client Fairness}}: An honest client is guaranteed that it only pays ($e$ coins) if the server provides an accepting proof, except with a negligible probability. 
\item[$\bullet$]\textit{\textbf{Server Fairness}}: An honest server is guaranteed that the client gets a correct proof if the client pays ($e$ coins),   except with a negligible probability. 
\end{itemize}
Formally, let $\mathcal{A}$ be the adversary who corrupts either $C$ or $C'$, as defined above. To satisfy a fair payment:
\begin{equation}
Pr[\mathtt{Pay}_{\scriptscriptstyle D}(.)=b 	\cap  \mathtt{Verify}_{\scriptscriptstyle D}(.)=b]\geq 1-\mu(\lambda'),   
\end{equation}

the following inequality must hold:
\begin{equation}\label{inequ::fair-payment}
Pr[\mathtt{Pay}_{\scriptscriptstyle D}(.)=b' 	\cap \mathtt{Verify}_{\scriptscriptstyle D}(.)=b] \leq \mu(\lambda'),
\end{equation}
where $D\subsetneq\{C,C'\},b\neq b'$, and $b, b'\subsetneq\{0,1\}$









%\begin{equation}
%Pr[\mathtt{Pay}_{\scriptscriptstyle D}(.)=1 	\cap \mathtt{Verify}_{\scriptscriptstyle D}(.)=0] \leq \mu(\lambda')
%\end{equation}
%\begin{equation}
%Pr[ \mathtt{Pay}_{\scriptscriptstyle D}(.)=0 	\cap \mathtt{Verify}_{\scriptscriptstyle D}(.)=1] \leq \mu(\lambda')
%\end{equation}
%\begin{equation}
%Pr[\mathtt{Pay}_{\scriptscriptstyle D}(.)=1 	\cap  \mathtt{Verify}_{\scriptscriptstyle D}(.)=1]\geq 1-\mu(\lambda')
%\end{equation}
%where $D\subsetneq\{C,C'\}$
\end{definition}

The above definition also takes into account the fact that the client at the time of delegated verification is not necessarily available to make the payment itself, so the payment is delegated to a third party, e.g. a smart contract. In this case, the definition  ensures that even if  the client or/and server are honest, the third party cannot affect  the fairness (except with a negligible probability).

% Moreover, it is not hard to see, if the inequality \ref{} holds, then the fairness is guaranteed, with a high probability:  \begin{equation*}
%Pr[\mathtt{Pay}_{\scriptscriptstyle D}(.)=b 	\cap  \mathtt{Verify}_{\scriptscriptstyle D}(.)=b]\geq 1-\mu(\lambda')
%\end{equation*} 


%In the following we explain the rational behind the above definition. In SO-PoR scheme, for each verification, the server sells  a proof: $\pi$ to a client and earns  $e$ coins if and only if the proof is accepted, i.e. $\mathtt{Verify}(.)=1$. In SO-PoR setting, the client at the time of delegated verification is not necessarily online to make the payment itself, so it is done by a third party (e.g. a smart contract). The definition must ensure that   even if both server and client are honest, the third party cannot affect  the fairness. 


\begin{definition}[SO-PoR Security]\label{SO-PoR-Security} A SO-PoR scheme is secure if it is $\epsilon$-extractable, and satisfies verification correctness, $\Upsilon$-real-time detection, and fair payment properties.

\end{definition}




\begin{remark}
The folklore assumption is that (in a secure blockchain) a smart contract function \emph{always outputs a correct result}. However, this is not the case and it may fail under certain circumstances.  For instance, as shown in \cite{LuuTKS15} all rational  miners may not verify a certain transaction. As another example,  an adversary (although with a negligibly small probability)  discards a  certain honestly generated blocks,  reverses the state of blockchain and contract, or breaks a client's signature scheme.  Accordingly, in our definitions above, we take such cases  into consideration and allow the possibility that a function outputs an incorrect result even though with a negligibly small probability. 
\end{remark}




\begin{remark}
SO-PoR model differs from traditional (e.g. \cite{DBLP:conf/ccs/JuelsK07,DBLP:conf/asiacrypt/ShachamW08}) and outsourced PoR  (e.g. \cite{armknecht2014outsourced,xu2016lightweight}) models in several aspects. Only  the SO-PoR model offers all the properties. In particular,  traditional PoRs only offer extractability while outsourced ones  offer liability as well, that allows a client (by re-running all verifications function) to detect a verifier if it provides an incorrect verification output, so the client   cannot rely on the verification result provided.  As another difference, the SO-PoR model  takes into account the case where an adversary can corrupt both the server and some miners at the same time.
\end{remark}

\begin{remark}
SO-PoR should also support  the traditional PoR where only client and server interact with each other  (e.g. client generates challenges, and verifies proof) when the client is available. To let SO-PoR definition support that too, we can simply define a flag: $\xi$, in each function, such that  when $\xi=1$, it acts as the traditional PoR; otherwise (when $\xi=0$), it performs as a delegated one. For the sake of simplicity, we let the flag  be implicit in the definitions above, where the default  is  $\xi=1$ 
\end{remark}


\subsection{SO-PoR Protocol}\label{SO-PoR-Protocol}
This section presents SO-PoR protocol in detail, followed by the rationale behind it.      





\begin{enumerate}[leftmargin=.46cm]

\item\textit{\textbf{Client-side Setup}}. 



\begin{enumerate}
%\item Signs and deploys a smart contract: $\mathcal{SC}$ to a blockchain.

% where the contract contains a set of public parameters: e.g. $z$: total number of verifications,  $|F|$: file bit size, $\Delta_{\scriptscriptstyle 1}$:  maximum time period  taken by the server to generate a proof, $\Delta_{\scriptscriptstyle 2}$: time window in which a message is (sent by the server and) received by the contract. The client deposits $e\cdot z$  coins for $z$ successful  verifications. 

\item  \textbf{\textit{\small {Gen. Public and Private Keys}}}:   Picks a fresh key: $\hat{k}$ and two vectors of keys: $\vv{\bm{v}}$ and $\vv{\bm{l}}$, where each vector contains $z$ fresh keys. It picks a large prime number:  $p$ whose size is determined by a security parameter, (i.e. its bit-length is equal to the bit-length of $\mathtt{PRF}(.)$'s output, $|p|=\iota$).  Moreover, it runs $\mathtt{Setup}(.)$ in  C-TLP scheme to generate a key pair: $(pk, sk)$

\item \textbf{\textit{\small {Gen. Other Public Parameters}}}:  Sets $c$ to the total number of blocks challenged in each verification. It defines  parameters: $w$ and $g$, where  $w$ is an index  of a future block: $\mathcal {B}_{\scriptscriptstyle w}$ in a blockchain that will be added to the blockchain (permanent state) at about the time  first delegated verification will  be done, and $g$ is  a security parameter referring to the number of blocks (in a row) starting from  $w$.  It  sets $z$: the total number of verifications,  $||{\bm{F}}||$: file bit size, $\Delta_{\scriptscriptstyle 1}$:  the maximum time  is taken by the server to generate a proof, $\Delta_{\scriptscriptstyle 2}$: time window in which a message is (sent by the server and) received by the contract, and $e$ amount of coins paid to the server for each successful  verification. Sets $\hat{pk}: (pk,e,g,w,p,c,z,\Delta_{\scriptscriptstyle 1},\Delta_{\scriptscriptstyle 2})$ 


\item\textbf{\textit{\small {Sign and Deploy   Smart Contract}}}: Signs and deploys a smart contract: $\mathcal{SC}$ to a blockchain.  It stores  public parameters: $(z,||{\bm{F}}||, \Delta_{\scriptscriptstyle 1},\Delta_{\scriptscriptstyle 2},c, g,p,w)$, on the contract. It deposits $e z$ coins to the contract. Then, it asks the server to sign the contract. The server signs if it agrees on all parameters.

\end{enumerate}
\item\textit{\textbf{Client-side Store}}.


\begin{enumerate}

\item \textbf{\textit{\small {Encode File}}}: Splits an error-corrected file, e.g. under Reed-Solomon codes, into $n$ blocks; ${\bm{F}}: [F_{\scriptscriptstyle 1},...,F_{\scriptscriptstyle n}]$,  where $ F_{\scriptscriptstyle i}\in \mathbb{F}_p$
\item\label{gen-client-server-tags}\textbf{\textit{\small {Gen. Permanent Tags}}}:  Using the key: $\hat{k}$, it computes $n$ pseudorandom values:  $r_{\scriptscriptstyle i}$ and single value: $\alpha$, as follows.  
 $$\alpha=\mathtt{PRF}(\hat{k},n+1)\bmod p, \  \  \ \  \  \  \forall i, 1\leq i\leq n: r_{\scriptscriptstyle i}=\mathtt{PRF}(\hat{k},i)\bmod p$$
 It uses the pseudorandom values to compute tags for the file blocks. 
 $$\forall i, 1\leq i\leq n: \sigma_{\scriptscriptstyle i}= r_{\scriptscriptstyle i}+ \alpha\cdot F_{\scriptscriptstyle  i}\bmod p$$ 
 So, at the end of this step,  a set of  tags are generated, $\sigma:\{\sigma_{\scriptscriptstyle 1},..., \sigma_{\scriptscriptstyle n}\}$
\item\label{Gen-Disposable-Tags}\textbf{\textit{\small {Gen. Disposable Tags}}}: For   $j\text{\small{-th}}$ verification  ($1\leq j\leq z$):
\begin{enumerate}
\item chooses the related key: $v_{\scriptscriptstyle j}\in\vv{\bm{v}}$ and computes $c$ pseudorandom indices. 
$$\forall b, 1\leq b\leq c: x_{\scriptscriptstyle b,j}=\mathtt{PRF}(v_{\scriptscriptstyle j}, b)\bmod n$$

\item picks the corresponding  key: $l_{\scriptscriptstyle j}\in \vv{\bm{l}}$ and computes $c$ pseudorandom values:  $r_{\scriptscriptstyle b,j}$ and single value: $\alpha_{\scriptscriptstyle j}$, as follows. 
$$\alpha_{\scriptscriptstyle j}=\mathtt{PRF}(l_{\scriptscriptstyle j},c+1)\bmod p, \ \ \ \ \ \forall b, 1\leq b\leq c: r_{\scriptscriptstyle b,j}=\mathtt{PRF}(l_{\scriptscriptstyle j},b)\bmod p$$


\item generates $c$ disposable tags.  
$$\forall b, 1\leq b\leq c: \sigma_{\scriptscriptstyle b,j}=r_{\scriptscriptstyle b,j}+\alpha_{\scriptscriptstyle j}\cdot F_{\scriptscriptstyle y}\bmod p$$
where $y= x_{\scriptscriptstyle b,j}$. At the end of this step, a set $\sigma_{\scriptscriptstyle j}$ of $c$ tags are computed,  $\sigma_{\scriptscriptstyle j}:\{\sigma_{\scriptscriptstyle 1,j},..., \sigma_{\scriptscriptstyle c,j}\}$

%\item encrypts $k_{\scriptscriptstyle 1}^{\scriptscriptstyle j}$ using the time-lock encryption, such that it can be decrypted at time $t^{\scriptscriptstyle j}_{\scriptscriptstyle 1}$, i.e., $C^{\scriptscriptstyle j}_{\scriptscriptstyle 1}=\mathcal{ENC}^{\scriptscriptstyle pk}_{\scriptscriptstyle sk,T}(k_{\scriptscriptstyle 1}^{\scriptscriptstyle j})$, where $T=t^{\scriptscriptstyle j}_{\scriptscriptstyle 1}\cdot S$, and $S$ is a parameter of the encryption: the number of squaring modulo $N$ per second that can be performed by a solver. Also, it uses the time-lock encryption to encrypt $k^{\scriptscriptstyle j}_{\scriptscriptstyle 2}$, such that it can be decrypted at time $ t^{\scriptscriptstyle j}_{\scriptscriptstyle 2}$,  i.e., $C^{\scriptscriptstyle j}_{\scriptscriptstyle 2}=\mathcal{ENC}^{\scriptscriptstyle pk}_{\scriptscriptstyle sk,T'}(k_{\scriptscriptstyle 2}^{\scriptscriptstyle j})$, where $T'= t^{\scriptscriptstyle j}_{\scriptscriptstyle 2}\cdot S$ and $t^{\scriptscriptstyle j}_{\scriptscriptstyle 2}> t^{\scriptscriptstyle j}_{\scriptscriptstyle 1}+\Delta_{\scriptscriptstyle 1}+\Delta_{\scriptscriptstyle 2}$,  $\Delta_{\scriptscriptstyle 1}$  is the time period within which some local computation (by the server) is performed on $k^{\scriptscriptstyle j}_{\scriptscriptstyle 1}$ and $\Delta_{\scriptscriptstyle 2}$ is the time window in which a message, e.g. $k^{\scriptscriptstyle j}_{\scriptscriptstyle 1}$, is (sent by the server and) received by the contract. Note that, the size of time windows are sufficiently large.

%\item\label{gen-hashes} computes $h_{\scriptscriptstyle j}= H(l_{\scriptscriptstyle j})$.


%$\resizeT{\textit{w}}_{\resizeS {\textit  j}}$

%\resizeS {\textit  w}_{\resizeS {\textit  j}}}

\end{enumerate} 

\item\label{Gen-Puzzles-}\textbf{\textit{\small {Gen. Puzzles}}}: Sets $\vv{\bm{m}}=[v_{\scriptscriptstyle 1},l_{\scriptscriptstyle 1},...,v_{\scriptscriptstyle z},l_{\scriptscriptstyle z}]$  and then encrypts the vector's elements, by running: $\mathtt{GenPuz}(\vv{\bm{m}},pk,sk)$ in   C-TLP scheme. This yields a  puzzle vector: $[(V_{\scriptscriptstyle 1},L_{\scriptscriptstyle 1}),...,(V_{\scriptscriptstyle z},L_{\scriptscriptstyle z})]$ and a commitment vector: $\vv{\bm{h}}$. The encryption is done in  such a way that in each $j\text{\small{-th}}$ pair, $V_{\scriptscriptstyle j}$ will be fully decrypted at times $t_{\scriptscriptstyle j}$ and $L_{\scriptscriptstyle j}$ will be decrypted at time $t'_{\scriptscriptstyle j}$, where  $ t_{\scriptscriptstyle j}+\Delta_{\scriptscriptstyle 1}+\Delta_{\scriptscriptstyle 2}\leq t'_{\scriptscriptstyle j} < t_{\scriptscriptstyle j+1}$  %$\Delta_{\scriptscriptstyle 1}$  is the maximum  period  the server needs to generate a proof and $\Delta_{\scriptscriptstyle 2}$ is the time window in which a message is (sent by the server and) received by the contract.




%\item Using  C-TLP scheme, generates a key pair: ($sk_{\scriptscriptstyle 1}, pk_{\scriptscriptstyle 1}$) and  encrypts    $[v_{\scriptscriptstyle 1},...,v_{\scriptscriptstyle z}]$ such that they will be fully decrypted at times $[t_{\scriptscriptstyle 1},...,t_{\scriptscriptstyle z}]$ respectively. This yields a  ciphertext vector: $[V_{\scriptscriptstyle 1},...,V_{\scriptscriptstyle z}]$. Invoking  C-TLP scheme again, it generates key pairs: ($sk_{\scriptscriptstyle 2}, pk_{\scriptscriptstyle 2}$) and  encrypts    $[l_{\scriptscriptstyle 1},...,l_{\scriptscriptstyle z}]$ that will be fully decrypted at times $[t'_{\scriptscriptstyle 1},...t'_{\scriptscriptstyle z}]$ respectively.  This yields a  ciphertext vector: $[L_{\scriptscriptstyle 1},...,L_{\scriptscriptstyle z}]$. Note,   $t'_{\scriptscriptstyle j}\geq t_{\scriptscriptstyle j}+\Delta_{\scriptscriptstyle 1}+\Delta_{\scriptscriptstyle 2}$, where $\Delta_{\scriptscriptstyle 1}$  is the maximum time period  the server needs to generate a proof and $\Delta_{\scriptscriptstyle 2}$ is the time window in which a message is (sent by the server and) received by the contract.

%\item And another one  with input messages  $l_{\scriptscriptstyle 1},...,l_{\scriptscriptstyle z}$ that will be decrypted at times $t'_{\scriptscriptstyle 1},...t'_{\scriptscriptstyle z}$ respectively, where  $t'_{\scriptscriptstyle j}=t_{\scriptscriptstyle j}+\Delta_{\scriptscriptstyle 1}+\Delta_{\scriptscriptstyle 2}$,  $\Delta_{\scriptscriptstyle 1}$  is the time period within which some local computation (by the server) is performed on $v_{\scriptscriptstyle j}$ and $\Delta_{\scriptscriptstyle 2}$ is the time window in which a message, e.g. $v_{\scriptscriptstyle j}$, is (sent by the server and) received by the contract. %Note that, the size of time windows are sufficiently large.



\item\label{Outsource-File}\textbf{\textit{\small {Outsource File}}}: Stores ${\bm{F}},n,\hat{pk}, \{\sigma,\sigma_{\scriptscriptstyle 1},..., \sigma_{\scriptscriptstyle z}, (V_{\scriptscriptstyle 1},L_{\scriptscriptstyle 1}),...,(V_{\scriptscriptstyle z},L_{\scriptscriptstyle z})\}$   on the server. Also, it stores $\vv{\bm{h}}$ on the smart contract. 
\end{enumerate}


\item\textit{\textbf{Cloud-Side Proof Generation}}. For   $j\text{\small{-th}}$ verification  ($1\leq j\leq z$), the cloud:


\begin{enumerate} 
\item\label{Solve-Puzzle-Regen-Indices}\textbf{\textit{\small {Solve Puzzle and Regen.  Indices}}}:   Receives and parses the output of $\mathtt{SolvPuz}(.)$ in C-TLP, to extract $v_{\scriptscriptstyle j}$, at time $t_{\scriptscriptstyle j}$. Using $v_{\scriptscriptstyle j}$, it regenerates $c$ pseudorandom indices. 
$$\forall b, 1\leq b\leq c: x_{\scriptscriptstyle b,j}=\mathtt{PRF}(v_{\scriptscriptstyle j}, b)\bmod n$$ %where $v_{\scriptscriptstyle j}$ is the key fully decrypted by the cloud at time $t_{\scriptscriptstyle j}$ for this verification.


\item \textbf{\textit{\small {Extract Key}}}: Extracts a seed: $u_{\scriptscriptstyle j}$, from the blockchain as follows: $u_{\scriptscriptstyle j}=\mathtt{H}( \mathcal {B}_{\scriptscriptstyle \gamma}||...||  \mathcal {B}_{\scriptscriptstyle \zeta})$, where $\gamma=w+(j-1)\cdot g$ and $\zeta=w+j\cdot g$

\item\label{Gen-PoR}\textbf{\textit{\small {Gen. PoR}}}: Generates a PoR proof.
 $$\mu_{\scriptscriptstyle j}=\sum\limits^{\scriptscriptstyle c}_{\scriptscriptstyle b=1}  \mathtt{PRF}(u_{\scriptscriptstyle j},b)\cdot F_{\scriptscriptstyle y}\bmod p, \  \  \ \xi_{\scriptscriptstyle j}= \sum\limits^{\scriptscriptstyle c}_{\scriptscriptstyle b=1}  \mathtt{PRF}(u_{\scriptscriptstyle j},b)\cdot \sigma_{\scriptscriptstyle b,j}\bmod p$$
 where  $y$ is a pseudorandom index: $y= x_{\scriptscriptstyle b,j}$ %Also, it runs $\mathtt{Prove}(.)$ in C-TLP, to generate a proof: $\ddot{p}_{\scriptscriptstyle j}$, of $v_{\scriptscriptstyle j}$'s correctness
 
 \item\label{Register-Proofs}\textbf{\textit{\small {Register Proofs}}}:  Sends the PoR proof: $(\mu_{\scriptscriptstyle j},\xi_{\scriptscriptstyle j})$   to the smart contract within  $\Delta_{\scriptscriptstyle1}$
 \item\label{fully-recover-l}\textbf{\textit{\small {Solve Puzzle and Regen.  Verification Key}}}: Receives and parses the output of algorithm $\mathtt{SolvPuz}(.)$ in C-TLP to extract $l_{\scriptscriptstyle j}$, at time $t'_{\scriptscriptstyle j}$. Also, it runs $\mathtt{Prove}(.)$ in C-TLP, to generate a proof: $\ddot{p}_{\scriptscriptstyle j}$, of $l_{\scriptscriptstyle j}$'s correctness. It sends $\ddot{p}_{\scriptscriptstyle j}$ (containing $l_{\scriptscriptstyle j}$)  to the contract, so it can be received by the contract within $\Delta_{\scriptscriptstyle 2}$ 
\end{enumerate}


\item \textit{\textbf{Smart Contract-Side Verification}}. For   $j\text{\small{-th}}$ verification  ($1\leq j\leq z$), the contract:



\begin{enumerate} 
\item\textbf{\textit{\small {Check Arrival Time}}}: checks the arrival time of the decrypted values sent by the server. In particular, it checks, if $(\mu_{\scriptscriptstyle j},\xi_{\scriptscriptstyle j})$ was received in the time window: $(t_{\scriptscriptstyle j}, t_{\scriptscriptstyle j}+\Delta_{\scriptscriptstyle 1}+\Delta_{\scriptscriptstyle 2}]$ and whether $l_{\scriptscriptstyle j}$ was received in the time window: $(t'_{\scriptscriptstyle j}, t'_{\scriptscriptstyle j}+\Delta_{\scriptscriptstyle 2}]$

\item\label{check-hash}\textbf{\textit{\small {Verify Puzzle Solution}}}: runs $\mathtt{Verify}(.)$ in C-TLP to verify $\ddot{p}_{\scriptscriptstyle j}$  (i.e. checks the correctness of $l_{\scriptscriptstyle j}\in \ddot{p}_{\scriptscriptstyle j}$). If approved, then it regenerates the seed:  $u_{\scriptscriptstyle j}=\mathtt{H}( \mathcal {B}_{\scriptscriptstyle \gamma}||...||  \mathcal {B}_{\scriptscriptstyle \zeta})$, where $\gamma=w+(j-1)\cdot g$ and $\zeta=w+j\cdot g$


%$s_{\scriptscriptstyle j}=H( \mathcal {B}_{\resizeS {\textit  w}_{\resizeSS {\textit  j}}}||,...,||  \mathcal {B}_{\resizeS {\textit  w}_{\resizeSS {\textit  j}}+g})$, given parameters $g$ and $\resizeT{\textit{w}}_{\resizeS {\textit  j}}$. 


\item\label{verify-PoR}\textbf{\textit{\small {Verify PoR}}}: regenerates the pseudorandom values and verifies the PoR proof.  
\begin{equation}\label{POR-ver}\xi_{\scriptscriptstyle j}\stackrel{\scriptscriptstyle ?}=\mu_{\scriptscriptstyle j}  \cdot\mathtt{PRF}(l_{\scriptscriptstyle j},c+1)+\sum\limits^{\scriptscriptstyle c}_{\scriptscriptstyle b=1} ( \mathtt{PRF}(u_{\scriptscriptstyle j},b)\cdot \mathtt{PRF}(l_{\scriptscriptstyle j},b))\bmod p
\end{equation}
\item\textbf{\textit{\small {Pay}}}: if  Equation \ref{POR-ver} holds, pays and asks the server to delete all disposable tags for this verification, i.e. $\sigma_{\scriptscriptstyle j}$
\end{enumerate}
If either check fails, it aborts and notifies the client. 

\vspace{-1mm}
\item \textit{\textbf{Client-server PoR}}: When the client is online, it can   interact  with the server  to check its data availability too. In particular, it sends $c$ random challenges and random indices to the server who computes POR using only: (a) the  messages sent by the client in this step, (b) the  file: ${\bm{F}}$, and (c) the tags:  $\sigma_{\scriptscriptstyle i}\in\sigma$, generated in step \ref{gen-client-server-tags}.  The proof generation and verification are similar to the MAC-based schemes, e.g.  \cite{DBLP:conf/asiacrypt/ShachamW08}. 
\end{enumerate}


\begin{theorem}\label{PoR-main-theorem} SO-PoR protocol is secure  if the MAC's are unforgeable, $\mathtt{PRF}(.)$ is a secure pseudorandom function, the blockchain and C-TLP protocol are secure, and $\mathtt{H}( \mathcal {B}_{\scriptscriptstyle \gamma}||...||  \mathcal {B}_{\scriptscriptstyle \zeta})$ outputs an unpredictable random value (where $\zeta-\gamma$ is a security parameter).
\end{theorem}


\begin{proof}[Outline]
Let $\mathcal{M}_{\scriptscriptstyle i}$ be a blockchain miners and $\beta$ be the maximum number of miners which can be corrupted in a secure blockchain. We first argue that the adversary who corrupts either $C\subseteq\{\mathcal{M}_{\scriptscriptstyle 1},...,\mathcal{M}_{\scriptscriptstyle\beta}\}$ or $C'\subseteq\{\mathcal{S},\mathcal{M}_{\scriptscriptstyle 1},...,\mathcal{M}_{\scriptscriptstyle\beta-1}\}$ with a high probability, cannot influence the output of $\mathtt{Verify}(.)$ performed by a smart contract in a blockchain (i.e. the verification correctness holds) due to the security of the underlying blockchain. Then, we argue that if a proof produced by an adversary who corrupts $C'\subseteq\{\mathcal{S},\mathcal{M}_{\scriptscriptstyle 1},...,\mathcal{M}_{\scriptscriptstyle\beta-1}\}$ is accepted by $\mathtt{Verify(.)}$ with probability at least $\epsilon$, then the file can be extracted by a means of an extraction algorithm, due to the security of the underlying MAC's, $\mathtt{PRF}(.)$ and C-TLP as well as the unpredictability of  random extractor beacon's output, i.e. $\mathtt{H}( \mathcal {B}_{\scriptscriptstyle \gamma}||...||  \mathcal {B}_{\scriptscriptstyle \zeta})$. Next,  we argue that  after the server broadcasts a proof at a certain time to the network, the client can get a correct output of $\mathtt{Verify}(.)$ at most after time period $\Upsilon$, by a means of  reading the blockchain, due to the correctness of $\mathtt{Verify}(.)$ and the maximum delay on the client’s view of the output  (i.e. $\Upsilon$-real-time detection). Also, we argue that fair payment is held due to the security of blockchain and correctness of $\mathtt{Pay}(.)$. We refer readers to Appendix \ref{SO-PoR-Security-Proof} for detailed proof.
\hfill\(\Box\)
\end{proof}



%\begin{remark}
%In SO-PoR, for a security reason the server must record $j\text{\small{-th}}$ PoR proof in the contract before $l_{\scriptscriptstyle j}$ is recovered. Also,  the way disposable tags are generated in SO-PoR  differs from those computed  in previous PoR schemes, despite having similarities structure-wise. Moreover, with slight adjustments, we can reduce the contract-side storage cost to constant.  For more details, we refer readers to Appendix \ref{SO-PoR-discussion} which also explains why strawman approaches are not suitable substitutes for SO-PoR. 
%\end{remark}
\vspace{-3mm}
%\subsection{Extension: Reducing Smart Contract Storage Cost to Constant}\label{storage-cost-reduction}
%
%
%With minor adjustments, we can reduce the smart contract storage cost from $O(z)$ to constant, $O(1)$ and offload the cost to the server. The idea is that the client after computing the commitmnet vector: $\vv{\bm{h}}=[h_{\scriptscriptstyle 1},...h_{\scriptscriptstyle z}]$,  in step \ref{Gen-Puzzles-}, it preserves the ordering of the elements (i.e. $h_{\scriptscriptstyle j}$ is associated with $j^{\scriptscriptstyle th} $ verification) and constructs a  Merkle tree  on top of them. It stores the tree and the vector on the server, and stores only the tree's  root: $R$, on the contract. In this case,  the server in step \ref{fully-recover-l} after recovering $\ddot{p}_{\scriptscriptstyle j}= (l_{\scriptscriptstyle j}, d_{\scriptscriptstyle j})$,  computes: $h_{\scriptscriptstyle j}=\mathtt{H}(l_{\scriptscriptstyle j}||d_{\scriptscriptstyle j})$, and sends a Merkle tree proof (that $h_{\scriptscriptstyle j}$ corresponds to  $R$) along with $\ddot{p}_{\scriptscriptstyle j}$ to the contract. In step \ref{check-hash}, the contract: (a) checks if $h_{\scriptscriptstyle j}=\mathtt{H}(l_{\scriptscriptstyle j}||d_{\scriptscriptstyle j})$, and  (b) verifies the Merkle tree proof.  The rest  remains unchanged.  As a result, the number of values stored in the contract is now $O(1)$. This adjustment comes with an added communication cost: $O(|h_{\scriptscriptstyle j}|\log z)$ for each verification. Nevertheless, the added cost is small and independent of the file size.   For instance, when  $z=10^{\scriptscriptstyle 6}$ and $|h_{\scriptscriptstyle j}|=256$, the  added communication cost is only about $5.1$ kilobit.

\vspace{-5mm}

\subsection{Evaluation}

In this section, we provide a summary of comparisons between   SO-PoR and outsourced PoRs  \cite{armknecht2014outsourced,xu2016lightweight,Storage-Time}. Among the two protocols  in \cite{Storage-Time} we only consider ``basic PoSt'' as it supports public verifiability. Briefly, in terms of property, only SO-PoR offers an explicit solution for real-time detection and fair payment. In terms of computation cost, the verification algorithm in SO-PoR is much faster than the other three protocols; Specifically, when $c=460$,  SO-PoR verification\footnote{As shown in \cite{DBLP:conf/ccs/AtenieseBCHKPS07}, to ensure $99\%$ of file blocks is retrievable, it  suffices to set $c=460$.} needs about $4.5$ times fewer computation than the verification required in the fastest outsourced PoR \cite{armknecht2014outsourced}.   Also, \cite{armknecht2014outsourced} has the worst store cost, that is  much higher than that of SO-PoR; e.g. for a $1$-GB file, SO-PoR requires over $46 \times 10^{\scriptscriptstyle 5}$ times fewer exponentiations than \cite{armknecht2014outsourced} needs in the same phase.  SO-PoR and \cite{Storage-Time} require a server to solve puzzles (and the imposed cost has to be compensated by the client) but the other two protocols do not need that. Also,  I/O cost and proof complexity of all protocols are $O(1)$ except \cite{Storage-Time} whose I/O cost and proof complexity are $O(\log n)$. The server-side bandwidth of SO-PoR is much lower than the rest;  e.g., for $1$-GB file and $z=100$ verifications, a server in SO-PoR requires $9\times 10^{\scriptscriptstyle4}$,  $7$ and $1729$ times fewer bits  than those required in \cite{armknecht2014outsourced}, \cite{xu2016lightweight} and \cite{Storage-Time} respectively.  A client in SO-PoR has a higher bandwidth than the rest (but this cost is one-off). Thus, SO-PoR offers additional properties, it has lower verification cost and lower server-side bandwidth than the rest while its other costs remain reasonable. Table \ref{table::O-PoR-Cost} outlines the cost comparison results. For a  full analysis, we refer readers to Appendix \ref{Full-Evaluation} where  we also compare SO-PoR costs with that of the most efficient  traditional PoR  \cite{DBLP:conf/asiacrypt/ShachamW08}.


\vspace{-3mm}
% !TEX root =main.tex


 \begin{table*}[!htbp]

\caption{ \small Outsourced PoR's Cost  Comparison. $z$:  total  verifications, $c$:   number of challenges for each verification, $n$:  total number of file blocks, $c'=(0.1)c$,  and $||\bm{F}||$:  file bit size.} \label{table::O-PoR-Cost} 
\begin{footnotesize}
\begin{center}
\renewcommand{\arraystretch}{.80}
%\scalebox{1}{
%\begin{subtable}{.56\linewidth}%xxxx
\begin{minipage}{1\linewidth}
%\caption{\small Computation and I/O Costs}
\setlength{\tabcolsep}{1pt}%compress the table horizontally
\begin{tabular}{|c|c|c|c|c|c|c|c|c|c|c|c|c|c|c|c|} 

   \hline
\cellcolor[gray]{.9}&\cellcolor[gray]{.9}&
 \multicolumn{4}{c|}{\cellcolor[gray]{.9}\scriptsize \underline{ \  \  \  \  \   \  \    \  \  \  \ \  \  \  \  \  \  \ \  \  \      Computation Cost     \  \  \  \  \   \  \    \  \  \  \ \  \  \  \  \  \  \ \  \  \   }}& \multicolumn{4}{c|}{\cellcolor[gray]{.9}\scriptsize \underline{\  \  \  \  \  \  \ \ \  \  \  \  \  \   \  \  \  \  \  \  \  \  Communication Cost   \  \  \  \  \   \  \    \  \  \  \ \  \  \  \  \  \  \ \  \  \   }}\\
% \cline{3-6}
  \cellcolor[gray]{.9}\multirow{-2}{*}{\scriptsize Protocols} &\cellcolor[gray]{.9}\multirow{-2}{*} {\scriptsize Operation}&\cellcolor[gray]{.9}\scriptsize$\mathtt{Store}$&\cellcolor[gray]{.9}\scriptsize$\mathtt{SolvPuz}$&\cellcolor[gray]{.9}\scriptsize$\mathtt{Prove}$&\cellcolor[gray]{.9}\scriptsize$\mathtt{Verify}$&\cellcolor[gray]{.9} {\scriptsize Client}&\cellcolor[gray]{.9} {\scriptsize Server}&\cellcolor[gray]{.9} {\scriptsize Verifier}&\cellcolor[gray]{.9}{\scriptsize Proof Size}
  \\
\hline
    %SO-PoR 1st row
\cellcolor[gray]{.9}& \multirow{2}{*}{\rotatebox[origin=c]{0}{\cellcolor[gray]{.9}\scriptsize }} \scriptsize Exp.&\scriptsize$z+1$&\scriptsize$T z$&\scriptsize$-$ &\scriptsize$-$&\scriptsize $128(n+$&&&\\
     \cline{2-6}  
     %SO-PoR 2nd row
 \multirow{-2}{*}{\rotatebox[origin=c]{0}{\cellcolor[gray]{.9}\scriptsize  SO-PoR }}&\cellcolor[gray]{.9}\scriptsize Add. or Mul.&\scriptsize$2(n+cz)$ &\scriptsize$z$&\scriptsize$4 c z$&\scriptsize$2z(1+c)$&\scriptsize  $cz+19z)$&\multirow{-2}{*}{\scriptsize $884z$}&\multirow{-2}{*}{\scriptsize$-$}&\multirow{-2}{*}{\scriptsize $O(1)$}\\
     \cline{2-6}   
      
     \hline 
       %[3] 1st row 
       
          \hline 
          
 \cellcolor[gray]{.9}  &\multirow{2}{*}{\rotatebox[origin=c]{0}{\scriptsize }}\cellcolor[gray]{.9}\scriptsize Exp.&\scriptsize$9 n$&\scriptsize$ -$&\scriptsize$-$&$-$&&\scriptsize$256z+$&\scriptsize $4672n+$&\\
     \cline{2-6}
     %[3] 2nd row 
\cellcolor[gray]{.9}   \multirow{-2}{*}{\rotatebox[origin=c]{0}{\scriptsize   \cite{armknecht2014outsourced}}}  &\cellcolor[gray]{.9}\scriptsize Add. or Mul.&\scriptsize$10 n$&$-$&\scriptsize$4 z(c+c')$&\scriptsize$z(9c+3)$&\multirow{-2}{*} {\scriptsize $128n$}&\scriptsize $||{\bm{F}}||$&\scriptsize $256z$&\multirow{-2}{*} {\scriptsize $O(1)$}\\   
      
      \hline
      
       \hline
      %[53] 1st row
  \cellcolor[gray]{.9}    &\multirow{2}{*}{\rotatebox[origin=c]{0}{\  \scriptsize }}\cellcolor[gray]{.9}\scriptsize Exp.&\scriptsize$-$&$-$&\scriptsize$z (3+c)$&\scriptsize$6 z$&&&&\\
     \cline{2-6}
     %[53] 2nd row
 \cellcolor[gray]{.9}&\cellcolor[gray]{.9}\scriptsize Add. or Mul.&\scriptsize$4 n$&\scriptsize$-$&\scriptsize$2 z(3 c+4)$&\scriptsize$2 c z$&&&&\\ 
    \cline{2-6}
    %[53] 3rd row 
\cellcolor[gray]{.9}\multirow{-3}{*}{\rotatebox[origin=c]{0}{\scriptsize   \cite{xu2016lightweight}}}&\cellcolor[gray]{.9}\scriptsize Pairing&\scriptsize$-$&\scriptsize$-$&\scriptsize$7 z$&$-$&\multirow{-3}{*} {\scriptsize $2048n$}&\multirow{-3}{*} {\scriptsize $6144z$}&\multirow{-3}{*} {\scriptsize$-$}&\multirow{-3}{*} {\scriptsize $O(1)$}\\ 
 \hline
  
  %%%%%%%%%%%%%%%%%%%%%%%%%%%%%%
 
 \hline
\cellcolor[gray]{.9}& \multirow{2}{*}{\rotatebox[origin=c]{0}{\cellcolor[gray]{.9}\scriptsize }} \scriptsize Exp.&\scriptsize$-$&\scriptsize$3T z$&\scriptsize$-$ &\scriptsize$3z$&&\scriptsize$128cz \log n+$&&\\
     \cline{2-6}  
 \multirow{-2}{*}{\rotatebox[origin=c]{0}{\cellcolor[gray]{.9}\scriptsize  \cite{Storage-Time} }}&\cellcolor[gray]{.9}\scriptsize Add. or Mul.&\scriptsize$-$ &\scriptsize$Tz$&\scriptsize$-$&\scriptsize$-$&\multirow{-2}{*} {\scriptsize $128$}&\scriptsize $4096z$&\multirow{-2}{*} {\scriptsize$-$}&\multirow{-2}{*} {\scriptsize$O(\log n)$} \\
     \cline{2-6}    
     \hline 
 
 %%%%%%%%%%%%%%%%%%%%%%%%%%%%%%
 
 
 
\end{tabular}  %xxxxxx
\end{minipage}
\end{center}
\end{footnotesize}
\end{table*}
\vspace{-2mm}

%--------------------------------------------------------


% \begin{table*}[!htbp]
%\begin{footnotesize}
%\begin{center}
%\caption{ \small Outsourced PoR Cost  Comparison} \label{table::O-PoR-Cost} 
%\renewcommand{\arraystretch}{.80}
%%\scalebox{1}{
%%\begin{subtable}{.56\linewidth}%xxxx
%\begin{minipage}{1\linewidth}
%\caption{\small Computation and I/O Costs}
%\setlength{\tabcolsep}{.55pt}%compress the table horizontally
%\begin{tabular}{|c|c|c|c|c|c|c|c|c|c|c|c|c|c|c|c|} 
%
%   \hline
%\cellcolor[gray]{.9}&\cellcolor[gray]{.9}&
% \multicolumn{4}{c|}{\cellcolor[gray]{.9}\scriptsize \underline{\  \  \  \  \  \  \ \ \  \  \  \  \  \      Algorithms' Computation Cost     \  \  \  \ \  \  \  \  \  \  \ \  \  \   }}&\cellcolor[gray]{.9}&\cellcolor[gray]{.9}&\cellcolor[gray]{.9}&\cellcolor[gray]{.9}&\cellcolor[gray]{.9}\\
%% \cline{3-6}
%  \cellcolor[gray]{.9}\multirow{-2}{*}{\scriptsize Protocols} &\cellcolor[gray]{.9}\multirow{-2}{*} {\scriptsize Operation}&\cellcolor[gray]{.9}\scriptsize$\mathtt{Store}$&\cellcolor[gray]{.9}\scriptsize$\mathtt{SolvPuz}$&\cellcolor[gray]{.9}\scriptsize$\mathtt{Prove}$&\cellcolor[gray]{.9}\scriptsize$\mathtt{Verify}$&\cellcolor[gray]{.9}\multirow{-2}{*} {\scriptsize I/O Cost}&\cellcolor[gray]{.9}\multirow{-2}{*} {\scriptsize Client}&\cellcolor[gray]{.9}\multirow{-2}{*} {\scriptsize Server}&\cellcolor[gray]{.9}\multirow{-2}{*} {\scriptsize Verifier}&\cellcolor[gray]{.9}\multirow{-2}{*} {\scriptsize Proof Size}
%  \\
%\hline
%    %SO-PoR 1st row
%\cellcolor[gray]{.9}& \multirow{2}{*}{\rotatebox[origin=c]{0}{\cellcolor[gray]{.9}\scriptsize }} \scriptsize Exp.&\scriptsize$z+1$&\scriptsize$T z$&\scriptsize$-$ &$-$&&\scriptsize $128(n+$&&&\\
%     \cline{2-6}  
%     %SO-PoR 2nd row
% \multirow{-2}{*}{\rotatebox[origin=c]{0}{\cellcolor[gray]{.9}\scriptsize  SO-PoR }}&\cellcolor[gray]{.9}\scriptsize Add. or Mul.&\scriptsize$2(n+cz)$ &\scriptsize$z$&\scriptsize$4 c z$&\scriptsize$2z(1+c)$&\multirow{-2}{*} {\scriptsize $O(1)$}&\scriptsize  $cz+19z)$&\multirow{-2}{*}{\scriptsize $884z$}&\multirow{-2}{*}{\scriptsize -}&\multirow{-2}{*}{\scriptsize $O(1)$}\\
%     \cline{2-6}   
%      
%     \hline 
%       %[3] 1st row 
%       
%          \hline 
%          
% \cellcolor[gray]{.9}  &\multirow{2}{*}{\rotatebox[origin=c]{0}{\scriptsize }}\cellcolor[gray]{.9}\scriptsize Exp.&\scriptsize$9 n$&\scriptsize$ -$&\scriptsize$-$&$-$&&&\scriptsize$256z+$&\scriptsize $4672n+$&\\
%     \cline{2-6}
%     %[3] 2nd row 
%\cellcolor[gray]{.9}   \multirow{-2}{*}{\rotatebox[origin=c]{0}{\scriptsize   \cite{armknecht2014outsourced}}}  &\cellcolor[gray]{.9}\scriptsize Add. or Mul.&\scriptsize$10 n$&$-$&\scriptsize$4 z(c+c')$&\scriptsize$z(9c+3)$&\multirow{-2}{*} {\scriptsize $O(1)$}&\multirow{-2}{*} {\scriptsize $128n$}&\scriptsize $||\vv{\bm{F}}||$&\scriptsize $256z$&\multirow{-2}{*} {\scriptsize $O(1)$}\\   
%      
%      \hline
%      
%       \hline
%      %[53] 1st row
%  \cellcolor[gray]{.9}    &\multirow{2}{*}{\rotatebox[origin=c]{0}{\  \scriptsize }}\cellcolor[gray]{.9}\scriptsize Exp.&\scriptsize$-$&$-$&\scriptsize$z (3+c)$&\scriptsize$6 z$&&&&&\\
%     \cline{2-6}
%     %[53] 2nd row
% \cellcolor[gray]{.9}&\cellcolor[gray]{.9}\scriptsize Add. or Mul.&\scriptsize$4 n$&$-$&\scriptsize$2 z(3 c+4)$&\scriptsize$2 c z$&&&&&\\ 
%    \cline{2-6}
%    %[53] 3rd row 
%\cellcolor[gray]{.9}\multirow{-3}{*}{\rotatebox[origin=c]{0}{\scriptsize   \cite{xu2016lightweight}}}&\cellcolor[gray]{.9}\scriptsize Pairing&\scriptsize$-$&\scriptsize$-$&\scriptsize$7 z$&$-$&\multirow{-3}{*} {\scriptsize $O(1)$}&\multirow{-3}{*} {\scriptsize $2048n$}&\multirow{-3}{*} {\scriptsize $6144z$}&\multirow{-3}{*} {-}&\multirow{-3}{*} {\scriptsize $O(1)$}\\ 
% \hline
%  
%  %%%%%%%%%%%%%%%%%%%%%%%%%%%%%%
% 
% \hline
%\cellcolor[gray]{.9}& \multirow{2}{*}{\rotatebox[origin=c]{0}{\cellcolor[gray]{.9}\scriptsize }} \scriptsize Exp.&\scriptsize$-$&\scriptsize$3T z$&\scriptsize$-$ &\scriptsize$3z$&&&\scriptsize$128cz \log n+$&&\\
%     \cline{2-6}  
% \multirow{-2}{*}{\rotatebox[origin=c]{0}{\cellcolor[gray]{.9}\scriptsize  \cite{Storage-Time} }}&\cellcolor[gray]{.9}\scriptsize Add. or Mul.&\scriptsize$-$ &\scriptsize$Tz$&\scriptsize$-$&\scriptsize$-$&\multirow{-2}{*} {\scriptsize $O(\log n)$}&\multirow{-2}{*} {\scriptsize $128$}&\scriptsize $4096z$&\multirow{-2}{*} {-}&\multirow{-2}{*} {\scriptsize$O(\log n)$} \\
%     \cline{2-6}    
%     \hline 
% 
% %%%%%%%%%%%%%%%%%%%%%%%%%%%%%%
% 
% 
% 
%\end{tabular}  %xxxxxx
%\end{minipage}
%\end{center}
%\end{footnotesize}
%\end{table*}
%

%--------------------------------------------------------













%In general, the overall bandwidth of SO-PoR is much lower than \cite{armknecht2014outsourced}, and is about $9\times$ higher than the outsourced PoR that requires a \emph{trusted} verifier, i.e. \cite{xu2016lightweight}, due to  higher  client-side bandwidth.  Also, a client bandwidth   in SO-PoR requires $128(cz+19z)$  more bits than a client in the privately verifiable PoR \cite{DBLP:conf/asiacrypt/ShachamW08}, while the server's bandwidth in SO-PoR is $3.4$ times higher than that in \cite{DBLP:conf/asiacrypt/ShachamW08}.




% !TEX root =main.tex
%\vspace{-1mm}

\section{C-TLP as Efficient Variant of VDF}\label{section::Variant-of-VDF}

\vspace{-2mm}

There are cases where a client wants a server to learn distinct random challenges, at different points in time within a certain period, without the client's involvement in that period. Such challenges can let the server generate certain proofs that include but are not limited to the \emph{continuous availability of services}, such as data storage or secure hardware. The obvious candidates that can meet the above needs are VDF (if  public verifiability is desirable) and TDF (if   private verifiability suffices). PoSt protocols  in \cite{Storage-Time} are two examples of the above cases. We observed that in these cases, VDF/TDF can be replaced with  C-TLP to gain better efficiency. The idea is that the client computes random challenges, encodes them into C-TLP puzzles and sends them to the server who can eventually solve each puzzle, extract a subset of challenges and use them for the related proof scheme while letting the public efficiently verify the solutions' correctness. To illustrate the efficiency gain,  we compare C-TLP performance with the two current VDF functions \cite{Wesolowski19,BonehBBF18}. Table \ref{table::VDF-comparison},  in Appendix \ref{table-VDF-cost-comparison}, summarises the result. The cost analysis considers the generic setting where $z$ outputs are generated. Among several VDF schemes proposed in \cite{BonehBBF18}, we focus on the one that uses sequential squaring, as it is more efficient than the other schemes in \cite{BonehBBF18}.   Brifely, the overall cost of \cite{BonehBBF18} in each of the three phases is much higher than C-TLP and \cite{Wesolowski19}. Now, we compare the computation cost of C-TLP  with  \cite{Wesolowski19}.  At setup, a client  in C-TLP  performs at most $3z+1$ more exponentiations than it does in \cite{Wesolowski19}. But, at both prove and verify phases, C-TLP   outperforms \cite{BonehBBF18}, especially when they are in the same model. In particular,  at the prove phase, C-TLP, in both models,  requires $Tz$ fewer multiplications than  \cite{Wesolowski19} does. Also, in the same phase, it requires $3$ times fewer exponentiations than \cite{Wesolowski19}. In the verify phase,  when C-TLP is in the
standard model, it has a slightly lower cost than \cite{Wesolowski19} has in the random oracle model. However, when both of them are in the random oracle model, C-TLP   has a much lower cost, as it requires no exponentiations whereas  \cite{Wesolowski19} needs $3z$ exponentiations.  Hence,  C-TLP   supports both standard and random oracle models and in both paradigms, it outperforms the fasted VDF, i.e. \cite{Wesolowski19}, designed in the random oracle model. Furthermore, the  proof size in C-TLP  is $3.2$ and $6.5$ times shorter than \cite{BonehBBF18} and  \cite{Wesolowski19} respectively, when they are in the same model.  In Appendix \ref{More-Efficient-Proof-of-Storage-Time},  we also show how C-TLP can be employed in the PoSt protocols \cite{Storage-Time} to reduce costs. 

%We highlight that, in general, VDF's offer certain features that (C-)TLP schemes do not offer, e.g. in VDF's parameters $z$  or their outputs do not need to be fixed ahead of time, or they can support multiple users. 

%\vspace{-4mm}
%% !TEX root =main.tex

%\vspace{-2mm}


 \begin{table}[htb]


%\begin{boxedminipage}{\columnwidth}
%\begin{center}
\caption{ \small VDF's Cost Comparison} \label{table::VDF-comparison} 
\begin{footnotesize}
\begin{center}
%{\small
\renewcommand{\arraystretch}{.87}
%\begin{minipage}{1\linewidth}
%\resizebox{\columnwidth}{!}{
{\small
%\scalebox{0.9}{
\begin{tabular}{|c|c|c|c|c|c|c|c|c|c|c|c|c|c|c|} 
   \hline
\cellcolor[gray]{0.9}&\cellcolor[gray]{0.9}&\cellcolor[gray]{0.9}&\multicolumn{3}{c|}{\scriptsize\cellcolor[gray]{.9} \underline{\ \ \ \ \ \ \   Computation Cost  \ \ \ \ \ \ \ }}& \cellcolor[gray]{0.9}\\

 %\cline{4-6}

\cellcolor[gray]{0.9}\multirow{-2}{*} {\scriptsize Protocols}&\cellcolor[gray]{0.9}\multirow{-2}{*}{\scriptsize Model}&\cellcolor[gray]{0.9}\multirow{-2}{*} {\scriptsize Operation}&\cellcolor[gray]{0.9}\scriptsize$\mathtt{Setup}$&\cellcolor[gray]{0.9}\scriptsize$\mathtt{Prove}$&\cellcolor[gray]{0.9}\scriptsize$\mathtt{Verify}$&\cellcolor[gray]{0.9}\multirow{-2}{*}{\scriptsize Proof size (bit)}\\
\hline
  %-----CR-TLP VDF
\cellcolor[gray]{0.9}&\cellcolor[gray]{0.9}&\cellcolor[gray]{0.9}\scriptsize Exp.&\scriptsize$3z+1$&\scriptsize$Tz$ &\scriptsize$2z$& \multirow{2}{*}{\rotatebox[origin=c]{0}{\scriptsize $1524z$}}  \\
     \cline{3-6}  
     
 \cellcolor[gray]{0.9}&\cellcolor[gray]{0.9}\multirow{-2}{*}{\rotatebox[origin=c]{0}{\scriptsize Standard}}&\cellcolor[gray]{0.9}\scriptsize Mul.&\scriptsize$z$&\scriptsize$-$ &\scriptsize$z$&\\
     \cline{2-7}  
\cellcolor[gray]{0.9}&\cellcolor[gray]{0.9}&\cellcolor[gray]{0.9}\scriptsize Exp.  &\scriptsize$z+1$&\scriptsize$Tz$&\scriptsize$-$&\multirow{2}{*}{\rotatebox[origin=c]{0}{\scriptsize $628z$}}  \\
     \cline{3-6}  
\cellcolor[gray]{0.9}\multirow{-4}{*}{\rotatebox[origin=c]{0}{\scriptsize C-TLP}}&\cellcolor[gray]{0.9}\multirow{-2}{*}{\rotatebox[origin=c]{0}{\scriptsize R.O.}}&\cellcolor[gray]{0.9}\scriptsize Mul.&\scriptsize$-$&\scriptsize$-$&\scriptsize$-$&\\
     \cline{2-7}    
     \hline 
     
          \hline 
          %-----Boneh
\cellcolor[gray]{0.9}&\cellcolor[gray]{0.9}&\cellcolor[gray]{0.9}\scriptsize Exp.&\scriptsize$2^{\scriptscriptstyle 30}$&\scriptsize$Tz$&\scriptsize$z$&\multirow{2}{*}{\rotatebox[origin=c]{0}{\scriptsize $2048z$}}\\
 \cline{3-6}  
\cellcolor[gray]{0.9}\multirow{-2}{*}{\rotatebox[origin=c]{0}{\scriptsize   \cite{BonehBBF18}}}&\cellcolor[gray]{0.9}\multirow{-2}{*}{\rotatebox[origin=c]{0}{\scriptsize R.O.}} &\cellcolor[gray]{0.9}\scriptsize Mul.&\scriptsize$-$&\scriptsize$2z\cdot 2^{\scriptscriptstyle 30}$&\scriptsize$2z\cdot 2^{\scriptscriptstyle 30}$& \\
     \cline{2-7} 
      \hline
      
       \hline
       %-----Wesolowski
\cellcolor[gray]{0.9}&\cellcolor[gray]{0.9}&\cellcolor[gray]{0.9}\scriptsize Exp.&$-$&\scriptsize$3Tz$&\scriptsize$3 z$&\multirow{2}{*}{\rotatebox[origin=c]{0}{\scriptsize $4096z$}}\\
\cline{3-6}  
\cellcolor[gray]{0.9}\multirow{-2}{*}{\rotatebox[origin=c]{0}{\scriptsize   \cite{Wesolowski19}}}&\cellcolor[gray]{0.9}\multirow{-2}{*}{\rotatebox[origin=c]{0}{\scriptsize R.O.}}&\cellcolor[gray]{0.9}\scriptsize Mul.&\scriptsize$-$&\scriptsize$Tz$&\scriptsize$-$&  \\
     \cline{2-6} 

 \hline
\end{tabular}
%}
}
%\end{boxedminipage}
%}%--small
\end{center}
\end{footnotesize}
\end{table}

\vspace{-6mm}






%The use of C-TLP offers the following additional  benefits. 

%Recall that two protocols: basic PoSt and compact PoSt supporting proof of storage-time are proposed in \cite{Storage-Time}, where  basic PoSt uses VDF and is publicly verifiable while compact PoSt  utilises trapdoor delay function (TDF) and is privately verifiable. Also, recall that VDF/TDF is used to allow the server to derive multiple challenges at different points over a certain time period $T$. 
%
%We highlight that  both delay functions (VDF and TDF) impose the same computation overhead to the server, i.e. $3Tz$ modular exponentiations and $Tz$ modular multiplication if the fastest delay function is used \cite{Wesolowski19}. Moreover, in the basic PoSt, a verifier has to perform $3z$ exponentiations to verify $z$ outputs of VDF.  We observed that if the delays functions in these schemes are replaced with  our C-TLP, then both basic and compact   PoSt protocols' overall cost will be significantly reduced. The idea as follows. As in PoSt protocols, the client at the setup precomputes random challenges (and their PoR tags). But, it encodes challenges for $z-1$ PoR proofs  into puzzles using C-TLP. It sends the challenges for the first PoR proof, the puzzles and encoded file to the server. As in the PoSt protocols, the server generates the first PoR proof using the challenges sent to it in the plaintext. Nevertheless, for the server to find $j^{\scriptscriptstyle th}$ challenge to generate  $j^{\scriptscriptstyle th}$ PoR proof, it solves the related puzzle. Therefore, it does not need to  call VDF or TDF anymore. If the basic PoSt  is used, then the server sends C-TLP proofs that can be efficiently verified by anyone. On the other hand, if the compact PoSt is used, then the server does not need to send the C-TLP's proofs, as the client already knows the random challenges. With the above adjustment  the server's computation cost would be $\frac{1}{3}$ of the costs imposed by either of PoSt protocols. The reason is that C-TLP's solve puzzle algorithm only involves $Tz$ exponentiations as opposed to $3Tz$ exponentiations needed by VDF/TDF. The use of C-TLP offers two more benefits as well; namely,  there will be $3z$ further reduction in the number of exponentiations:  (a) at the verifier side, in the basic PoSt, as the server does not need to perform any exponentiation to check the correctness of C-TLP's output, whereas VDF's verify algorithm requires $3z$ exponentiations, and (b) at the client side, in the compact PoSt,  as the client does not need to evaluate TDF at the setup to precompute the challenges which in total involves $3z$ modular exponentiations over $\phi(N)$.



\input{conclusion}

%\section{Introduction}
%ACM's consolidated article template, introduced in 2017, provides a
%consistent \LaTeX\ style for use across ACM publications, and
%incorporates accessibility and metadata-extraction functionality
%necessary for future Digital Library endeavors. Numerous ACM and
%SIG-specific \LaTeX\ templates have been examined, and their unique
%features incorporated into this single new template.
%
%If you are new to publishing with ACM, this document is a valuable
%guide to the process of preparing your work for publication. If you
%have published with ACM before, this document provides insight and
%instruction into more recent changes to the article template.
%
%The ``\verb|acmart|'' document class can be used to prepare articles
%for any ACM publication --- conference or journal, and for any stage
%of publication, from review to final ``camera-ready'' copy, to the
%author's own version, with {\itshape very} few changes to the source.

%\section{Template Overview}
%As noted in the introduction, the ``\verb|acmart|'' document class can
%be used to prepare many different kinds of documentation --- a
%double-blind initial submission of a full-length technical paper, a
%two-page SIGGRAPH Emerging Technologies abstract, a ``camera-ready''
%journal article, a SIGCHI Extended Abstract, and more --- all by
%selecting the appropriate {\itshape template style} and {\itshape
%  template parameters}.
%
%This document will explain the major features of the document
%class. For further information, the {\itshape \LaTeX\ User's Guide} is
%available from
%\url{https://www.acm.org/publications/proceedings-template}.

%\subsection{Template Styles}
%
%The primary parameter given to the ``\verb|acmart|'' document class is
%the {\itshape template style} which corresponds to the kind of publication
%or SIG publishing the work. This parameter is enclosed in square
%brackets and is a part of the {\verb|documentclass|} command:
%\begin{verbatim}
%  \documentclass[STYLE]{acmart}
%\end{verbatim}
%
%Journals use one of three template styles. All but three ACM journals
%use the {\verb|acmsmall|} template style:
%\begin{itemize}
%\item {\verb|acmsmall|}: The default journal template style.
%\item {\verb|acmlarge|}: Used by JOCCH and TAP.
%\item {\verb|acmtog|}: Used by TOG.
%\end{itemize}
%
%The majority of conference proceedings documentation will use the {\verb|acmconf|} template style.
%\begin{itemize}
%\item {\verb|acmconf|}: The default proceedings template style.
%\item{\verb|sigchi|}: Used for SIGCHI conference articles.
%\item{\verb|sigchi-a|}: Used for SIGCHI ``Extended Abstract'' articles.
%\item{\verb|sigplan|}: Used for SIGPLAN conference articles.
%\end{itemize}
%
%\subsection{Template Parameters}
%
%In addition to specifying the {\itshape template style} to be used in
%formatting your work, there are a number of {\itshape template parameters}
%which modify some part of the applied template style. A complete list
%of these parameters can be found in the {\itshape \LaTeX\ User's Guide.}
%
%Frequently-used parameters, or combinations of parameters, include:
%\begin{itemize}
%\item {\verb|anonymous,review|}: Suitable for a ``double-blind''
%  conference submission. Anonymizes the work and includes line
%  numbers. Use with the \verb|\acmSubmissionID| command to print the
%  submission's unique ID on each page of the work.
%\item{\verb|authorversion|}: Produces a version of the work suitable
%  for posting by the author.
%\item{\verb|screen|}: Produces colored hyperlinks.
%\end{itemize}
%
%This document uses the following string as the first command in the
%source file:
%\begin{verbatim}
%\documentclass[sigconf]{acmart}
%\end{verbatim}

%\section{Modifications}
%
%Modifying the template --- including but not limited to: adjusting
%margins, typeface sizes, line spacing, paragraph and list definitions,
%and the use of the \verb|\vspace| command to manually adjust the
%vertical spacing between elements of your work --- is not allowed.
%
%{\bfseries Your document will be returned to you for revision if
%  modifications are discovered.}

%\section{Typefaces}
%
%The ``\verb|acmart|'' document class requires the use of the
%``Libertine'' typeface family. Your \TeX\ installation should include
%this set of packages. Please do not substitute other typefaces. The
%``\verb|lmodern|'' and ``\verb|ltimes|'' packages should not be used,
%as they will override the built-in typeface families.
%
%\section{Title Information}
%
%The title of your work should use capital letters appropriately -
%\url{https://capitalizemytitle.com/} has useful rules for
%capitalization. Use the {\verb|title|} command to define the title of
%your work. If your work has a subtitle, define it with the
%{\verb|subtitle|} command.  Do not insert line breaks in your title.
%
%If your title is lengthy, you must define a short version to be used
%in the page headers, to prevent overlapping text. The \verb|title|
%command has a ``short title'' parameter:
%\begin{verbatim}
%  \title[short title]{full title}
%\end{verbatim}

%\section{Authors and Affiliations}
%
%Each author must be defined separately for accurate metadata
%identification. Multiple authors may share one affiliation. Authors'
%names should not be abbreviated; use full first names wherever
%possible. Include authors' e-mail addresses whenever possible.
%
%Grouping authors' names or e-mail addresses, or providing an ``e-mail
%alias,'' as shown below, is not acceptable:
%\begin{verbatim}
%  \author{Brooke Aster, David Mehldau}
%  \email{dave,judy,steve@university.edu}
%  \email{firstname.lastname@phillips.org}
%\end{verbatim}
%
%The \verb|authornote| and \verb|authornotemark| commands allow a note
%to apply to multiple authors --- for example, if the first two authors
%of an article contributed equally to the work.
%
%If your author list is lengthy, you must define a shortened version of
%the list of authors to be used in the page headers, to prevent
%overlapping text. The following command should be placed just after
%the last \verb|\author{}| definition:
%\begin{verbatim}
%  \renewcommand{\shortauthors}{McCartney, et al.}
%\end{verbatim}
%Omitting this command will force the use of a concatenated list of all
%of the authors' names, which may result in overlapping text in the
%page headers.
%
%The article template's documentation, available at
%\url{https://www.acm.org/publications/proceedings-template}, has a
%complete explanation of these commands and tips for their effective
%use.
%
%Note that authors' addresses are mandatory for journal articles.
%
%\section{Rights Information}
%
%Authors of any work published by ACM will need to complete a rights
%form. Depending on the kind of work, and the rights management choice
%made by the author, this may be copyright transfer, permission,
%license, or an OA (open access) agreement.
%
%Regardless of the rights management choice, the author will receive a
%copy of the completed rights form once it has been submitted. This
%form contains \LaTeX\ commands that must be copied into the source
%document. When the document source is compiled, these commands and
%their parameters add formatted text to several areas of the final
%document:
%\begin{itemize}
%\item the ``ACM Reference Format'' text on the first page.
%\item the ``rights management'' text on the first page.
%\item the conference information in the page header(s).
%\end{itemize}
%
%Rights information is unique to the work; if you are preparing several
%works for an event, make sure to use the correct set of commands with
%each of the works.
%
%The ACM Reference Format text is required for all articles over one
%page in length, and is optional for one-page articles (abstracts).
%
%\section{CCS Concepts and User-Defined Keywords}
%
%Two elements of the ``acmart'' document class provide powerful
%taxonomic tools for you to help readers find your work in an online
%search.
%
%The ACM Computing Classification System ---
%\url{https://www.acm.org/publications/class-2012} --- is a set of
%classifiers and concepts that describe the computing
%discipline. Authors can select entries from this classification
%system, via \url{https://dl.acm.org/ccs/ccs.cfm}, and generate the
%commands to be included in the \LaTeX\ source.
%
%User-defined keywords are a comma-separated list of words and phrases
%of the authors' choosing, providing a more flexible way of describing
%the research being presented.
%
%CCS concepts and user-defined keywords are required for for all
%articles over two pages in length, and are optional for one- and
%two-page articles (or abstracts).
%
%\section{Sectioning Commands}
%
%Your work should use standard \LaTeX\ sectioning commands:
%\verb|section|, \verb|subsection|, \verb|subsubsection|, and
%\verb|paragraph|. They should be numbered; do not remove the numbering
%from the commands.
%
%Simulating a sectioning command by setting the first word or words of
%a paragraph in boldface or italicized text is {\bfseries not allowed.}

%\section{Tables}
%
%The ``\verb|acmart|'' document class includes the ``\verb|booktabs|''
%package --- \url{https://ctan.org/pkg/booktabs} --- for preparing
%high-quality tables.
%
%Table captions are placed {\itshape above} the table.
%
%Because tables cannot be split across pages, the best placement for
%them is typically the top of the page nearest their initial cite.  To
%ensure this proper ``floating'' placement of tables, use the
%environment \textbf{table} to enclose the table's contents and the
%table caption.  The contents of the table itself must go in the
%\textbf{tabular} environment, to be aligned properly in rows and
%columns, with the desired horizontal and vertical rules.  Again,
%detailed instructions on \textbf{tabular} material are found in the
%\textit{\LaTeX\ User's Guide}.
%
%Immediately following this sentence is the point at which
%Table~\ref{tab:freq} is included in the input file; compare the
%placement of the table here with the table in the printed output of
%this document.
%
%\begin{table}
%  \caption{Frequency of Special Characters}
%  \label{tab:freq}
%  \begin{tabular}{ccl}
%    \toprule
%    Non-English or Math&Frequency&Comments\\
%    \midrule
%    \O & 1 in 1,000& For Swedish names\\
%    $\pi$ & 1 in 5& Common in math\\
%    \$ & 4 in 5 & Used in business\\
%    $\Psi^2_1$ & 1 in 40,000& Unexplained usage\\
%  \bottomrule
%\end{tabular}
%\end{table}
%
%To set a wider table, which takes up the whole width of the page's
%live area, use the environment \textbf{table*} to enclose the table's
%contents and the table caption.  As with a single-column table, this
%wide table will ``float'' to a location deemed more
%desirable. Immediately following this sentence is the point at which
%Table~\ref{tab:commands} is included in the input file; again, it is
%instructive to compare the placement of the table here with the table
%in the printed output of this document.
%
%\begin{table*}
%  \caption{Some Typical Commands}
%  \label{tab:commands}
%  \begin{tabular}{ccl}
%    \toprule
%    Command &A Number & Comments\\
%    \midrule
%    \texttt{{\char'134}author} & 100& Author \\
%    \texttt{{\char'134}table}& 300 & For tables\\
%    \texttt{{\char'134}table*}& 400& For wider tables\\
%    \bottomrule
%  \end{tabular}
%\end{table*}
%
%Always use midrule to separate table header rows from data rows, and
%use it only for this purpose. This enables assistive technologies to
%recognise table headers and support their users in navigating tables
%more easily.
%
%\section{Math Equations}
%You may want to display math equations in three distinct styles:
%inline, numbered or non-numbered display.  Each of the three are
%discussed in the next sections.
%
%\subsection{Inline (In-text) Equations}
%A formula that appears in the running text is called an inline or
%in-text formula.  It is produced by the \textbf{math} environment,
%which can be invoked with the usual
%\texttt{{\char'134}begin\,\ldots{\char'134}end} construction or with
%the short form \texttt{\$\,\ldots\$}. You can use any of the symbols
%and structures, from $\alpha$ to $\omega$, available in
%\LaTeX~\cite{Lamport:LaTeX}; this section will simply show a few
%examples of in-text equations in context. Notice how this equation:
%\begin{math}
%  \lim_{n\rightarrow \infty}x=0
%\end{math},
%set here in in-line math style, looks slightly different when
%set in display style.  (See next section).
%
%\subsection{Display Equations}
%A numbered display equation---one set off by vertical space from the
%text and centered horizontally---is produced by the \textbf{equation}
%environment. An unnumbered display equation is produced by the
%\textbf{displaymath} environment.
%
%Again, in either environment, you can use any of the symbols and
%structures available in \LaTeX\@; this section will just give a couple
%of examples of display equations in context.  First, consider the
%equation, shown as an inline equation above:
%\begin{equation}
%  \lim_{n\rightarrow \infty}x=0
%\end{equation}
%Notice how it is formatted somewhat differently in
%the \textbf{displaymath}
%environment.  Now, we'll enter an unnumbered equation:
%\begin{displaymath}
%  \sum_{i=0}^{\infty} x + 1
%\end{displaymath}
%and follow it with another numbered equation:
%\begin{equation}
%  \sum_{i=0}^{\infty}x_i=\int_{0}^{\pi+2} f
%\end{equation}
%just to demonstrate \LaTeX's able handling of numbering.
%
%\section{Figures}
%
%The ``\verb|figure|'' environment should be used for figures. One or
%more images can be placed within a figure. If your figure contains
%third-party material, you must clearly identify it as such, as shown
%in the example below.
%\begin{figure}[h]
%  \centering
%  \includegraphics[width=\linewidth]{pic/sample-franklin}
%  \caption{1907 Franklin Model D roadster. Photograph by Harris \&
%    Ewing, Inc. [Public domain], via Wikimedia
%    Commons. (\url{https://goo.gl/VLCRBB}).}
%  \Description{A woman and a girl in white dresses sit in an open car.}
%\end{figure}
%
%Your figures should contain a caption which describes the figure to
%the reader.
%
%Figure captions are placed {\itshape below} the figure.
%
%Every figure should also have a figure description unless it is purely
%decorative. These descriptions convey what’s in the image to someone
%who cannot see it. They are also used by search engine crawlers for
%indexing images, and when images cannot be loaded.
%
%A figure description must be unformatted plain text less than 2000
%characters long (including spaces).  {\bfseries Figure descriptions
%  should not repeat the figure caption – their purpose is to capture
%  important information that is not already provided in the caption or
%  the main text of the paper.} For figures that convey important and
%complex new information, a short text description may not be
%adequate. More complex alternative descriptions can be placed in an
%appendix and referenced in a short figure description. For example,
%provide a data table capturing the information in a bar chart, or a
%structured list representing a graph.  For additional information
%regarding how best to write figure descriptions and why doing this is
%so important, please see
%\url{https://www.acm.org/publications/taps/describing-figures/}.
%
%\subsection{The ``Teaser Figure''}
%
%A ``teaser figure'' is an image, or set of images in one figure, that
%are placed after all author and affiliation information, and before
%the body of the article, spanning the page. If you wish to have such a
%figure in your article, place the command immediately before the
%\verb|\maketitle| command:
%\begin{verbatim}
%  \begin{teaserfigure}
%    \includegraphics[width=\textwidth]{pic/sampleteaser}
%    \caption{figure caption}
%    \Description{figure description}
%  \end{teaserfigure}
%\end{verbatim}

%\section{Citations and Bibliographies}
%
%The use of \BibTeX\ for the preparation and formatting of one's
%references is strongly recommended. Authors' names should be complete
%--- use full first names (``Donald E. Knuth'') not initials
%(``D. E. Knuth'') --- and the salient identifying features of a
%reference should be included: title, year, volume, number, pages,
%article DOI, etc.
%
%The bibliography is included in your source document with these two
%commands, placed just before the \verb|\end{document}| command:
%\begin{verbatim}
%  \bibliographystyle{ACM-Reference-Format}
%  \bibliography{bibfile}
%\end{verbatim}
%where ``\verb|bibfile|'' is the name, without the ``\verb|.bib|''
%suffix, of the \BibTeX\ file.
%
%Citations and references are numbered by default. A small number of
%ACM publications have citations and references formatted in the
%``author year'' style; for these exceptions, please include this
%command in the {\bfseries preamble} (before the command
%``\verb|\begin{document}|'') of your \LaTeX\ source:
%\begin{verbatim}
%  \citestyle{acmauthoryear}
%\end{verbatim}
%
%  Some examples.  A paginated journal article \cite{Abril07}, an
%  enumerated journal article \cite{Cohen07}, a reference to an entire
%  issue \cite{JCohen96}, a monograph (whole book) \cite{Kosiur01}, a
%  monograph/whole book in a series (see 2a in spec. document)
%  \cite{Harel79}, a divisible-book such as an anthology or compilation
%  \cite{Editor00} followed by the same example, however we only output
%  the series if the volume number is given \cite{Editor00a} (so
%  Editor00a's series should NOT be present since it has no vol. no.),
%  a chapter in a divisible book \cite{Spector90}, a chapter in a
%  divisible book in a series \cite{Douglass98}, a multi-volume work as
%  book \cite{Knuth97}, a couple of articles in a proceedings (of a
%  conference, symposium, workshop for example) (paginated proceedings
%  article) \cite{Andler79, Hagerup1993}, a proceedings article with
%  all possible elements \cite{Smith10}, an example of an enumerated
%  proceedings article \cite{VanGundy07}, an informally published work
%  \cite{Harel78}, a couple of preprints \cite{Bornmann2019,
%    AnzarootPBM14}, a doctoral dissertation \cite{Clarkson85}, a
%  master's thesis: \cite{anisi03}, an online document / world wide web
%  resource \cite{Thornburg01, Ablamowicz07, Poker06}, a video game
%  (Case 1) \cite{Obama08} and (Case 2) \cite{Novak03} and \cite{Lee05}
%  and (Case 3) a patent \cite{JoeScientist001}, work accepted for
%  publication \cite{rous08}, 'YYYYb'-test for prolific author
%  \cite{SaeediMEJ10} and \cite{SaeediJETC10}. Other cites might
%  contain 'duplicate' DOI and URLs (some SIAM articles)
%  \cite{Kirschmer:2010:AEI:1958016.1958018}. Boris / Barbara Beeton:
%  multi-volume works as books \cite{MR781536} and \cite{MR781537}. A
%  couple of citations with DOIs:
%  \cite{2004:ITE:1009386.1010128,Kirschmer:2010:AEI:1958016.1958018}. Online
%  citations: \cite{TUGInstmem, Thornburg01, CTANacmart}. Artifacts:
%  \cite{R} and \cite{UMassCitations}.

%\section{Acknowledgments}
%
%Identification of funding sources and other support, and thanks to
%individuals and groups that assisted in the research and the
%preparation of the work should be included in an acknowledgment
%section, which is placed just before the reference section in your
%document.
%
%This section has a special environment:
%\begin{verbatim}
%  \begin{acks}
%  ...
%  \end{acks}
%\end{verbatim}
%so that the information contained therein can be more easily collected
%during the article metadata extraction phase, and to ensure
%consistency in the spelling of the section heading.
%
%Authors should not prepare this section as a numbered or unnumbered {\verb|\section|}; please use the ``{\verb|acks|}'' environment.

%\section{Appendices}
%
%If your work needs an appendix, add it before the
%``\verb|\end{document}|'' command at the conclusion of your source
%document.
%
%Start the appendix with the ``\verb|appendix|'' command:
%\begin{verbatim}
%  \appendix
%\end{verbatim}
%and note that in the appendix, sections are lettered, not
%numbered. This document has two appendices, demonstrating the section
%and subsection identification method.
%
%\section{SIGCHI Extended Abstracts}
%
%The ``\verb|sigchi-a|'' template style (available only in \LaTeX\ and
%not in Word) produces a landscape-orientation formatted article, with
%a wide left margin. Three environments are available for use with the
%``\verb|sigchi-a|'' template style, and produce formatted output in
%the margin:
%\begin{itemize}
%\item {\verb|sidebar|}:  Place formatted text in the margin.
%\item {\verb|marginfigure|}: Place a figure in the margin.
%\item {\verb|margintable|}: Place a table in the margin.
%\end{itemize}

%%
%% The acknowledgments section is defined using the "acks" environment
%% (and NOT an unnumbered section). This ensures the proper
%% identification of the section in the article metadata, and the
%% consistent spelling of the heading.
%\begin{acks}
%To Robert, for the bagels and explaining CMYK and color spaces.
%\end{acks}

%%
%% The next two lines define the bibliography style to be used, and
%% the bibliography file.
\bibliographystyle{splncs03}
\bibliography{main-ref}

%%%%%%%%%%%%%%%
\appendix
% !TEX root =main.tex
\section{Survey of Related Work}\label{Survey-of-Related-Work}
\subsection{Time-lock Puzzles}\label{survey-time-lock-puzzle}

The idea to send information into the \emph{future}, i.e.
time-lock puzzle/encryption, was first put forth by Timothy C. May. A time-lock puzzle allows a party to encrypt a message such that it cannot be decrypted  until a certain amount of time has passed. In general,  a  time-lock scheme should allow    generating (and verifying) a puzzle to take less time than solving it. The time-lock puzzle scheme that May proposed lies on a trusted agent. Later on, Rivest \textit{et al} \cite{Rivest:1996:TPT:888615} propose a protocol that does not require a trusted agent and is secure against a receiver
who may have access to many  computation resources that can be run in parallel. It is based on Blum-Blum-Shub pseudorandom number generator that relies on modular repeated squaring, believed to be sequential. The scheme in \cite{Rivest:1996:TPT:888615} allows (only) the puzzle creator to  verify the correctness of the puzzle solution using a secret key and the original secret message.  This scheme has been the core of (almost) all later time-lock puzzles schemes that supports the encapsulation of an arbitrary message. Later on, \cite{BonehN00,DBLP:conf/fc/GarayJ02} proposed timed commitment schemes that offer more security properties, in the sense that they   allow a puzzle generator to prove (in Zero-knowledge) to a puzzle solver that the correct solution (e.g. a signature of a public document) will be recovered after a certain time before the solver starts solving the puzzle. These schemes are more complex, due to the use of zero-knowledge proofs, and less efficient than \cite{Rivest:1996:TPT:888615}.    Recently, \cite{MalavoltaT19,BrakerskiDGM19}  propose protocols for homomorphic time-lock puzzles, where an arbitrary function can be run over puzzles before they are solved. They mainly use  fully homomorphic encryption and   the RSA puzzle, proposed in \cite{BrakerskiDGM19}, in a nutshell. In these protocols, all puzzles have an identical time parameter, and their solutions are supposed to be discovered at the same time. The main difference between the two protocols is the security assumption they rely on (i.e. the former uses a non-standard assumption while the latter relies on a standard one). Since both schemes use a generic fully homomorphic encryption, it is not hard to make them publicly verifiable. However, they  are only of theoretical interest as in practice they impose  high computation and communication costs, due to the use of fully homomorphic encryptions. 


Very recently, Chvojka \textit{et al.} in \cite{ChvojkaJSS20} propose  incremental time-release encryption that allows a server to decrypt messages sequentially over time. This is the closest work to ours. Nevertheless, the scheme uses the RSA time-lock puzzle \cite{Rivest:1996:TPT:888615} in a \emph{completely non-black-box manner}, offers no (public) verification, and is based on  asymmetric key encryption instead of symmetric key encryption used in the majority of time lock-puzzle schemes including in \cite{Rivest:1996:TPT:888615}. The latter issue leads to a higher computation cost (compared to standard time lock puzzles) and can negatively affect  efficiency gains the scheme tries to achieve.  At a high level, the scheme works as follows. A client who wants its messages $[m_{\scriptscriptstyle 1},m_{\scriptscriptstyle 2},…, m_{\scriptscriptstyle n}]$ to be revealed at time $[f_{\scriptscriptstyle 1}, f_{\scriptscriptstyle 2}…, f_{\scriptscriptstyle n}]$, at setup computes $g^{\scriptscriptstyle 2^{\scriptscriptstyle T_{\scriptscriptstyle 1}}} (\bmod N),$  $g^{\scriptscriptstyle 2^{\scriptscriptstyle T_{\scriptscriptstyle 2}}}(\bmod N),…,$  $g^{\scriptscriptstyle 2^{\scriptscriptstyle T_{\scriptscriptstyle n}}}(\bmod N)$, where $g$ is a random value, $N$ is a RSA modulus and  $T_{\scriptscriptstyle i}$ is the number of squaring required to find message $m_{\scriptscriptstyle i}$ at time $f_{\scriptscriptstyle i}$. Then, it uses  asymmetric key encryption to encrypt each $m_{\scriptscriptstyle i}$, such that $g^{\scriptscriptstyle 2^{\scriptscriptstyle T_{\scriptscriptstyle i}}}$ is used as  randomness for the encryption’s key generation algorithm. Then, it publishes all ciphertexts and $g$ at time $f_{\scriptscriptstyle 0}<f_{\scriptscriptstyle 1}$. For a server to discover $m_{\scriptscriptstyle 1}$, it performs $T_{\scriptscriptstyle 1}$ squaring to find $g^{\scriptscriptstyle 2^{\scriptscriptstyle T_{\scriptscriptstyle 1}}}$. Then, it runs the encryption’s key generation algorithm that takes $g^{\scriptscriptstyle 2^{\scriptscriptstyle T_{\scriptscriptstyle 1}}}$ as input and returns secret and public keys. The server uses the secret key to decrypt the message. After that, it performs modular squaring on $g^{\scriptscriptstyle 2^{\scriptscriptstyle T_{\scriptscriptstyle 1}}}$ until it computes $g^{\scriptscriptstyle 2^{\scriptscriptstyle T_{\scriptscriptstyle 2}}}$. Then, it calls the key generation algorithm again this time using $g^{\scriptscriptstyle 2^{\scriptscriptstyle T_{\scriptscriptstyle 2}}}$ as the randomness to find the secret key with which it can discover $m_{\scriptscriptstyle 2}$. It carries on the above process until it discovers $m_{\scriptscriptstyle n}$. The idea of computing $g^{\scriptscriptstyle 2^{\scriptscriptstyle T_{\scriptscriptstyle i+1}}}$ from $g^{\scriptscriptstyle 2^{\scriptscriptstyle T_{\scriptscriptstyle i}}}$, proposed in the above scheme,  is very similar to the notion of ``time-line’’ proposed by Garay \textit{et al.} in \cite{DBLP:conf/fc/GarayJ02}. Also, the asymmetric encryption used in the above scheme is not standard. Because its key generation algorithm is  deterministic and  takes an extra input which is randomness from which public key and private keys are extracted. Whereas, in standard asymmetric encryption schemes, a key generation algorithm is probabilistic and  only takes a security parameter. Moreover, as discussed in section \ref{C-TLP-overview}, in the case where the time intervals are equal, e.g. $T_{\scriptscriptstyle N}-T_{\scriptscriptstyle N-1}=T_{\scriptscriptstyle j}-T_{\scriptscriptstyle j-1}$, the client still has to perform in total $n-1$ modular multiplications to efficiently compute all $g^{\scriptscriptstyle 2^{\scriptscriptstyle T_{\scriptscriptstyle 1}}} , g^{\scriptscriptstyle 2^{\scriptscriptstyle T_{\scriptscriptstyle 2}}},… g^{\scriptscriptstyle 2^{\scriptscriptstyle T_{\scriptscriptstyle n}}}$, which is not optimal either. 

%Furthermore, \cite{KarameC10} proposes a privately verifiable puzzle scheme that has up to $12\times$  lower cost than \cite{Rivest:1996:TPT:888615} in puzzle generation and verification phases. However,  it relies on  a new and non-standard assumption (i.e. computationally infeasible to compute a small private exponent when a public exponent is much larger than the RSA modulus).



We also cover two related but different notions, pricing puzzles, and verifiable delay functions. 

\noindent\textbf{\textit{Pricing Puzzles.}} Also known as \emph{client puzzles}. It was first put forth by Dwork \textit{et al.} \cite{DworkN92} who defined it as a function that requires a certain amount of computation resources to solve a puzzle.  In general, the pricing puzzles are based on either hash inversion problems or number-theoretic. In the former category, 
a puzzle generator  generates a puzzle as: $h= \mathtt{H}(m||r)$, where $\mathtt{H}$ is a hash function, $m$ is a public value and $r$ is a random value of a fixed size. Given $h, \mathtt{H}$ and $m$, the solver must find $r$ such that the above equation holds. The size of $r$ is picked in such a way that the expected time to find the solution is fixed (however it does not rule out finding the solution on the first attempt). The above hash-based scheme allows a solver to find a solution faster if it has more computational power resources  running in parallel. The application area of such a puzzle includes  defending against denial-of-service (DoS)  attacks, reaching a consensus in cryptocurrencies, etc. A  variant  of such a puzzle uses iterative hashing; for instance, to generate a set of puzzles  in the case where the solver receives a service proportional to the number of puzzles it solves \cite{groza2006chained},  or to generate password puzzle to mitigate DoS attacks \cite{Ma05}. However, the iterative hashing schemes are partially parallelizable, in the sense that every single invocation of the hash function can be run in parallel. Later on,  \cite{MahmoodyMV11} investigates the possibility of constructing (time-lock) puzzles in the random oracle model.  Their main result was negative, that rules out time-lock puzzles that require more parallel time to solve than the total work required to generate.  Also \cite{MahmoodyMV11} proposes an iterative hash-based mechanism (very similar to \cite{Ma05}) that allows a puzzle generator to generate a puzzle with $n$ parallel queries to the random oracle, but the solver needs $n$ rounds of serial queries. Nevertheless, this scheme is also partially parallelizable, as each instance of the puzzle can be solved in parallel. Note that the above hash-based puzzle schemes would have very limited applications if they are used directly to  encapsulate a message: $m'$ of arbitrary size. The reason is that, in these schemes,  the solution  size: $|r|$ plays a vital role in (adjusting) the  time taken to solve the puzzle. If the solution size becomes bigger, as a result of combining $r$ with $m'$, i.e. $r \odot m'$, then it would take longer to find the solution. This means  the puzzles can be used only in the cases where  the time required to find a solution is long enough, and is a function of $|r \odot m'|$, which seriously restricts its application. Researchers also propose non-parallelizable pricing puzzles based on number theoretic \cite{WatersJHF04,KuppusamyRSBN12,KarameC10} whose main application is to resist DoS attacks. These schemes  have a more efficient verification mechanism than the one proposed in \cite{Rivest:1996:TPT:888615}. But, they are only privately verifiable and not designed to encapsulate an arbitrary message. 





\noindent\textbf{\textit{Verifiable Delay Function (VDF).}} Allows a prover to provide a publicly verifiable proof stating  it has performed  a pre-determined number of sequential computations. It has many applications, e.g. in decentralised systems to extract  trustworthy public randomness from a blockchain. VDF was first formalised by Boneh \textit{et al} in \cite{BonehBBF18} who proposed several VDF constructions based on SNARKs along with either  incrementally verifiable computation or injective polynomials, or based on time-lock puzzles, where  the SNARKs based approaches require a trusted setup.  Later on,  \cite{Wesolowski19} improved the previous VDF's  from different perspectives and proposed a scheme  based on RSA time-lock encryption, in the random oracle model. To date, this protocol is the most efficient VDF.  It also supports batch verification, such that given a single proof a verifier can efficiently check the validity of multiple outputs of the verifiable delay function. As discussed above, (most of) VDF schemes are built  upon time-lock puzzles, however the converse is not necessarily the case, as VDF's are not designed to encapsulate an  arbitrary private message, and they take a public message as input while time-lock puzzles are designed to conceal a private input message. 



\subsection{Proof of Storage}\label{related-work-PoR}

\subsubsection{Traditional Proof of Storage}

Proof of storage (PoS) is a cryptographic protocol that allows a client to  efficiently verify the integrity or availability of its  data stored in a remote server, not necessarily trusted \cite{DBLP:conf/cai/Kamara13}. PoS  can be categorised into two broad classes:   Proofs of Retrievability (PoR) and Proofs of Data Possession (PDP). The main difference between the two categories is the level of security assurance provided. PoR schemes guarantee that the server maintains knowledge of \emph{all} of the client's outsourced data, while  PDP protocols only ensure that the server is storing \emph{most} of the client data. From a technical point of view,  the main difference in most prior PDP and PoR constructions is that PoR schemes store a redundant encoding of the client data on the server by employing an error-correcting code, e.g. Reed-Solomon codes, while such encoding is not used in  PDP schemes. Hence, PoR schemes provide stronger security guarantees compared to PDP at the  cost of additional storage space and encoding/decoding, (for more details see \cite{kupcu2010efficient,DBLP:conf/ccs/ShiSP13,Cash:2017:DPR:3038037.3038087}). Furthermore, each PoR and PDP scheme can be also grouped into two categories; namely, publicly and privately verifiable. In a publicly verifiable scheme, everyone without knowing a secret can verify  proof, whereas a verifier in a privately verifiable scheme requires the knowledge of a secret. Moreover, in general, in PoR  and PDP schemes there are no assumptions on the behaviour of the adversary. It is allowed to deviate from the protocol arbitrarily, i.e.  an ‘’active’’ adversary.





%As highlighted in \cite{kupcu2010efficient,DBLP:conf/ccs/ShiSP13,Cash:2017:DPR:3038037.3038087}, the main difference between PoR and PDP is the level of the security they achieve; 

The notion of PoR first was introduced and defined in \cite{DBLP:conf/ccs/JuelsK07}, where the authors designed a protocol that uses  random sentinels, symmetric key encryption,  error-correcting code, and pseudorandom permutation. In this protocol, a client in the setup phase, applies error-correcting code to every file block and encrypts each encoded block. Then, it computes a set of random sentinels and appends them to the encrypted file. It  permutes all values (i.e. sentinels and encrypted file blocks) and sends them to the cloud  server.  To check if  the server has retained the file, the client specifies  random positions of some sentinels in the encoded file and asks
the server to return those sentinel values. Next, the client checks if it gets the sentinels it asked for. In this scheme, the security holds, as the server cannot feasibly distinguish between sentinels and the actual file blocks and the sentinels have been distributed uniformly among the file blocks. But, as individual sentinel is only one-time verifiable, there is an upper bound on the number of verifications performed by the client, and when reached, the client has to re-encode the file.  To overcome the issues related to the bounded number of verifications, the authors have also suggested that sentinels can be replaced with  MAC on every file block or a Merkle tree constructed on the file blocks. This protocol is computationally efficient and its communication cost is linear with the number of challenges sent\footnote{It is not hard to make the communication cost of this scheme constant; for instance, by letting the server order the challenged sentinels, concatenate them and send a hash of the concatenated value to the client.}. The sentinel and MAC-based schemes above are privately verifiable while the one uses a Merkle tree supports public verifiability. However, the publicly verifiable Merkle tree-based approach has a communication cost logarithmic with the file size and the prover has to send a set of file blocks to the verifier that yields a high communication cost. 


Later on, \cite{DBLP:conf/asiacrypt/ShachamW08} improves the previous sentinel-based scheme and definition, and proposes two PoR protocols (with an unlimited number of verifications), one uses  MAC and the other one relies on BLS signatures. In particular, a client at setup phase, encodes its file blocks using error-correcting code and then for each encoded file block it generates a  tag that can be either a MAC or BLS signature of that block. In the verification phase, it specifies a set of random indices corresponding to the file's blocks; and accordingly the server sends a proof to it. These two  schemes have a low communication cost, as due   to the homomorphic property of the tags,  the prover can aggregate  proofs  into a single authenticator value and no file blocks  are sent to the prover. Also, the protocol based on MAC supports efficient private verification. But, the one that uses BLS signatures supports public verifiability at the cost of public key operations and it is far less efficient than the MAC-based one. Within the  last decade, researchers have extended PoR protocols from several perspectives, e.g. those that support:  efficient update \cite{DBLP:conf/ccs/ShiSP13}, the delegation of file pre-processing \cite{ArmknechtBBK16}, or the delegation of verification \cite{armknecht2014outsourced}.  


On the other hand, PDP was first introduced in \cite{DBLP:conf/ccs/AtenieseBCHKPS07}. PDP protocols focus only on verifying the  integrity of outsourced data. In this setting,  clients can ensure that a certain percentage of data blocks are available. The authors in \cite{DBLP:conf/ccs/AtenieseBCHKPS07} propose two schemes, for public and private verification both of which use RSA-based homomorphic verifiable tags  generated for each file block and use the spot-checking techniques (similar to \cite{DBLP:conf/asiacrypt/ShachamW08}) to check a random subset of a file's blocks. In both schemes, the proof size is constant, while the verification cost is high as it requires  many  exponentiations over an RSA ring. Later on, an  efficient and scalable  PDP scheme that supports a limited number of verifications is proposed in \cite{AteniesePMT08}. The scheme supports only privately verifiable PDP and is based on a combination of pre-computation technique and symmetric key primitives.  Ever since, researcher proposed different variants of PDP, e.g.  dynamic PDP \cite{ErwayKPT09}, multi-replica PDP\cite{DistributedPDP}, keyword-based PDP \cite{SenguptaR18}.  

\
Thus,  (a) in publicly verifiable PoS it is assumed the verifier is fully trusted with the verification correctness, (b) these schemes either have a very high verification cost when the proof size is constant (when signature-based tags are used) or have a communication cost logarithmic with the entire file size if their verification is efficient (when a Merkle tree is used),  and (c) the privately verifiable proofs can have a constant proof size and efficient verification algorithm, but the data owner has to perform the verification. 



\subsubsection {Outsourced PoR}\label{Outsourced-PoS}

Recently,  \cite{armknecht2014outsourced,xu2016lightweight} propose \emph{outsourced} PoR protocols that allow  clients to outsource the   verification to a third party auditor not necessarily  trusted. The scheme in \cite{armknecht2014outsourced} uses MAC-based tags, zero-knowledge proofs, and error-correcting codes. At a high level, the protocol works as follows. At the setup,   the client  encodes its data using error-correcting codes, generates MAC on every file block, and stores the encoded file and tags on the server.  Then, the auditor downloads the encoded file, generates another set of MAC's on the file blocks. It uploads the MAC's to the cloud. Also, the auditor proves to the client in zero-knowledge that it has created each MAC correctly. To do that, it uses a non-interactive zero-knowledge proof in the random oracle model. If the client accepts all zero-knowledge proofs, it sings every proof and sends the signatures back to the auditor. In the verification phase, the auditor sends two sets of random challenges extracted from a blockchain. Upon receiving the challenges, the server provides two separate proofs, one for the auditor and the other one for the client. The auditor verifies the proof generated for it and  locally stores the clients' proofs. In the case where the auditor's proof is not accepted, an honest auditor would inform the client who will come online and checks both its proofs and the auditors' proof. The scheme provides two layers of verification to the client: \textit{CheckLog} and \textit{ProveLog}.  \textit{CheckLog}  is more efficient than the other one, as the auditor sends  much fewer challenges to the server to generate the client's proof for each verification. In the case where the client's check fails, the client will assume that either the server or auditor has acted maliciously, and to pinpoint the malicious party, it needs to proceed to \textit{ProveLog} where allows the client to audit the auditor. In this verification, the auditor must reveal its secret keys with which the client checks all the proofs provided by the server to the auditor for the entire period that the client was offline. In this protocol, the verification phase is efficient (due to the use of MACs). In particular, the verification's computation cost for the auditor and client is linear with the number of challenges and their communication cost is constant in the file size.  

Although the protocol in  \cite{armknecht2014outsourced}  is appealing, it has  several shortcomings: (a) auditor can get a free ride: in the case where a known highly reputable  server, e.g. Google or Amazon, always generates   accepting proofs for both client and auditor, an economically rational auditor can skip performing its part (e.g. to save computation cost) and get still paid by the client. This deviation from the protocol can never be detected by the client in the protocol.  (b) no guarantee for a real-time detection/notification: the client may not be notified in the real-time about the data unavailability if the auditor is malicious; therefore, this scheme is not suitable for the case where a client's involvement for the data extraction is  immediately needed once a misbehaviour is detected, e.g., in the case of hardware depreciation. (c) a high cost of auditor onboarding:  revoking an auditor and onboarding a new one, imposes a high cost on the client and new auditor, as new auditor need to re-run the setup phase (that includes data downloading, zero-knowledge proofs) and the client needs to verify and sign the zero-knowledge proofs. (d) \textit{ProveLog} costs even an honest auditor: the only way for the client to ensure the auditor has fully followed the protocol is to run \textit{ProveLog} that requires the auditor to reveal its secret values. After that,  an honest auditor has to re-run the computationally expensive setup  again. Since the auditor is not trusted and no guarantee that it alerts the client as soon as the server's misbehaviour is detected, its involvement in the protocol seems unnecessary; and the whole scheme can be replaced with a much simpler one: the client, similar to a standard privately verifiable PoR, encodes its data and stores it in the server. Then, for each verification time, the server gets the random challenges from the blockchain and publishes  proofs in  bulletin board (to timestamp it). When the client comes online, it checks the proofs and accordingly takes the required action, e.g. pays the server, tries to recover the file.

Xu et al. in \cite{xu2016lightweight} propose a publicly verifiable PoR to improve the computation cost at setup. The scheme uses   BLS signatures-like tags, polynomial arithmetics, and polynomial commitments; unlike previous schemes,  each tag has a more complex structure.  In this protocol, an auditor is assumed to be fully trusted during the verifications. Later on, however, when  it is revoked by the client it may become malicious, i.e. may collude with the server and reveal its secret. This scheme allows a client to update its tags when a new auditor joins. To do that, the client needs to download all the blocks' tags, refresh them locally, and uploaded the fresh tags to the server. The verification for auditor and client involves public key operations and is more expensive than the one in \cite{armknecht2014outsourced}. The use of an auditor's revocation remains unclear in the paper, as  auditors are assumed to be honest in this protocol. A delegatable PDP is proposed in \cite{ShenT11} to ensure only authorised parties can verify the integrity of remote data. However, the delegated party in this scheme is fully trusted with the correctness of verification it performs. The scheme uses authenticator tags  similar to   BLS signatures based ones.


Very recently, proof of ``storage-time'' has been proposed in \cite{Storage-Time}. The paper proposes two protocols: basic PoSt and compact PoSt. At a high level, they offer the guarantee to a client that its data has been available on a remote storage server for a fixed time period: $T$.  The idea is that a client uploads to a server its data once, then  the server generates proofs of storage (e.g. PoR) periodically, collects them and sends the collection after the time $T$ to a verifier (i.e. client in basic PoSt or third-party in compact PoSt) who can check the proof. The first protocol, basic PoSt, uses Merkle tree-based PoR and VDF. In this protocol, The client precomputes a set of metadata (tags, challenges, and their related proofs). It  sends  the file, metadata, and a single challenge to the server. The server uses the  challenge to generate a PoR. It feeds the hash of the PoR to VDF to generate an output. It considers the hash of VDF's output as the next challenge from which  the next PoR is generated. This process goes on until all $z$ PoR's are generated. It sends all PoR proofs along with the proofs proving the correctness of VDF's outputs to a verifier. Given the two sets of proofs, a verifier can check their correctness and ultimately conclude that the file was retrievable within the time period. The scheme is publicly verifiable. However, the verification's computation cost is significant.  As, it requires  the validator to validate the correctness of VDF's outputs, that imposes a high cost. In total (for $z$ PoR proofs) it involves at least $3z$ modular exponentiations even if it uses  the fastest VDF, proposed in \cite{Wesolowski19}. Such costs are missed out and not taken into account in the paper. Moreover, the use of the Merkle tree introduces a high communication cost, as well. The authors suggest a smart contract can play the validator's role. However, we argue that this would impose a significant financial cost to the contract and users due to very high computation and communication costs stem from the verification and prove algorithms. The second protocol, compact PoSt, allows the server to combine PoR proofs that ultimately reduces the communication cost. The protocol's design is very similar to the previous one, but it replaces the Merkle tree approach with simply hashing the entire file and replaces the VDF with a trapdoor delay function (TDF) that requires a secret to generate/verify the delay function's output. Therefore, this protocol is only privately verifiable. The paper claims that the verification in this protocol can be performed by a trusted third-party, e.g. smart contract. Nevertheless, we argue that this is not the case. As,  the scheme requires a set of secrets (e.g. challenges) to perform the verification, while a smart contract does not maintain a private state. If the challenges are given to the contract, then the server can read the contract and compute all proofs in one go (which violates the security requirement set out in the paper). Moreover, the same security issue would arise in the case where the verification is delegated to a third-party auditor who may collude with the server (as the auditor can send all challenges to the server at once). Therefore, the only secure option, in the compact PoSt, is that  the verifier performs the validation itself (i.e. private verification/validation).  Note that   all the above outsourced PoR schemes \cite{armknecht2014outsourced,xu2016lightweight,Storage-Time}  assume the client behaves honestly towards the server. Otherwise, a malicious client can generate the tags in a way that  makes an honest server generate invalid proofs. 







Hence, the existing outsourced PoR protocols either suffer from several security issues and guarantee no real-time detection or have to fully trust verifiers with the verification correctness, or are very inefficient. 


 \subsubsection{Distributed PoR using a (Tailored) Blockchain}
 
 
Distributed  PoR allows  data to be distributed to numerous storage servers to achieve robustness, and address a single point of failure issue. Permacoin \cite{MillerPermacoin} is one of those  that distribute data among  customised blockchain nodes and  repurposes  mining resources of Bitcoin blockchain miners. In Permacoin, each miner needs to prove that it has a portion of the file and to do so it provides  proof of data retrievability verified by other miners. The miner, in this scheme needs to invest on both computation resources  and storage resources (to generate proof of retrievability). The protocol uses a Merkle tree built on top of file blocks to support publicly verifiable PoR. The mining procedure in this protocol is based on iterative hashing. In Permacoin, in each verification, the prover has to send both challenged file blocks and  the proof path in the tree, that imposes communication cost logarithmic to the size of the \emph{entire original file}.  In this scheme, an accepting proof only indicates a portion of the data holding by that miner (prover) is retrievable. Thus, for a data owner to have a guarantee that the entire data is retrievable, it  has to either wait for a long period of time (depending on the file size and the number of miners) or hope that there are enough active miners in each epoch.  Also, since the miners are both verifiers and provers (storage providers), it is assumed that the storage providers are economically rational (which is weaker than a malicious one). %Furthermore, the scheme suffers a vital security issue that allows a misbhaving miner to anticipate the challenge  file blocks for next rounds, if it is a miner for  consecutive blocks.  There are two protocols proposed in \cite{}, basic and advanced. The main difference between the two is that in the latter one the miner also signs a random index  and feeds the signature to the  procedure via which the next random index is created. For the sake of simplicity, we consider the basic protocol,  we will discus the same security issue is inherited by the advanced one as well. In particular, to compute a random index $r_{i}$, a miner computes $r_{i}=H(puz||pk||i||s)$ where $s$ is picked by the miner and $puz$ contains the header of previous block. Value $puz$ is the only source of randomness and would be unpredictable to the adversary if the previous block is proposed by an honest miner who includes a random $s$ in the block. However, if the adversary is chosen as a miner for $m$ consecutive blocks, then it can predict the random indices for every next $m-1$  blocks. Therefore, for a certain time period the POR (and Permacoin) security  is violated (e.g. can pre-compute the proof) and the availability of even the partial file is not guaranteed for a time period. Such issue does not hold in Bitcoin (with a high probability) as the randomness of each block is determined by a solution (nonce) of proof of work, not used in Permacoin. 





Filecoin and KopperCoin in \cite{Filecoin,KoppBK16} respectively, offer  similar features, i.e. repurposing Bitcoin and supporting distributed PoR.  However, Filecoin, in addition to a Merkle tree, uses  generic zero-knowledge proofs (i.e. zk-SNARKS) that result in a high  overall computation  cost and requires a trusted party to generate the system parameters. Filecoin uses proof of work as well as PoR. On the other hand, 
 KopperCoin uses BLS signature-based publicly verifiable PoR that has a constant communication cost but has a high computation cost. Unlike Filecoin, KopperCoin does not use a PoW. 

In the same line of research, \cite{KoppMHKB17} proposes a customised blockchain  that supports distributed PoR as well as  preserving the privacy of on-chain payments between storage users and providers. It however is computationally expensive as it uses publicly verifiable  tags based on  BLS signatures for PoR and ring signatures for the privacy-preserving payments.  Likewise, \cite{RujRBK18} proposes a high-level distributed PoR framework whose aim is efficient utilization of  storage resources offered by storage providers. It also uses BLS signatures and a Merkle tree along with a smart contract  for payments. Similarly, \cite{XueX0B18,XueXB19} propose  schemes that allow a user to store its encrypted data in the blockchain. The scheme uses a Merkle tree  for PoR   and a smart contract to transfer fees to the blockchain nodes storing  the data. In these two schemes, the data owner  is the party who sends random challenges  to  storage nodes, so it has to be  online when  verification is needed. Storj \cite{storj14} also falls in this category where there is a tailored blockchain comprising a set of storage nodes that store a part of data and provide  proofs when they are challenged. In Storj, there are trusted nodes, Satellite, who do the verifications on behalf of the clients.  The scheme uses a Merkle tree-based   PoR. 

So,  existing distributed PoR protocols have either a large proof size or  high verification cost. Also,  they do not guarantee  that the \emph{entire} file is retrievable in  \emph{real-time}. 



 \subsubsection{Verifying Remote (off-chain) data via a Blockchain}
 To relax the assumption that an auditor is fully trusted with the correctness of verification in the publicly verifiable PoR schemes while storing data off-chain, e.g. in a cloud, researchers proposed numerous protocols, e.g. \cite{RennerMK18,HaoXWJW18,ZhangDLZ18,Audita18,blockchain-data-audit-18,sia14}, that delegate the verification procedure  to blockchain nodes.  The high-level framework  proposed in \cite{RennerMK18} requires only a hash of the entire file is stored in a smart contract, where when later on the user access the file, it computes the hash of the file and compares it with the one stored in the contract. In this scheme, the entire file has to be accessed by the user for each verification which is what exactly PoR schemes avoid doing. The protocol  in \cite{HaoXWJW18} uses BLS signature-based tags and Merkle tree where the tags are broadcast to all blockchain nodes. However, surprisingly, the protocol assumes the cloud server is fully trusted and stores data safely, which raises the question that \textit{``why is a data verification needed in the first place?''}. In this protocol, the tags are generated by the cloud who sends them to the nodes. The tags are never checked against the outsourced data.  So, the only security guarantee \cite{HaoXWJW18} offers is the immutability of tags. The high-level scheme proposed in \cite{ZhangDLZ18} allows a client to pay the storage server in a fair manner. The scheme uses a blockchain for payment    and Merkle tree for PoR. In this protocol, the client has to be online and send random challenges to the server for every verification. 
 
 Audita  \cite{Audita18} uses an augmented blockchain and RSA signature-based  PDP  \cite{DBLP:conf/ccs/AtenieseBCHKPS07} to achieve its goal. In this scheme miners, for each epoch, pick a dealer who sends a challenge to the storage node(s) to get  proofs of data possession. Similar to Permacoin \cite{MillerPermacoin}, Audita substitutes proof of work with PDP, so if a proof is accepted a new block is added to the chain. But, in Audita every miner carries out  expensive public key based verifications.  The protocol proposed in \cite{blockchain-data-audit-18},   similar to \cite{armknecht2014outsourced}, uses a third-party auditor. It mainly uses, a smart contract and  BLS signature-based tags. In this protocol, unlike \cite{armknecht2014outsourced},  when the auditor raises a dispute during the verification it calls a smart contract who performs the verification again to detect a misbehaving party, i.e. the server or auditor.  The protocol is computationally more expensive than \cite{armknecht2014outsourced} and inherits the same  issues, i.e. issues (a-c) stated above.  Sia \cite{sia14} is a mechanism in which a data owner  distributes  its data among off-chain storage servers who periodically provide a PoR  to a smart contract signed between the data owner and the servers. Each server gets paid if its proof is accepted by the contract. In this scheme, a Merkle tree-based PoR is used. 
 
As evident,  the schemes designed for  blockchain-based verification of data stored in a storage server either require clients to access  whole outsourced data for  every verification, or impose a high communication/computation cost, or  clients have to be online for each verification. 

 \subsubsection{Fair Exchange of Digital Services}
 
 Campanelli \textit{et al.}  \cite{CampanelliGGN17} propose a scheme that allows different parties to exchange digital services (and goods) over Bitcoin blockchain. This scheme, for instance in PoR context, allows the storage provider to get paid  if and only if the data owner receives an accepting proof. It has two variants, publicly and  privately verifiable both of which use a smart contract.   In addition, in the privately verifiable variant,  MAC-based tags and generic secure multi-party computation are used, e.g. Yao's garbled circuit \cite{Yao82b}, while in the publicly verifiable one  BLS signature tags and zk-SNARK are used.  Nevertheless, this scheme assumes that either the client is  available and online when PoR is provided (in privately verifiable variant) or the third party, acting on a client's behalf, performs PoR verification honestly (in publicly verifiable one). 

%BonehBBF18,Wesolowski19












% !TEX root =main.tex
\vspace{-5mm}

\section {Discussion on   Time-lock Puzzle}\label{R-TLP-proof} 

%Below, we provide the security definition of time lock puzzle and proof of the RSA puzzle scheme proposed in \cite{Rivest:1996:TPT:888615}. Informally, a time-lock puzzle's security requires that the puzzle solution  remain hidden from all adversaries running in parallel within the time period, $\Delta$.  


%It is essential that no adversary can find a solution (in particular distinguishes two puzzles computed on  solutions provided by the adversary)  in  time $\delta(\Delta)<\Delta$, utilising  up to many  processors running in parallel and after a potentially large amount of pre-computation. In other words, it is critical to bound the adversary's allowed parallelism. So, such factors are explicitly incorporated  into the puzzle's definitions \cite{BonehBBF18,MalavoltaT19,garay2019}. 
%
%%Now we have given more power to the adversary in the case that it has polynomial number of processors running in parallel. In this case, we set an upper bound on the time after which a solution is found by this adversary: \delta(\Delta)<\Delta. This means if the adversary (who has many parallel processors) run in time shorter or equal to \delta(\Delta) it cannot find the solution. 
%
%
%\begin{definition}[Time-lock Puzzle Security] A time-lock puzzle is secure if for all $\lambda$ and $\Delta$, all probabilistic polynomial time adversaries $\mathcal{A}=(\mathcal{A}_{\scriptscriptstyle 1},\mathcal{A}_{\scriptscriptstyle 2})$ where $\mathcal{A}_{\scriptscriptstyle 1}$ runs in total time $O(poly(\Delta,\lambda))$ and $\mathcal{A}_{\scriptscriptstyle 2}$ runs in  time $\delta(\Delta)<\Delta$ using at most $\pi(\Delta)$ parallel processors, there exists a negligible function $\mu(.)$, such that: 
%
%$$ Pr\left[
%  \begin{array}{l}
%\mathcal{A}_{\scriptscriptstyle 2}(pk, \Theta,state)  \rightarrow b
%\end{array} \middle |
%    \begin{array}{l}
%\mathtt{R\text{-}TLP.Setup}(1^{\scriptscriptstyle\lambda},\Delta)\rightarrow (pk,sk)\\
%\mathcal{A}_{\scriptscriptstyle 1}(1^{\scriptscriptstyle\lambda},pk, \Delta)\rightarrow (s_{\scriptscriptstyle 0},s_{\scriptscriptstyle 1},state)\\
%b\stackrel{\scriptscriptstyle\$}\leftarrow \{0,1\}\\
%\mathtt {GenPuz}(s_{\scriptscriptstyle b}, pk, sk)\rightarrow \Theta\\
%\end{array}    \right]\leq \frac{1}{2}+\mu(\lambda)$$
%
%\end{definition}



% !TEX root =S-PoR.tex




In general, time-lock puzzles by definition are sequential functions.  In their construction, it is required a  function inherently sequential, e.g. modular squaring, is called iteratively for a certain number of times.  The notions of sequential function and iterated sequential functions, in the presence of an adversary possessing a polynomial number of processors, have been generalised and defined in \cite{BonehBBF18}. Below, for the sake of completeness, we provide those  definitions. 


\begin{definition}[$\Delta,\delta(\Delta))$-Sequential function]
For a function: $\delta(\Delta)$, time parameter: $\Delta$ and security parameter: $\lambda=O(\log(|X|))$,  $f:X\rightarrow Y$ is a $(\Delta,\delta(\Delta))$-sequential function if the following conditions hold:
\begin{itemize}
\item[$\bullet$] There exists an algorithm that for all $x\in X$evaluates $f$ in parallel time $\Delta$ using $poly(\log(\Delta),\lambda)$ processors.
\item[$\bullet$] For all adversaries $\mathcal{A}$ that run in parallel time strictly less than $\delta(\Delta)$ with $poly(\Delta,\lambda)$ processors: 
$$Pr\left[y_{\scriptscriptstyle A}=f(x)\middle |  y_{\scriptscriptstyle A}\stackrel{\scriptscriptstyle \$}\leftarrow \mathcal {A}(\lambda,x), x\stackrel{\scriptscriptstyle \$}\leftarrow X\right]\leq negl(\lambda)$$
where $\delta(\Delta)=(1-\epsilon)\Delta$ and $\epsilon<1$
\end{itemize}
\end{definition}

\begin{definition}[Iterated Sequential function] Let $g:X\rightarrow X$ be a $(\Delta,\delta(\Delta))$-sequential function. A function $f: \mathbb{N}\times X\rightarrow X$ defined as $f(k,x)=g^{\scriptscriptstyle (k)}(x)=\underbrace{g\circ g\circ... \circ g}_{\scriptscriptstyle k\times}$ is  an iterated sequential function, with round function $g$, if for all $k=2^{\scriptscriptstyle o(\lambda)}$ the function $h:X\rightarrow X$ defined by  $h(x)=f(k,x)$ is $(k\Delta,\delta(\Delta))$-sequential. 

\end{definition}

%\underbrace{}_{\scriptscriptstyle \xi+1}
The main property of an iterated sequential function is that iteration of the round function $g$, is the fastest way to evaluate the function. Iterated squaring  in a finite group of unknown order, is widely believed to be a candidate for an iterated sequential function. Its definition is  as follows. 




\begin{assumption}[Iterated Squaring]\label{assumption::SequentialSquaring} Let N be a  strong RSA modulus, $r$ be a generator of $\mathbb{Z}_{\scriptscriptstyle N}$,   $\Delta$ be a time parameter, and $T=poly(\Delta,\lambda)$. For  any $\mathcal{A}$, defined above,  there is a negligible function $\mu(.)$ such that: 

%$$Pr[b\leftarrow \mathcal{A}(N,g,\mathcal{T},x,y)]\leq \frac{1}{2}+\mu(\lambda)$$
$$ Pr\left[
  \begin{array}{l}
\mathcal{A}(N, r,y) \rightarrow b
\end{array} 
\middle |
    \begin{array}{l}
r \stackrel{\scriptscriptstyle \$}\leftarrow \mathbb{Z}_{\scriptscriptstyle N}, b\stackrel{\scriptscriptstyle \$}\leftarrow \{0,1\}\\
\text{if} \ \ b=0,\   y \stackrel{\scriptscriptstyle \$}\leftarrow \mathbb{Z}_{\scriptscriptstyle N} \\
\text {otherwise}\ y=r^{\scriptscriptstyle 2^{\scriptscriptstyle T}}
\end{array}    \right]\leq \frac{1}{2}+\mu(\lambda)$$

\end{assumption}


%\begin{equation}

%\end{equation}



\begin{theorem}[RSA-based TLP Security]\label{theorem::R-LTP-Sec}
Let $N$ be a  strong RSA modulus and $\Delta$ be the period within which the solution/secret remains hidden. If the sequential squaring assumption holds,  factoring $N$ is a hard problem and the symmetric key encryption is  semantically secure, then RSA-based TLP scheme (presented in Section \ref{Time-lock-Encryption})  is a secure time-lock puzzle.
% in the presence of an adversary whose run-time is upper bonded by $T=poly(\lambda,\Delta)$.
\end{theorem}



\begin{proof}[sketch] 
Let  a solver be $\mathcal{A}=(\mathcal{A}_{\scriptscriptstyle 1},\mathcal{A}_{\scriptscriptstyle 2})$, where $\mathcal{A}_{\scriptscriptstyle 1}$ runs in total time $O(poly(\Delta,\lambda))$ and $\mathcal{A}_{\scriptscriptstyle 2}$ runs in  time $\delta(\Delta)<\Delta$ using at most $\pi(\Delta)$ parallel processors. For  the solver to find  the secret significantly earlier than $\delta(\Delta)$, given the public parameters, it needs to either: (a) compute $\phi(N)$, so it can generate the blinding factor: $b$ as fast as it is done in the encryption phase, or (b) break the symmetric key encryption, or (c) extract $k$ from $o_{\scriptscriptstyle 2}$ by finding the blinding factor without performing a sufficient number of  squaring (and without knowing $\phi(N)$). However,  finding  $\phi(N)$ is as hard as factoring $N$, also as long as the encryption is secure and $k$ is sufficiently large  it cannot break the symmetric key encryption. Moreover,  the blinding factor is a uniformly random element of the ring (due to Assumption \ref{assumption::SequentialSquaring}), so it prevents the solver from finding the blinding factor without carrying out enough squaring within time significantly less than $\delta(\Delta)$. Thus, any  adversary $\mathcal{A}$ cannot find the secret without carrying out a sufficient number of squaring that would take it $\delta(\Delta)$ \hfill\(\Box\)
\end{proof}


\begin{remark}

In the original TLP protocol proposed \cite{Rivest:1996:TPT:888615}, it is implicitly assumed that the type of secret key $k$ is converted in step \ref{TLP::mask-k}. Because, in step \ref{TLP::pick-k} its type is byte string, while in step \ref{TLP::mask-k} its type is  integer of $\mathbb{Z}_{\scriptscriptstyle N}$. Such point plays a role when the protocol is implemented, in practice.


\end{remark}



% !TEX root =main.tex
\section {Traditional PoR Model}\label{PoR-Model}

In this section, we restate a  definition of the traditional PoR scheme \cite{DBLP:conf/asiacrypt/ShachamW08}. In general, a PoR scheme considers the case where an honest client wants to store its file(s) on a  potentially malicious server, i.e active adversary. It is a challenge-response interactive protocol where the server proves to the client that its file is intact and retrievable. A PoR scheme comprises five algorithms: 

\begin{itemize}
\item[$\bullet$] $\mathtt{Setup}(1^{\scriptscriptstyle\lambda})\rightarrow sk$:  a probabilistic algorithm, run by a client, that takes as input a security:  $1^{\scriptscriptstyle\lambda}$ and outputs a secret key.

\

\item[$\bullet$] $\mathtt{Store}(sk,F)\rightarrow (F^{*}, \sigma)$: a probabilistic algorithm, run by a client, that takes as input the secret key: $sk$ and a file: $F$. It encodes $F$, denoted by $F^{*}$ as well as generating a set of  tags: $\sigma$, where $F^{*}$ and  $\sigma$ is stored on the server.

\

\item[$\bullet$] $\mathtt{GenChal}(|F^{*}|, 1^{\scriptscriptstyle\lambda})\rightarrow \vv{\bm{c}}$: a probabilistic algorithm, run by a client, that takes as input the encoded file size: $|F^{*}|$ and security:  $1^{\scriptscriptstyle\lambda}$. It outputs a set of pairs, $\ddot{c}_{\scriptscriptstyle j}:(x_{\scriptscriptstyle j},y_{\scriptscriptstyle j})$, where each pair includes a file block index: $x_{\scriptscriptstyle j}$ and coefficient: $y_{\scriptscriptstyle j}$, both of them are picked uniformly at random.

\

\item[$\bullet$] $\mathtt{Prove}(F^{*}, \sigma, \vv{\bm{c}})\rightarrow \pi$:  takes the encoded file: $F^{*}$, (a subset of) tags: $\sigma$, and a vector of unpredictable random challenges: $\vv{\bm{c}}$ as inputs and outputs a proof of the file retrievability. It is run by a server.

\

\item[$\bullet$] $\mathtt{Verify}(sk,\vv{\bm{c}}, \pi)\rightarrow\{0,1\}$: takes the secret key: $sk$,  vector of random challenges: $\vv{\bm{c}}$, and the proof $\pi$ as inputs. It outputs either $0$ if it rejects,  or $1$ if it accepts the proof. It is run by a client.

\end{itemize}


Informally, a PoR scheme has two main properties: correctness and soundness. The correctness requires that for any key, all files, the verification algorithm accepts a proof generated by an honest verifier. The soundness requires that if  a prover convinces the verifier (with a high probability) then the file is  stored by the prover; This is formalized via the notion of an extractor algorithm, that is able to extract the file in interaction with the adversary using a polynomial number of  rounds. In contrast to the definition in \cite{DBLP:conf/asiacrypt/ShachamW08} where $\mathtt{GenChal}(.)$ is implicit, in the above we have explicitly defined  it, as its modified version  plays an integral role in SO-PoR definition (and protocol). 

% !TEX root =main.tex




\section {SO-PoR Model}\label{SO-PoR-Model}
In this section, we provide a formal definition  of SO-PoR. As previously stated, it builds upon the traditional PoR model \cite{DBLP:conf/asiacrypt/ShachamW08}, presented in Appendix \ref{PoR-Model}.  In  SO-PoR, unlike the traditional PoR, a client may not be available every time  verification is needed. Therefore, it wants to  delegate  a set of verifications that it cannot carry out itself. In this setting, it (in addition to file retrievability)  must have three guarantees: (a) \emph{verification correctness}: every verification is performed honestly, so  the client can rely on the verification result  without the need to re-do it, (b) \emph{real-time detection}: the client is notified in almost real-time when server's  proof is rejected, and (c) \emph{fair payment}: in every verification, the server is paid only if the server's  proof  is accepted. In SO-PoR, three parties are involved: an honest client, potentially malicious server  and a standard smart contract. SO-PoR also allows a client to perform the verification itself, analogous to the traditional  PoR, when it is available. 

%To satisfy the aforementioned requirements, and keep verifications' cost low, SO-PoR mainly utilises a smart contract (for verification and payment) and the chained time-lock puzzle to eventually release secret values used to: (a) generate challenges and (b) verify proofs. Therefore, i


\begin{definition}
A Smart Outsourced PoR (SO-PoR) scheme consists of seven algorithms ($\mathtt{Setup}, \mathtt{Store},$ $ \mathtt {SolvPuz}, $ $ \mathtt{GenChall}, \mathtt{Prove},$ $ \mathtt{Verify},  \mathtt{Pay}$) defined below: 


\
\begin{itemize}
\item[$\bullet$] $\mathtt{Setup}(1^{\scriptscriptstyle\lambda},\Delta, z)\rightarrow (\hat{sk},\hat{pk})$:  a probabilistic algorithm, run by a client.  It  takes as input a security: $1^{\scriptscriptstyle\lambda}$, time parameter: $\Delta$, and the number of verification delegated: $z$. It  outputs a set of  secret and public keys.

\

\item[$\bullet$] $\mathtt{Store}(\hat{sk},\hat{pk}, F,z)\rightarrow ({\bm{F}}, \sigma, \vv{\bm{o}},aux)$: a probabilistic algorithm, run only once by a client. It  takes as  input the secret key: $\hat{sk}$, public key: $\hat{pk}$, a file: $F$, and the number of verifications: $z$ that the client wants to delegate. It outputs an encoded file: ${\bm{F}}$,  a set of tags: $\sigma$, a set of $z$ puzzles: $\vv{\bm{o}}$, and public auxiliary data: $aux$. First three outputs are stored on the server and last output: $aux$, is   stored on a smart contract. 

\

\item[$\bullet$] $\mathtt {SolvPuz}(\hat{pk},\vv{\bm{o}})\rightarrow \vv{\bm{s}}$:  a deterministic algorithm that takes as input the public key: $\hat{pk}$ and puzzle vector: $\vv{\bm{o}}$.  It for each  $j\text{\small{-th}}$ verification outputs a  pair: $\ddot{s}_{\scriptscriptstyle j}:(v_{\scriptscriptstyle j},l_{\scriptscriptstyle j})$ of solutions, where $v_{\scriptscriptstyle j}$ and $l_{\scriptscriptstyle j}$ are outputted at time $t_{\scriptscriptstyle j}$ and $t'_{\scriptscriptstyle j}$ respectively and $t'_{\scriptscriptstyle j}> t_{\scriptscriptstyle j}$. Therefore, the algorithm in total outputs $z$ pairs. Value $l_{\scriptscriptstyle j}$ is sent  to the smart contract right after it is discovered. This algorithm is run  by the server.

\


\item[$\bullet$] $\mathtt{GenChall}(j,|{\bm{F}}|, 1^{\scriptscriptstyle\lambda},\ddot{s}_{\scriptscriptstyle j},aux)\rightarrow \vv{\bm{c}}$: a probabilistic algorithm that takes as input a verification index: $j$, the encoded file size: $|{\bm{F}}|$, security parameter: $1^{\scriptscriptstyle\lambda}$, first component of the related solution pair, $v_{\scriptscriptstyle j}\in \ddot{s}_{\scriptscriptstyle j}$, and public parameters: $pp\in aux$ containing  a blockchain and its parameters. It outputs pairs $\ddot{c}_{\scriptscriptstyle j} : (x_{\scriptscriptstyle j} , y_{\scriptscriptstyle j} )$, where each pair includes a pseudorandom  block's index:  $x_{\scriptscriptstyle j}$ and random coefficient: $y_{\scriptscriptstyle j}$. Also, values $x_{\scriptscriptstyle j}$ are derived from $v_{\scriptscriptstyle j}$ while $y_{\scriptscriptstyle j}$ are derived from $pp$. This algorithm is run by the server for each verification. 


%$pp$ is a public parameters for the beacon and it includes, blockchain, chain quality, and index. $\mathtt{GenCoeffs}()$ is called here

\

\item[$\bullet$] $\mathtt{Prove}(j,{\bm{F}}, \sigma,  \vv{\bm{c}})\rightarrow \pi$: a probabilistic algorithm that takes the verification index $j$, encoded file: ${\bm{F}}$ , (a subset of) tags: $\sigma$, and a vector of unpredictable challenges: $\vv{\bm{c}}$, as inputs and outputs a proof of  file retrievability. It is run by the server for each verification.

\

\item[$\bullet$] $\mathtt{Verify}(j,\pi,\ddot{s}_{\scriptscriptstyle j},aux)\rightarrow d:\{0,1\}$: a deterministic algorithm that takes the verification index $j$, proof: $\pi$,  second component of the related solution pair: $l_{\scriptscriptstyle j}\in \ddot{s}_{\scriptscriptstyle j}$, and public auxiliary data: $aux$.  If the proof is accepted, it outputs $d=1$; otherwise, outputs $d=0$. The default value of $d$ is $0$. This algorithm is run by the smart contract for each verification and invoked only once for each verification by only the server. 

\

\item[$\bullet$] $\mathtt{Pay}(j,d)\rightarrow d'=\{0,1\}$: a deterministic algorithm that takes the verification index $j$, the verification output: $d$. If $d=1$, it transfers $e$ amounts to the server and outputs $1$. Otherwise, it does not transfer anything, and outputs $0$. The default value of $d'$ is $0$. The algorithm is run by the  contract, and  invoked only by $\mathtt{Verify}(.)$. 
\end{itemize}
\end{definition}





%
%note that in the above, $\mathtt{Store}$ is a wrapper function that calls $\mathtt{GenPuz}(\vv{\bm{m}},\hat{sk},\hat{pk})$ and $\mathtt{Store}(\hat{sk},F)$ as subroutine,{\color{blue}xx explain what $\vv{\bm{m}}$ is for}
%




  
A SO-PoR scheme must satisfy two main properties: \emph{correctness} and \emph{soundness}. The correctness requires, for any: file, public-private key pairs, and puzzle solutions, both the verification  and pay algorithms, i.e. $\mathtt{Verify}(.)$ and $\mathtt{Pay}(.)$, output $1$ when interacting with  the  prover, verifier, and client  all of which are honest.  The soundness however is split into four properties: extractability, verification correctness, real-time detection, and fair payment, formally defined below.  Before we define the first property,  extractability, we provide the following  experiment between an environment: $\mathcal{E}$ and  adversary: $\mathcal{A}$ who corrupts $C\subsetneq\{\mathcal{S},\mathcal{M}_{\scriptscriptstyle 1},...,\mathcal{M}_{\scriptscriptstyle\beta}\}$, where $\beta$ is the maximum number of miners which can be corrupted in a secure blockchain. In this game, $\mathcal{A}$ plays the role of corrupt parties and $\mathcal{E}$ simulating an honest party's role. 


\begin{enumerate}
\item $\mathcal{E}$ executes $\mathtt{Setup}(.)$ algorithm and provides public key: $\hat{pk}$, to $\mathcal{A}$.   
\item $\mathcal{A}$ can pick  arbitrary file $F'$, and  uses it to make queries to  $\mathcal{E}$ to run:  $\mathtt{Store}(\hat{sk},\hat{pk},$ $ F',z)$ $\rightarrow (F'^{\scriptscriptstyle *}, \sigma, \vv{\bm{o}},aux)$  and return the output to $\mathcal{A}$. Also, upon receiving the output of $\mathtt{Store}()$, $\mathcal{A}$ can locally run  algorithms: $\mathtt {SolvPuz}(\hat{pk},\vv{\bm{o}})$ and   $\mathtt{GenChall}(j,$ $|F'^{\scriptscriptstyle *}|, $ $ 1^{\scriptscriptstyle\lambda},\ddot{s}_{\scriptscriptstyle j},aux)\rightarrow \vv{\bm{c}}$ as well as  $\mathtt{Prove}(j,F^{\scriptscriptstyle *}, \sigma, $ $ \vv{\bm{c}})\rightarrow \pi$,  to get their outputs as well. 
\item $\mathcal{A}$ can request $\mathcal{E}$ the execution of $\mathtt{Verify}(j,\pi,\ddot{s}_{\scriptscriptstyle j},aux)$ for any $F'$ used to query $\mathtt{Store}()$. Accordingly, $\mathcal{E}$ informs  $\mathcal{A}$ about the verification output. The adversary can send a polynomial number of queries to $\mathcal{E}$. Finally, $\mathcal{A}$ outputs the description of a prover: $\mathcal{A}'$ for any file it has already chosen above. 
\end{enumerate}

It is said a cheating prover: $\mathcal{A}'$ is $\epsilon$-admissible if it convincingly answers $\epsilon$ fraction of verification challenges \cite{DBLP:conf/asiacrypt/ShachamW08}. Informally, a SO-PoR scheme supports extractability, if there is an extractor algorithm: $\mathtt{Ext}(\hat{sk},\hat{pk},\mathtt{P}')$, that takes the secret-public keys and the description of the  machine implementing the prover's role: $\mathcal{A}'$ and outputs the file: $F'$. The extractor can reset the adversary to the beginning of the challenge phase and repeat this step polynomially many times for  of extraction, i.e. the extractor can rewind it.

\begin{definition}[$\epsilon$-extractable]\label{extractable} A SO-PoR scheme is $\epsilon$-extractable if  for every adversary: $\mathcal{A}$ who corrupts $C\subsetneq\{\mathcal{S},\mathcal{M}_{\scriptscriptstyle 1} $ $,..., \mathcal{M}_{\scriptscriptstyle\beta}\}$, plays the experiment above, and outputs an $\epsilon$-admissible cheating prover: $\mathcal{A}'$ for a file $F'$,  there exists an extraction algorithm that recovers $F'$ from $\mathcal{A}'$, given honest parties public-private keys and $\mathcal{A}'$,  i.e. $\mathtt{Ext}(\hat{sk},\hat{pk},\mathcal{A}')\rightarrow F'$, except with a negligible probability. 
\end{definition}

% . The extractor has the ability to reset the adversary to the beginning of the challenge phase and repeat this step polynomially many times for the purpose of extraction

In the above game, the environment, acting on honest parties' behalf, performs the verification correctly; which is not always the case in SO-PoR. As the verification can be run by miners a subset of which are potentially corrupted. Even in this case, the verification correctness must hold, e.g.  if a corrupt server sends an  invalid proof then even if $\beta-1$ miners are corrupt (and colluding with it) the verification function will not output $1$ and if the server is honest and submits a valid proof then the verification function does not output $0$ even if $\beta$ miners are corrupt, except with a negligible probability. This is formalised below. 


\begin{definition}[Verification Correctness]\label{Verification-Correctness} Let $\beta$ be the maximum number of miners that can be corrupted in a secure blockchain network and $\lambda'$ be the blockchain security parameter. Also, let $\mathcal{A}$ be the adversary who (plays the above game and) corrupts parties in either $C\subseteq\{\mathcal{S},\mathcal{M}_{\scriptscriptstyle 1},...,\mathcal{M}_{\scriptscriptstyle\beta-1}\}$ or $C'\subseteq\{\mathcal{M}_{\scriptscriptstyle 1},...,\mathcal{M}_{\scriptscriptstyle\beta}\}$.  In SO-PoR, we say the correctness of $j\text{\small-th}$ verification  is guaranteed if: 
 
$$\begin{array}{l}
\text{in the former case}: Pr[\mathtt{Verify}_{\scriptscriptstyle C}(j,\pi,\ddot{s}_{\scriptscriptstyle j},aux)=1]\leq \mu(\lambda')\\
\text{in the latter case}: Pr[\mathtt{Verify}_{\scriptscriptstyle C'}(j,\pi,\ddot{s}_{\scriptscriptstyle j},aux)=0]\leq \mu(\lambda')
\end{array}$$
where $\mu(.)$ is a negligible function. 
\end{definition}

Also, a client needs to have a guarantee that for each verification it can get a correct result within a (fixed) time period. 

\begin{definition}[$\Upsilon$-real-time Detection]\label{real-time Detection} Let $\mathcal{A}$, as defined above, be the adversary who corrupts either $C$ or $C'$.
A client, for each $j\text{\small{-th}}$ delegated verification, will get a correct output of  $\mathtt{Verify(.)}$, by  means of reading a blockchain, within time window $\Upsilon$, after the time when the server is supposed to send  its proof  to the blockchain network. Formally,

$$\mathtt{Read}(\Upsilon,\mathtt{Verify}_{\scriptscriptstyle D}(j,\pi,\ddot{s}_{\scriptscriptstyle j},aux))\rightarrow \{0,1\}$$
where $D\subsetneq\{C,C'\}$, except with a negligible probability. 
\end{definition}





\begin{definition}[Fair Payment]\label{Fair-Payment}  SO-PoR supports a fair payment if the client and server fairness are satisfied: 

\begin{itemize}
\item[$\bullet$] \textit{\textbf{Client Fairness}}: An honest client is guaranteed that it only pays ($e$ coins) if the server provides an accepting proof, except with a negligible probability. 
\item[$\bullet$]\textit{\textbf{Server Fairness}}: An honest server is guaranteed that the client gets a correct proof if the client pays ($e$ coins),   except with a negligible probability. 
\end{itemize}
Formally, let $\mathcal{A}$ be the adversary who corrupts either $C$ or $C'$, as defined above. To satisfy a fair payment:
\begin{equation}
Pr[\mathtt{Pay}_{\scriptscriptstyle D}(.)=b 	\cap  \mathtt{Verify}_{\scriptscriptstyle D}(.)=b]\geq 1-\mu(\lambda'),   
\end{equation}

the following inequality must hold:
\begin{equation}\label{inequ::fair-payment}
Pr[\mathtt{Pay}_{\scriptscriptstyle D}(.)=b' 	\cap \mathtt{Verify}_{\scriptscriptstyle D}(.)=b] \leq \mu(\lambda'),
\end{equation}
where $D\subsetneq\{C,C'\},b\neq b'$, and $b, b'\subsetneq\{0,1\}$









%\begin{equation}
%Pr[\mathtt{Pay}_{\scriptscriptstyle D}(.)=1 	\cap \mathtt{Verify}_{\scriptscriptstyle D}(.)=0] \leq \mu(\lambda')
%\end{equation}
%\begin{equation}
%Pr[ \mathtt{Pay}_{\scriptscriptstyle D}(.)=0 	\cap \mathtt{Verify}_{\scriptscriptstyle D}(.)=1] \leq \mu(\lambda')
%\end{equation}
%\begin{equation}
%Pr[\mathtt{Pay}_{\scriptscriptstyle D}(.)=1 	\cap  \mathtt{Verify}_{\scriptscriptstyle D}(.)=1]\geq 1-\mu(\lambda')
%\end{equation}
%where $D\subsetneq\{C,C'\}$
\end{definition}

The above definition also takes into account the fact that the client at the time of delegated verification is not necessarily available to make the payment itself, so the payment is delegated to a third party, e.g. a smart contract. In this case, the definition  ensures that even if  the client or/and server are honest, the third party cannot affect  the fairness (except with a negligible probability).

% Moreover, it is not hard to see, if the inequality \ref{} holds, then the fairness is guaranteed, with a high probability:  \begin{equation*}
%Pr[\mathtt{Pay}_{\scriptscriptstyle D}(.)=b 	\cap  \mathtt{Verify}_{\scriptscriptstyle D}(.)=b]\geq 1-\mu(\lambda')
%\end{equation*} 


%In the following we explain the rational behind the above definition. In SO-PoR scheme, for each verification, the server sells  a proof: $\pi$ to a client and earns  $e$ coins if and only if the proof is accepted, i.e. $\mathtt{Verify}(.)=1$. In SO-PoR setting, the client at the time of delegated verification is not necessarily online to make the payment itself, so it is done by a third party (e.g. a smart contract). The definition must ensure that   even if both server and client are honest, the third party cannot affect  the fairness. 


\begin{definition}[SO-PoR Security]\label{SO-PoR-Security} A SO-PoR scheme is secure if it is $\epsilon$-extractable, and satisfies verification correctness, $\Upsilon$-real-time detection, and fair payment properties.

\end{definition}




\begin{remark}
The folklore assumption is that (in a secure blockchain) a smart contract function \emph{always outputs a correct result}. However, this is not the case and it may fail under certain circumstances.  For instance, as shown in \cite{LuuTKS15} all rational  miners may not verify a certain transaction. As another example,  an adversary (although with a negligibly small probability)  discards a  certain honestly generated blocks,  reverses the state of blockchain and contract, or breaks a client's signature scheme.  Accordingly, in our definitions above, we take such cases  into consideration and allow the possibility that a function outputs an incorrect result even though with a negligibly small probability. 
\end{remark}




\begin{remark}
SO-PoR model differs from traditional (e.g. \cite{DBLP:conf/ccs/JuelsK07,DBLP:conf/asiacrypt/ShachamW08}) and outsourced PoR  (e.g. \cite{armknecht2014outsourced,xu2016lightweight}) models in several aspects. Only  the SO-PoR model offers all the properties. In particular,  traditional PoRs only offer extractability while outsourced ones  offer liability as well, that allows a client (by re-running all verifications function) to detect a verifier if it provides an incorrect verification output, so the client   cannot rely on the verification result provided.  As another difference, the SO-PoR model  takes into account the case where an adversary can corrupt both the server and some miners at the same time.
\end{remark}

\begin{remark}
SO-PoR should also support  the traditional PoR where only client and server interact with each other  (e.g. client generates challenges, and verifies proof) when the client is available. To let SO-PoR definition support that too, we can simply define a flag: $\xi$, in each function, such that  when $\xi=1$, it acts as the traditional PoR; otherwise (when $\xi=0$), it performs as a delegated one. For the sake of simplicity, we let the flag  be implicit in the definitions above, where the default  is  $\xi=1$ 
\end{remark}

% !TEX root =main.tex

\section{Further Remarks on SO-PoR Protocol}\label{SO-PoR-discussion}


\begin{remark}
In SO-PoR, for a security reason the server must record $j\text{\small{-th}}$ PoR proof in the contract before $l_{\scriptscriptstyle j}$ is recovered. Also,  the way disposable tags are generated in SO-PoR  differs from those computed  in previous PoR schemes, despite having similarities structure-wise. Moreover, with slight adjustments, we can reduce the contract-side storage cost to constant.  For more details, we refer readers to Appendix \ref{SO-PoR-discussion} which also explains why strawman approaches are not suitable substitutes for SO-PoR. 
\end{remark}


\begin{remark}
 In each verification, e.g. $j\text{\small{-th}}$ one, it is required that the server can: (a) learn the random challenges, (b) compute a proof, and (c) record it in the smart contract, before it is able to learn key $l_{\scriptscriptstyle j}$; otherwise, (i.e. if it learns $l_{\scriptscriptstyle j}$ before sending and registering the proof), it can tamper with the data and pass the verification. Because by knowing $l_{\scriptscriptstyle j}$ it can construct valid tags for the data that has been tampered with. That is why, in the protocol, it is required: $t'_{\scriptscriptstyle j}\geq t_{\scriptscriptstyle j}+\Delta_{\scriptscriptstyle 1}+\Delta_{\scriptscriptstyle 2}$. Also, C-TLP parameters (i.e. maximum power of the adversary and delay parameters) are public, so are time points/window at which PoR proofs should be provided in SO-PoR. Therefore, in SO-PoR the client can always set TLP parameters such that a puzzle is solved at a certain time point or within an appropriately sized window, so the server can provide each PoR proof to the contract who verifies it within a certain time. Moreover, since the parameters are public, the server can check them, before it agrees to serve the client.
\end{remark}


\begin{remark}
The way disposable tags are generated in SO-PoR  differs from those computed  in traditional/outsourced PoR schemes, in spite of having similarities structure-wise. Specifically, (unlike existing protocols, e.g. \cite{DBLP:conf/asiacrypt/ShachamW08,armknecht2014outsourced}) in SO-PoR, each random value, $r_{\scriptscriptstyle b,j}$, used to generate  a disposable tag of a  block (for $j\text{\small{-th}}$ verification), is not derived from the block index. Instead, it depends on (a  fresh secret key for $j\text{\small{-th}}$ verification and) the total number of blocks challenged in each verification that is a public value. This means, the verifier does not need to know and verify each challenged block's index in the verification phase, which leads to a lower cost.  
\end{remark}

\begin{remark}
With minor adjustments, we can reduce the smart contract storage cost from $O(z)$ to constant, $O(1)$ and offload the cost to the server. The idea is that the client after computing the commitmnet vector: $\vv{\bm{h}}=[h_{\scriptscriptstyle 1},...h_{\scriptscriptstyle z}]$,  in step \ref{Gen-Puzzles-}, it preserves the ordering of the elements (i.e. $h_{\scriptscriptstyle j}$ is associated with $j\text{\small{-th}}$ verification) and constructs a  Merkle tree  on top of them. It stores the tree and the vector on the server, and stores only the tree's  root: $R$, on the contract. In this case,  the server in step \ref{fully-recover-l} after recovering $\ddot{p}_{\scriptscriptstyle j}= (l_{\scriptscriptstyle j}, d_{\scriptscriptstyle j})$,  computes: $h_{\scriptscriptstyle j}=\mathtt{H}(l_{\scriptscriptstyle j}||d_{\scriptscriptstyle j})$, and sends a Merkle tree proof (that $h_{\scriptscriptstyle j}$ corresponds to  $R$) along with $\ddot{p}_{\scriptscriptstyle j}$ to the contract. In step \ref{check-hash}, the contract: (a) checks if $h_{\scriptscriptstyle j}=\mathtt{H}(l_{\scriptscriptstyle j}||d_{\scriptscriptstyle j})$, and  (b) verifies the Merkle tree proof.  The rest  remains unchanged.  As a result, the number of values stored in the contract is now $O(1)$. This adjustment comes with an added communication cost: $O(|h_{\scriptscriptstyle j}|\log z)$ for each verification. Nevertheless, the added cost is small and independent of the file size.   For instance, when  $z=10^{\scriptscriptstyle 6}$ and $|h_{\scriptscriptstyle j}|=256$, the  added communication cost is only about $5.1$ kilobit.

\end{remark}

\begin{remark}
One might be willing to use a combination of existing publicly verifiable PoR and smart contract, such that the contract performs the verification on the client's behalf. However, this approach would have a higher computation or communication cost (especially in the verification phase) than our protocol. Specifically,  there exist two publicly verifiable PoR schemes, based on either (a) BLS signatures, e.g. \cite{DBLP:conf/asiacrypt/ShachamW08}, or (b) Merkle tree, e.g. \cite{MillerPermacoin}. The former approach, in total, requires  $zc$ exponentiations in the verification phase, whereas SO-PoR requires no exponentiations in this phase. Also, the BLS signature-based scheme takes a very long time to encode  a file, as the required number of exponentiations is  $O(|{\bm{F}}|)$. For instance, as measured in \cite{armknecht2014outsourced}, it takes about $55$ minutes to encode a  $64$-MB file,  meaning  it would take about $14$ hours to encode a $1$-GB file. But, in SO-PoR the number of exponentiations, in the store phase, is independent of and much fewer than the file size. On the other hand, the proof size in the Merkle tree-based approach is logarithmic with the file size, i.e. $256zc\log |{\bm{F}}|$ bits, that leads to a high server-side's communication cost. By contrast,  the proof size in SO-PoR is  independent of the file size and is much shorter, i.e. $884z$ bits. 


\end{remark}


% !TEX root =main.tex


\section{SO-PoR Security Proof}\label{SO-PoR-Security-Proof}
In this section, we first restate the main security theorem of SO-PoR protocol (presented in Section \ref{SO-PoR-Protocol}) and then prove it.  


\


\noindent\textbf{Theorem \ref{PoR-main-theorem}.} \textit{SO-PoR protocol is secure (according to Definition \ref{SO-PoR-Security}) if the tags/MAC's are unforgeable, $\mathtt{PRF}(.)$ is a secure pseudorandom function, the blockchain is secure, C-TLP protocol is secure, and $\mathtt{H}( \mathcal {B}_{\scriptscriptstyle \gamma}||...||  \mathcal {B}_{\scriptscriptstyle \zeta})$ outputs an unpredictable random value (where $\zeta-\gamma$ is a security parameter).}


\begin{proof}[sketch]  In the following, we prove that SO-PoR protocol satisfies every  property that were defined in Appendix \ref{SO-PoR-Model}.   

\

\noindent\textbf{\textit{Verification correctness}} (according to Definition \ref{Verification-Correctness}). We first argue that the adversary who corrupts either $C\subseteq\{\mathcal{M}_{\scriptscriptstyle 1},...,\mathcal{M}_{\scriptscriptstyle\beta}\}$ or $C'\subseteq\{\mathcal{S},\mathcal{M}_{\scriptscriptstyle 1},...,\mathcal{M}_{\scriptscriptstyle\beta-1}\}$ with a high probability, cannot influence the output of $\mathtt{Verify}(.)$ performed by a smart contract in a blockchain; in other words, the verification correctness holds. In short, the verification correctness boils down to the security of the underlying blockchain. In the case where the adversary corrupts $C$ (when the server provides an accepting proof),  for the adversary to make the verification function output $0$, it has to: (a) either forge the server's signature, or (b)  fork the blockchain so the chain  comprising the accepting proof is discarded. In case (a),  if it manages to forge the signature, it can generate a transaction that includes   a rejecting proof where the transaction is signed on the server's behalf. In this case, it  can broadcast the transaction as soon as the transaction containing an accepting proof is broadcast,  to make the latter transaction stale. Nevertheless, the probability of such  an event is negligible, $\epsilon(\lambda')$, as long as the signature is secure. Moreover, due to the liveness property of blockchain, an honestly generated transaction will eventually appear on an honest miner's chain \cite{DBLP:conf/crypto/GarayKL17}. In case (b),  the adversary has to generate long enough (valid) chain that excludes the accepting proof, but this also has a negligible success probability, $\epsilon(\lambda')$, under the assumption that the hash power of the adversary is lower than those of  honest miners (i.e. under the honest majority assumption) and due to the liveness property. Now we turn our attention to the case where the adversary corrupts $C'$ (when the server provides a rejecting proof). In this case, for the adversary to make the verification function to output $1$, the honest miners must not validate the transaction that contains the proof. Nevertheless, as long as the blockchain is secure and  the computational advantage of skipping transaction validation is low, i.e. the validation imposes a low computation cost, the miners check   the transaction's validation \cite{LuuTKS15}. Also, as shown in \cite{LuuTKS15} when a transaction validation imposes a high computation cost, two generic techniques can be used to support exact or probabilistic correctness (of a smart contract function output). We conclude the correctness of $\mathtt{Verify}(.)$ output  is guaranteed with a high probability.

\

\noindent\textbf{\textit{$\epsilon$-extractable}} (according to Definition \ref{extractable}). In the following, we show that if   a proof produced by an adversary: $\mathcal{A}$ who corrupts $C'\subseteq\{\mathcal{S},\mathcal{M}_{\scriptscriptstyle 1},...,\mathcal{M}_{\scriptscriptstyle\beta-1}\}$ is accepted by $\mathtt{Verify(.)}$ with probability at least $\epsilon$, then the file can be extracted by a means of an extraction algorithm. As mentioned  before, $\mathtt{Verify(.)}$ can use both disposable and permanent tags, in the latter case $\mathtt{Verify(.)}$ is run by a smart contract, while in the former one the client runs it. For the sake of simplicity, we first consider the case where $C'=\mathcal{S}$. In this case, the extractability proof is similar to the one in \cite{DBLP:conf/asiacrypt/ShachamW08},  with a few differences, in SO-PoR: (a) the extractor can use both disposable tags and  permanent tags (when the former run out), (b) assumes  C-TLP protocol is secure, (c) assumes $\mathtt{H}( \mathcal {B}_{\scriptscriptstyle \gamma}||...||  \mathcal {B}_{\scriptscriptstyle \zeta})$ outputs a random value  even if $\beta$ miners try to influence its output. Note that in \cite{DBLP:conf/asiacrypt/ShachamW08} only the permanent tags are used, and since  the client generates the challenges and performs the verification, it does not use other primitives; hence, it does not require other security assumptions. As proven in  Theorems \ref{Solution-Privacy} and \ref{Solution-Validity}, C-TLP protocol is secure. Moreover, as  analysed and proven in \cite{DBLP:journals/iacr/AbadiCKZ19,armknecht2014outsourced}, an output of $\mathtt{H}( \mathcal {B}_{\scriptscriptstyle \gamma}||...||  \mathcal {B}_{\scriptscriptstyle \zeta})$ is random value even if some blocks are generated or selectively disseminated by malicious miners. We conclude that the extractor can extract the file when $C'=\mathcal{S}$, and  the extractor is interacting a  $\epsilon$-admissible prover. Now we move on to the case where $C'\subseteq\{\mathcal{S},\mathcal{M}_{\scriptscriptstyle 1},...,\mathcal{M}_{\scriptscriptstyle\beta-1}\}$. In this case, the proof (provided by $\mathcal{S}$) is rejecting but the corrupt miners may try to make $\mathtt{Verify}(.)$ output $1$. Note,  if they succeed to do so, then the file can be  extracted only by using the permanent tags but not the disposable ones. However, as shown above (i.e. due to the verification correctness), they have a negligible probability of success, when the blockchain is secure. 

\

\noindent\textbf{\textit{$\Upsilon$-real-time Detection}} (according to Definition \ref{real-time Detection}). In the following, we argue that after the server broadcasts a proof at a certain time, say $t$, to the network, the client can get a correct output of $\mathtt{Verify}(.)$ at most after time period $\Upsilon$, by a means of  reading the blockchain. The proof  is split into two parts: (a) correctness of $\mathtt{Verify}(.)$ output, and (b) the maximum delay on the client's view of the output. Since we have already shown above that (when $C$ or $C'$ is corrupt) the correctness of $\mathtt{Verify}(.)$  output is guaranteed,  we focus on the latter property, i.e. the delay. To describe the delay, we need to recall two blockchain notions: \emph{liveness} and \emph{slackness} \cite{DBLP:conf/crypto/GarayKL17,DBLP:conf/crypto/BadertscherMTZ17}. Informally, liveness states that an honestly generated transaction  will eventually be included more than $\mathtt{k}$ blocks deep in an honest party's  blockchain \cite{DBLP:conf/crypto/GarayKL17}. It is parameterised by wait time: $\mathtt{u}$ and depth: $\mathtt{k}$. We can fix the parameters as follows. We set $\mathtt{k}$ as the minimum depth of a block  considered as the blockchain's \emph{state}  (i.e. a part of the blockchain that remains unchanged with a high probability, e.g. $\mathtt{k}\geq 6$) and $\mathtt{u}$ the waiting time  that the transaction gets $\mathtt{k}$ blocks deep. As shown in \cite{DBLP:conf/crypto/BadertscherMTZ17}, there is a slackness on  honest parties' view of the blockchain.  In particular, there is no guarantee that at any given time, all honest miners have the same view of the blockchain, or even the  state. But, there is an upper-bound on the slackness,  denoted by $\mathtt{WindowSize}$, after which all honest parties would have the same view on a certain part of the blockchain state. This means when an honest party (e.g. the server) propagates its transaction (containing the proof) all honest parties will see it on their chain after at most: $\Upsilon=\mathtt{WindowSize}+\mathtt{u}$ time period. So when  the adversary corrupts $C$ or $C'$, but in the latter case the server constructs a valid transaction (regardless of the proof status) the client by reading the blockchain (i.e. probing the miners) can get a correct result after at most time period $\Upsilon$ when the server sends the proof. Also, when parties in $C'$ are corrupt and the transaction (containing the proof) is not valid,  as discussed above  (for verification correctness) the honest miners would detect the invalid transaction and do not include in the chain, therefore the output of $\mathtt{Verify(.)}$ would be the same as its default value: $0$; the same holds  when the server sends nothing to the network. This concludes the proof related to $\Upsilon$-real-time detection in SO-PoR protocol.


\

\noindent\textbf{\textit{Fair Payment}} (according to Definition \ref{Fair-Payment}).  The proof  takes into consideration that the correctness of an output of $\mathtt{Verify}(.)$  is guaranteed (as shown above). It boils down to the correctness of $\mathtt{Pay}(.)$ as it is interrelated to the output of $\mathtt{Verify}(.)$. In particular, for the adversary to make  inequality \ref{inequ::fair-payment} not  hold, it has to break $\mathtt{Pay}(.)$ correctness, e.g. pays the server despite the verification function outputs $0$. Therefore, it would suffice to show the adversary who corrupts $C$ or $C'$ cannot affect $\mathtt{Pay}(.)$ output's correctness. To do so, we can apply the same argument used to prove  the correctness of $\mathtt{Verify}(.)$ above, as the correctness of  $\mathtt{Pay}(.)$ relies on the security of the blockchain as well.  \hfill\(\Box\)
\end{proof}

% !TEX root =main.tex


\section{SO-PoR's Full Evaluation}\label{Full-Evaluation}

In the following, we  provide a full analysis of SO-PoR and  compare its properties and detailed costs to those protocols that support outsourced PoR (O-PoR), i.e. \cite{armknecht2014outsourced,xu2016lightweight,Storage-Time}. 
Note, there are two protocols  proposed in  \cite{Storage-Time}; in the following, we only consider the one that supports public verifiability, i.e. basic PoSt. Recall,  we consider a generic case where a client outsources $z$ verifications and  in our cost analysis, we also compare SO-PoR cost with the cost of the most efficient privately verifiable PoR \cite{DBLP:conf/asiacrypt/ShachamW08}, too. 

%The basic PoSt in \cite{Storage-Time} supports public verification; that means a smart cotnract can play the role of a validator. 


\noindent\textbf{\textit{Properties}}. We start with a crucial feature that any O-PoR must have: real-time detection. Recall,  real-time detection requires a client to receive a correct verification result in (almost)  real-time without the need for it to re-execute the verification itself. This is offered only by SO-PoR. By contrast, in \cite{armknecht2014outsourced}  the auditor may never notify the  client, even if it does, its notification would not be reliable, and the client has to redo the verification to verify the auditor's claim. Similarly, in \cite{xu2016lightweight} the client has to fully trust the auditor to get notified on-time. The basic PoSt in \cite{Storage-Time} requires the server to collect all PoR's and send them to a validator in one go. This means, the client and validator cannot detect data tampering in real-time if a subset of the PoR's are invalid; instead they need to wait until $z$ PoR's are collected by the server.  So, \cite{armknecht2014outsourced,xu2016lightweight,Storage-Time} are not suitable for the cases where a client must be notified by a potentially malicious auditor as soon as an unauthorised modification on the sensitive data is detected.  The  fair payment is another vital property in O-PoR, as the cloud server  and auditor must be paid fairly, in the \emph{real world} when they serve a client. This feature is explicitly captured by  only SO-PoR.  The protocols in \cite{armknecht2014outsourced,xu2016lightweight} do not have any mechanism in place. In \cite{armknecht2014outsourced}, one may allow  the auditor to pay the server on the client's behalf. But, this is problematic.   The server and auditor can collude to save costs, in a way that the server  generates accepting proofs for the client but generates no proof for the auditor, and still the auditor pays it. This violates the fair payment and cannot be detected by the client unless it performs all the verification itself. On the other hand, a client in \cite{xu2016lightweight}  has to fully trust the auditor (with the payment too), otherwise the auditor can collude with the server to violate the fair payment. The authors of \cite{Storage-Time}  briefly state that the basic PoSt's verification can be performed by a smart contract who, after ensuring the proofs are valid,  pays the server. Nevertheless, as stated previously, the verification cost of this protocol is too high for a smart contract, i.e. imposes at least $z$ modular exponentiations over RSA modulus and requires a high number of messages logarithmic with the file size to be sent to the contract. Thus, even though it can support fair payment \emph{in theory}, it is very costly in practice.  Another  property is the cost of onboarding a new verifier, as it determines how flexible the client can be, to pick a new auditor when its current one is misbehaving. This cost in \cite{armknecht2014outsourced} is significantly  high, as it requires the verifier to download the entire file, generate metadata, and prove in zero-knowledge the correctness of metadata to the client. But, that cost in SO-PoR and \cite{xu2016lightweight,Storage-Time} is very low as the client only sends them a small set of parameters without the need to access the outsourced data.  Furthermore, as stated above, a client in  \cite{xu2016lightweight} has to fully trust the auditor with the correctness of verification but this is not the case in   SO-PoR and \cite{armknecht2014outsourced,Storage-Time}, as they consider a potentially malicious auditor (under different assumptions).  Table \ref{table::O-PoR-Property} summaries the result of  the comparison between the four protocols' main properties. 

% !TEX root =main.tex


 \begin{table*}[htb]
\begin{footnotesize}
\begin{center}
\caption{ \small{O-PoR Property Comparison. $\checkmark^{*}$ indicates the property is met only in theory.}} \label{table::O-PoR-Property} 
\renewcommand{\arraystretch}{.9}
\scalebox{0.98}{
\begin{tabular}{|c|c|c|c|c|c|c|c|c|c|c|c|c|c|} 
   \hline
\cellcolor[gray]{0.9}&
 \multicolumn{3}{c|}{\cellcolor[gray]{0.9}\scriptsize  \underline{\ \ \ \ \ \ \ \ \ \ \ \ \ \ \ \ \ \ \ \ \ \ \ \ \ \ \ \ \ \ \ \ \ \ \ \ \ \ \ \ \ \ \ \ \ \   \ \ \ \ \ Properties   \ \ \ \ \ \ \ \ \ \ \ \ \ \ \ \ \ \ \ \ \ \ \ \ \ \ \ \ \ \ \ \ \ \ \ \ \ \ \ \ \ \ \ \   \ \ \ }}\\
 %\cline{2-5}
 \cellcolor[gray]{0.9}\multirow{-2}{*} {\scriptsize Protocols}&\cellcolor[gray]{0.9}\scriptsize Real-time Detection&\cellcolor[gray]{0.9}\scriptsize Fair Payment&\cellcolor[gray]{0.9}\scriptsize Untrusted Auditor\\
\hline
 \cellcolor[gray]{0.9}{\scriptsize  SO-PoR }&\scriptsize$\checkmark$&\scriptsize$\checkmark$&\scriptsize$\checkmark$\\
    
     \cline{2-4}    
     \hline 
     
          \hline 
  \cellcolor[gray]{0.9}{\scriptsize   \cite{armknecht2014outsourced}}&\multirow{2}{*}{\rotatebox[origin=c]{0}{\scriptsize }} \scriptsize $\times$&\scriptsize$\times$&\scriptsize$\checkmark$\\
     \cline{2-4}
      
      \hline
      
       \hline
      
\cellcolor[gray]{0.9}{\scriptsize   \cite{xu2016lightweight}}&\multirow{2}{*}{\rotatebox[origin=c]{0}{\  \scriptsize }} \scriptsize$\times$&\scriptsize$\times$&\scriptsize$\times$\\
     \cline{2-4}


%%%%%%%%%%%%%%%%%%%%%%%
      \hline
      
       \hline
      
\cellcolor[gray]{0.9}{\scriptsize   \cite{Storage-Time}}&\multirow{2}{*}{\rotatebox[origin=c]{0}{\  \scriptsize }} \scriptsize$\times$&\scriptsize$\checkmark^{*}$&\scriptsize$\checkmark^{*}$\\
     \cline{2-4}


%%%%%%%%%%%%%%%%%%%%%%
 \hline
\end{tabular}
}
\end{center}
\end{footnotesize}
\end{table*}


 \noindent\textbf{\textit{Computation Complexity}}. In our analysis, we do not take into account the cost of erasure-coding a file, as it is identical in all schemes. We first analyse the computation cost of  SO-PoR.  A client in step \ref{gen-client-server-tags} performs $n$  multiplications and $n$ additions to generate permanent tags. In step \ref{Gen-Disposable-Tags}, it performs $cz$ multiplications and $cz$ additions to generate disposable tags for $z$ verifications. The client in step \ref{Gen-Puzzles-} invokes $\mathtt{GenPuz(.)}$ function, in C-TLP, that  costs $O(z)$. So, the client's total computation cost of preparing and storing a file is $O(n+cz)$. Now we consider the cloud server's cost that can be categorised into two classes: (a) solving a puzzle: $\mathtt{SolvPuz}()$, run only once, and (b) generating PoR run for each verification. In particular, the cloud in step \ref{Solve-Puzzle-Regen-Indices},  invokes $\mathtt{SolvPuz}()$, in C-TLP, that costs $O(T z)$ this includes the cost in step \ref{fully-recover-l} as well. To compute proofs, in step \ref{Gen-PoR}, it performs $2 c z$ multiplications and  $2 c z$ additions. So, the server to generate proof   is $O(cz)$. Next, we analyse the cost of the smart contract. In step \ref{check-hash}, it invokes $\mathtt{Verify}(.)$, in C-TLP, that in total costs $O(z)$, this involves invoking $z$  hash function's instances. Also, the contract in step \ref{verify-PoR} performs $z(1+c)$ and $z(1+c)$ modular multiplications and additions respectively. Thus, the total cost  is $O(cz)$ involving  mainly modular additions and multiplications.
%\cite{armknecht2014outsourced,xu2016lightweight}.

Now we analyse the computation cost of \cite{armknecht2014outsourced}. To prepare file tags, a client performs: $n$ multiplications and $n$ additions. Also, to verify tags generated by the auditor, the client  has to engage in a zero-knowledge protocol that requires it to carry out   $6n$ exponentiations and $2 n$ multiplications. Therefore, the client's computation complexity is $O(n)$. Also, the auditor in total performs $3n$ multiplications, $3 n$ additions and $3 n$ exponentiations to prepare file's metadata, so its complexity at this phase is $O(n)$. For the cloud to generate $z$ proofs, in total it performs $2z(c+c')$ multiplications and the same number of additions, where $c'$ is the number of challenges sent by the auditor to the cloud (on client's behalf) and $c>c'$, e.g. $c'=(0.1)c$. So, the total complexity of the cloud is $O(z  (c+c'))$. Next, we consider the verification cost. The auditor performs $z(1+c) $ multiplications and $z(1+c) $ additions to verify PoR. It also  performs $2cz $ additions in \textit{CheckLog} algorithm that requires the client to perform $z(2c+1)$ additions and $cz$ multiplications. Nevertheless, as discussed in Section \ref{Related-Work}, running only \textit{CheckLog} does not allow the client to detect a misbehaving auditor. Thus, it has to run \textit{ProveLog} too, that requires the client to perform $cz$ multiplications and $cz$ additions and requires the auditor to reveal all its secrets to the client. So, the total verification complexity is $O(cz)$. Now we turn our attention to the computation complexity of \cite{xu2016lightweight}. To prepare metadata, the client needs to perform $2  n$ multiplication and $2  n$ additions, so the client's complexity is $O(n)$. For the cloud to generate a proof it needs to perform $3  z$ exponentiations,  $z  (3  c+6)$ multiplications  and $z  (3 c+2)$ additions. So, its complexity is $O(cz)$. On the other hand, the verifier performs $cz$ exponentiations to compute  challenges. To verify the proof, it carries out  $6 z$ exponentiations, $cz$ multiplications,    $cz$ additions, and  $7  z$ pairings. Therefore, the verifier complexity is  $O(cz)$ dominated by expensive exponentiations and pairing operations. Also, we analyse the  computation cost of efficient privately verifiable PoR  in \cite{DBLP:conf/asiacrypt/ShachamW08}. A client in the store phase performs $n$ multiplications and $n$ additions to construct the tags. So, its complexity is $O(n)$. A server performs $2cz$ multiplications and $2cz$ additions to generate proofs, so in total $4cz$ or $O(cz)$ modular operations for $z$ verifications it carries out. The client, as verifier this time, performs in total $2z(1+c)$ or $O(z(1+c))$ modular operations.  Next, we turn our attention to  \cite{Storage-Time}, and evaluate the cost of the protocol that supports public verifiability, basic PoSt. As stated by the authors, they add a VDF to  the PoR construction in \cite{Filecoin} which uses a Merkle tree-based PoR. Since,  this PoR scheme only involves invocations of hash function, here we only focus on VDF cost as it dominates other computation costs. We assume that the most efficient    publicly verifiable delay function VDF \cite{Wesolowski19} is used.  In the setup, the client constructs a Merkle tree on the entire data  by invoking a hash function many times and also generates a random challenge. We ignore these costs as they are dominated by VDF's costs. To generate $z$ PoR's, the server invokes VDF $z$ times that in total runs in time period $T$. This involves $3Tz$ modular exponentiations over $\mathbb{Z}_{\scriptscriptstyle N}$ (where $N$ is a RSA modulus), and $Tz$ modular multiplications. Therefore, its complexity is $O(Tz)$. For a validator to ckeck the proofs output by VDF, it performs $3z$ modular exponentiations over $\mathbb{Z}^{*}_{\scriptscriptstyle N}$ (or $\bmod\phi(N)$). So the verifier's complexity is $O(z)$. 



Now we compare the protocols above. The verification in SO-PoR is much faster than the other three protocols; firstly, it requires no exponentiations in this phase, whereas \cite{xu2016lightweight,Storage-Time} do, and secondly, it requires $\frac{9c+3}{2(1+c)}$ times fewer computation than \cite{armknecht2014outsourced};  Specifically, when $c=460$,  SO-PoR verification requires about $4.5$ times fewer computation than the verification in \cite{armknecht2014outsourced} needs. In SO-PoR, the cloud server needs to perform $Tz$ exponentiations to solve puzzles, however this is independent of the file size. The sever in \cite{Storage-Time} also performs $3Tz$ modular exponentiations, which is  $3$ times higher than the number of exponentiations done by the server in SO-PoR. The  protocols in \cite{armknecht2014outsourced,DBLP:conf/asiacrypt/ShachamW08} do not include the puzzle-solving procedure (and they do not offer all features that SO-PoR does).  Furthermore, the proving cost in SO-PoR is similar to that of in \cite{armknecht2014outsourced}, and is  much better than \cite{xu2016lightweight}, as the latter one requires both exponentiations and pairing operations while the prove algorithm in SO-PoR  does not involve any exponentiations. The  proving cost in \cite{Storage-Time} is the lowest, as it requires only invocations of a hash function.  Also, the store phase in SO-PoR has a much lower computation cost than the one in \cite{armknecht2014outsourced}. The reason is that the number of exponentiations required (in this phase) in SO-PoR is independent of file size and is only linear with the number of delegated verifications; however, the number of exponentiations in \cite{armknecht2014outsourced} is linear with the file size. For instance, when $||{\bm{F}}||=1$-GB, the  total number of blocks is:   $n=\frac{1-\text{GB}}{128-\text{bit}}=625\times 10^{\scriptscriptstyle 5}$. Since the number of exponentiations in \cite{armknecht2014outsourced} is linear with the number of blocks, i.e. $9n$, the total number of exponentiations imposed by store algorithm is: $5625\times 10^{\scriptscriptstyle 5}$ which is very high. This is the reason why in the experiment in \cite{armknecht2014outsourced} only a small file size: $64$-MB, is used, that can be stored locally without the need to use  cloud storage, in the first place.   Now, we turn our attention to SO-PoR. Let the verification be done every month for a $10$-year period, in this case,  $z=120$.  So, the total number of exponentiations required by the store in SO-PoR is $121$. This means the store phase in SO-PoR requires over $46\times 10^{\scriptscriptstyle 5}$ times  fewer exponentiations than the one in \cite{armknecht2014outsourced} needs. On the other hand, the store algorithm in \cite{xu2016lightweight} does not involve any exponentiations; however, its  number of modular additions and multiplication  is higher than the ones imposed by the store in SO-PoR. The  store cost in \cite{Storage-Time} is the lowest, as it requires only invocations of a hash function. Furthermore, the verification and prove cost of SO-PoR and privately verifiable PoR \cite{DBLP:conf/asiacrypt/ShachamW08} are identical. 




%\noindent\textbf{\textit{I/O Cost}}. In SO-PoR and \cite{xu2016lightweight,armknecht2014outsourced}, for a server to generate a PoR, it only needs to access a constant number of blocks. Therefore, their total I/O cost is $O(z)$. But, unlike the majority of existing PoR schemes, in \cite{Storage-Time}  the server has to access the \emph{entire file blocks} to generate a PoR, so its cost is much higher than the rest. In particular, its total I/O cost is $O(|F|z)$. Note that I/O cost plays a crucial role in the scalability of the server and its ability to serve multiple clients/queries degrades when the I/O cost is significantly high. 




%Note, the total number of exponentiations required to prepare a file is $9\cdot n$. 

%Therefore, after running  \textit{ProveLog}, the auditor (who may not trust the client) has to   run again store algorithm which requires: (a) $9\cdot n$ exponentiations, and (b) downloading the entire file. 
 
  
% 
%  \noindent\textbf{\textit{I/O Cost}}. 
%  
%In SO-PoR and�\cite{armknecht2014outsourced,DBLP:conf/asiacrypt/ShachamW08}, for a server to generate a PoR, it only needs to access a constant number of blocks.
%  
%  
%  Therefore, their I/O cost in total is $O(z)$. However, in \cite{Storage-Time} the server has to access the entire file blocks to generate a proof, therefore its total I/O complexity is much higher, i.e. $O(|F|z)$.
% 
% 
 
 
 \
 
  \noindent\textbf{\textit{Communication Complexity}}. In our analysis, we do not take into account the communication cost of uploading an encoded file, i.e. $||{\bm{F}}||$, when the client for the first time sends it to the cloud, as it is identical in all schemes. The communication cost of SO-PoR is as follows. The client, in step \ref{Outsource-File}, sends $n$ permanent tags, $z c$ disposable tags, and the output of $\mathtt{GenPuz}(.)$ to the server and contract, where each (permanent/disposable) tag: $\sigma_{\scriptscriptstyle j}\in \mathbb{F}_p$ and $|\sigma_{\scriptscriptstyle j}|=128$-bit. Note the client also sends a few public parameters: $\hat{pk}$, whose size is short.  Therefore, the client's bandwidth  is: $128  (n+ c  z+19 z)$ bits, while its communication complexity is $O(n+c z)$. The cloud in step \ref{Register-Proofs}, sends $z$ pairs $(\mu_{\scriptscriptstyle j},\xi_{\scriptscriptstyle j})$, where $\mu_{\scriptscriptstyle j},\xi_{\scriptscriptstyle j}\in \mathbb{F}_p$ and $|\mu_{\scriptscriptstyle j}|=|\xi_{\scriptscriptstyle j}|=128$-bit. Also, in step \ref{fully-recover-l}, it sends to the contract the output of $\mathtt{Prove}(.)$, in  C-TLP, whose total size is $628 z$ bits. So, the clouds total bandwidth is about $884  z$ bits and its complexity is $O(z)$ which is independent of and constant in the file size. 
 
 The communication cost of \cite{armknecht2014outsourced} is as follows. The client sends $n$ tags to the server, where the size of each tag is about $128$ bits. So its bandwidth is $128  n$, and its complexity is  $O(n)$. The auditor also sends $n$ tags to the server, where each tag size is also $128$ bits. It also sends the tags to the client along with $zk$ proofs that contain $4n$ elements in total,  where $2n$ of them are elements of $\mathbb{Z}_{\scriptscriptstyle N}$ and each element size is  $2048$ bits, and each of the other $2n$ elements is $160$ bits long. Also, the auditor in \textit{ProveLog} sends $z$ pairs to the client with the  bandwidth of $256  z$. So, the auditor's total bandwidth and complexity is  $4672  n+256  z$ and $O(n+z)$ respectively. Moreover, the cloud sends the entire file, $F$, to the auditor in the store phase and also sends $2  z$ pairs of PoR to the auditor, where each element of the pair is of size $128$ bits. Therefore, the cloud's total  bandwidth is $||{\bm{F}}||+256  z$, while its complexity is $O(||{\bm{F}}||+z)$. Note that the cloud's proof size complexity is constant and independent of the file size, i.e. $O(1)$. Now, we analyse the communication cost of  \cite{xu2016lightweight}. The client bandwidth and complexity are $2048  n$ and $O(n)$ respectively, as it sends to the cloud $2  n$ tags, where each tag size is $1024$ bits. Also, the cloud bandwidth  and complexity are $6144 z$ and $O(z)$ respectively, as for each verification the cloud sends to the verifier $6$ elements each of them is $1024$-bit long. Furthermore, in \cite{DBLP:conf/asiacrypt/ShachamW08} the client bandwidth in the store phase is $128n$, while the server  bandwidth is $256z$. In this scheme, the complexity of a proof size is $O(1)$. In \cite{Storage-Time}, the client can send only the file and random challenge of size $128$-bit to the server, who creates a Merkle tree on top of the file's blocks. Therefore, the client's bandwidth is  $128$-bit. The server sends to the verifier $cz$ PoR proofs that cost it in total  at least $128cz\log(n)$ bits. Also, the server sends VDF's proofs for $z$ outputs, that cost in total $4096z$.  Therefore, the server's total bandwidth is $(128cz\log n)+4096z$. The reason the cost involves $c$ (the number of challenges) is that unlike the other three schemes, this scheme does not support a linear combination of tags/proofs (e.g. homomorphic tags), and in each proving phase $c$ proofs are generated.  Furthermore, in this scheme, the complexity of a proof size is logarithmic with the number of file blocks,  $O(\log(n))$.
 
 To conclude, the verifier-side bandwidth of SO-PoR (and \cite{xu2016lightweight,Storage-Time}) is much lower than \cite{armknecht2014outsourced}. For instance, when $||{\bm{F}}||=1$-GB and $z=100$, a verifier in SO-PoR requires $62\times 10^{\scriptscriptstyle 6}$  fewer bits than the one in \cite{armknecht2014outsourced} does. A client in SO-PoR has a higher bandwidth than it would have in the rest of the protocols. But, this cost is one-off, at the setup phase.  The server-side bandwidth of SO-PoR is the lowest;  for instance (for the same parameters above) a server in SO-PoR requires $9\times 10^{\scriptscriptstyle4}$,  $7$, and $1729$ times fewer bits  than those required in \cite{armknecht2014outsourced}, \cite{xu2016lightweight} and \cite{Storage-Time} respectively.   Moreover, \cite{Storage-Time} has the worst proof size complexity, which is logarithmic to the file size; while the proof size complexity of the rest of the schemes  is constant.  Thus, SO-PoR's server-side bandwidth is significantly lower than the rest  while having constant proof size. 
 
 
 
 

 
 
 
 \begin{remark}
In \cite{armknecht2014outsourced}, the  additional costs   to secure parties against a malicious client stem from only the store phase, where  an auditor downloads the entire file, generates zero-knowledge proofs, and has the client sign them after verifying the proofs. Therefore, the overheads of proving and verifying phases, in this protocol, would remain unchanged if the protocol considers an honest client.  
 \end{remark}
% !TEX root =main.tex


%\section{Cost Comparison between VDF’s and C-TLP}\label{table-VDF-cost-comparison}
%
%Table \ref{table::VDF-comparison}  summarises the result of comparison between C-TLP performance with the two current VDF functions \cite{Wesolowski19,BonehBBF18}. The cost analysis considers the generic setting where $z$ outputs are generated. 
%% !TEX root =main.tex

%\vspace{-2mm}


 \begin{table}[htb]


%\begin{boxedminipage}{\columnwidth}
%\begin{center}
\caption{ \small VDF's Cost Comparison} \label{table::VDF-comparison} 
\begin{footnotesize}
\begin{center}
%{\small
\renewcommand{\arraystretch}{.87}
%\begin{minipage}{1\linewidth}
%\resizebox{\columnwidth}{!}{
{\small
%\scalebox{0.9}{
\begin{tabular}{|c|c|c|c|c|c|c|c|c|c|c|c|c|c|c|} 
   \hline
\cellcolor[gray]{0.9}&\cellcolor[gray]{0.9}&\cellcolor[gray]{0.9}&\multicolumn{3}{c|}{\scriptsize\cellcolor[gray]{.9} \underline{\ \ \ \ \ \ \   Computation Cost  \ \ \ \ \ \ \ }}& \cellcolor[gray]{0.9}\\

 %\cline{4-6}

\cellcolor[gray]{0.9}\multirow{-2}{*} {\scriptsize Protocols}&\cellcolor[gray]{0.9}\multirow{-2}{*}{\scriptsize Model}&\cellcolor[gray]{0.9}\multirow{-2}{*} {\scriptsize Operation}&\cellcolor[gray]{0.9}\scriptsize$\mathtt{Setup}$&\cellcolor[gray]{0.9}\scriptsize$\mathtt{Prove}$&\cellcolor[gray]{0.9}\scriptsize$\mathtt{Verify}$&\cellcolor[gray]{0.9}\multirow{-2}{*}{\scriptsize Proof size (bit)}\\
\hline
  %-----CR-TLP VDF
\cellcolor[gray]{0.9}&\cellcolor[gray]{0.9}&\cellcolor[gray]{0.9}\scriptsize Exp.&\scriptsize$3z+1$&\scriptsize$Tz$ &\scriptsize$2z$& \multirow{2}{*}{\rotatebox[origin=c]{0}{\scriptsize $1524z$}}  \\
     \cline{3-6}  
     
 \cellcolor[gray]{0.9}&\cellcolor[gray]{0.9}\multirow{-2}{*}{\rotatebox[origin=c]{0}{\scriptsize Standard}}&\cellcolor[gray]{0.9}\scriptsize Mul.&\scriptsize$z$&\scriptsize$-$ &\scriptsize$z$&\\
     \cline{2-7}  
\cellcolor[gray]{0.9}&\cellcolor[gray]{0.9}&\cellcolor[gray]{0.9}\scriptsize Exp.  &\scriptsize$z+1$&\scriptsize$Tz$&\scriptsize$-$&\multirow{2}{*}{\rotatebox[origin=c]{0}{\scriptsize $628z$}}  \\
     \cline{3-6}  
\cellcolor[gray]{0.9}\multirow{-4}{*}{\rotatebox[origin=c]{0}{\scriptsize C-TLP}}&\cellcolor[gray]{0.9}\multirow{-2}{*}{\rotatebox[origin=c]{0}{\scriptsize R.O.}}&\cellcolor[gray]{0.9}\scriptsize Mul.&\scriptsize$-$&\scriptsize$-$&\scriptsize$-$&\\
     \cline{2-7}    
     \hline 
     
          \hline 
          %-----Boneh
\cellcolor[gray]{0.9}&\cellcolor[gray]{0.9}&\cellcolor[gray]{0.9}\scriptsize Exp.&\scriptsize$2^{\scriptscriptstyle 30}$&\scriptsize$Tz$&\scriptsize$z$&\multirow{2}{*}{\rotatebox[origin=c]{0}{\scriptsize $2048z$}}\\
 \cline{3-6}  
\cellcolor[gray]{0.9}\multirow{-2}{*}{\rotatebox[origin=c]{0}{\scriptsize   \cite{BonehBBF18}}}&\cellcolor[gray]{0.9}\multirow{-2}{*}{\rotatebox[origin=c]{0}{\scriptsize R.O.}} &\cellcolor[gray]{0.9}\scriptsize Mul.&\scriptsize$-$&\scriptsize$2z\cdot 2^{\scriptscriptstyle 30}$&\scriptsize$2z\cdot 2^{\scriptscriptstyle 30}$& \\
     \cline{2-7} 
      \hline
      
       \hline
       %-----Wesolowski
\cellcolor[gray]{0.9}&\cellcolor[gray]{0.9}&\cellcolor[gray]{0.9}\scriptsize Exp.&$-$&\scriptsize$3Tz$&\scriptsize$3 z$&\multirow{2}{*}{\rotatebox[origin=c]{0}{\scriptsize $4096z$}}\\
\cline{3-6}  
\cellcolor[gray]{0.9}\multirow{-2}{*}{\rotatebox[origin=c]{0}{\scriptsize   \cite{Wesolowski19}}}&\cellcolor[gray]{0.9}\multirow{-2}{*}{\rotatebox[origin=c]{0}{\scriptsize R.O.}}&\cellcolor[gray]{0.9}\scriptsize Mul.&\scriptsize$-$&\scriptsize$Tz$&\scriptsize$-$&  \\
     \cline{2-6} 

 \hline
\end{tabular}
%}
}
%\end{boxedminipage}
%}%--small
\end{center}
\end{footnotesize}
\end{table}

\vspace{-6mm}





\section{More Efficient Proof of Storage-Time}\label{More-Efficient-Proof-of-Storage-Time}

In the following, we show how C-TLP can be used in the PoSt protocols that were proposed in  \cite{Storage-Time} to improve their costs. As stated previously,  two protocols: basic PoSt and compact PoSt, supporting proof of storage-time (in the random oracle) are proposed in \cite{Storage-Time}, where  basic PoSt uses VDF and is publicly verifiable while compact PoSt  uses a trapdoor delay function (TDF) and is privately verifiable. Also, recall that VDF/TDF is used to allow the server to derive multiple challenges at different points over a certain time period $T$. Note, both types of delay function (VDF and TDF) impose the same computation cost to the server, i.e. $3Tz$ modular exponentiations and $Tz$ modular multiplication if the fastest delay function is used \cite{Wesolowski19}. 

Now we show how to replace the delay function in these schemes with C-TLP, in the random oracle.  As in the PoSt protocols, the client at the setup precomputes random challenges (and their PoR tags). But, it encodes all challenges for $z-1$ PoR proofs (excluding first one)  into puzzles using C-TLP. It sends to the server all puzzles, encoded file and plaintext challenges for the first PoR proof. As before, the server generates the first PoR proof using the challenges sent to it in the plaintext. Nevertheless, to find $j\text{\small{-th}}$ challenge to generate  $j\text{\small{-th}}$ PoR proof, the server solves the related puzzle, where $j>1$. It sends to the client all PoR proofs (or a combination of them) after period $T$. Also, if the basic PoSt  is used,  the server sends C-TLP proofs that can be efficiently verified by anyone. On the other hand, if the compact PoSt is used, then the server does not need to send the C-TLP's proofs, as the client already knows the random challenges. The adjustments considerably improve the PoSt protocols' costs. In particular,  the server's computation cost would be $\frac{1}{3}$ of the costs imposed by either of the original PoSt protocols. Also,  there will be $3z$ further reduction in the number of exponentiations:  (a) at the verifier side, in the basic PoSt, as it does not need to perform any exponentiation to check the correctness of C-TLP's output,  (b) at the client-side, in the compact PoSt,  as the client does not need to evaluate TDF at the setup to precompute the challenges which in total involves $3z$ modular exponentiations over $\phi(N)$. Moreover, the proof size would be reduced by a factor of $6.5$.

\begin{remark}
Although the use of C-TLP in the PoSt protocols can reduce the computation and communication costs, (a) the server-side I/O cost in these schemes (i.e. $O(\log n)$ in the basic PoSt and $O(n)$ in the compact one), and (b) PoR's proof size complexity in the basic PoSt, will remain the same. Because these costs stem from the underlying PoR schemes, used as a black box, by the two PoSt protocols.
\end{remark}

\begin{remark}
In general, VDF's offer certain features that (C-)TLP schemes do not offer, e.g. in VDF's parameters $z$  or their outputs do not need to be fixed ahead of time, or they can support multiple users. 
\end{remark}

%%%%%%%%%%%%%%%%%

\end{document}
\endinput
%%
%% End of file `sample-sigconf.tex'.
